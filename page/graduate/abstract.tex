\cleardoublepage
\chapternonum{摘要}

隐匿查询(Private Information Retrieval, PIR)是一种密码学原语。它允许用户从数据库中访问数据而不泄露具体请求的索引,从而保护用户隐私。然而,PIR往往面临巨大的计算负担,需要服务器遍历整个数据库以完成一次查询请求。通过运行离线预处理阶段,PIR可以实现亚线性的在线计算。此外,在如黑名单查询、密码泄露查询等许多PIR场景中,客户端有验证服务器返回的数据是否正确的需求。然而,目前的亚线性PIR协议不支持这类验证。

本文提出了一种简单且高效的亚线性PIR协议,能够保护存在一台恶意服务器时的数据隐私性和完整性。利用这一PIR协议,本文构造了一种基于编码的分布式PIR查询框架,解决了PIR查询中内存瓶颈与分布式容错问题。利用PIR协议通信不会泄露信息且可以使用公开信道传输的特点,本文将提出的PIR协议与框架同区块链应用结合,利用智能合约,实现了公开的验证逻辑。

实验表明,就在线计算效率而言,本文提出的PIR协议优于最先进的两服务器亚线性PIR协议与单服务器亚线性PIR协议,并且在运行成本上具有一定优势。相对于最先进的可验证PIR协议,本文的协议在查询性能上也有较大提升。本文提出的区块链PIR应用具有较高的性能,在包含三百万黑名单的数据库中查询时,能够提供超过6000的QPS,适合大规模查询服务的需求。

\bigskip
\noindent \textbf{关键词}:隐匿查询、区块链、应用密码学、隐私保护
\cleardoublepage
\chapternonum{Abstract}