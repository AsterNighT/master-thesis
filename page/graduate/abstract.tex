\cleardoublepage
\chapternonum{摘要}

隐匿查询(Private Information Retrieval, PIR)是一种密码学原语。它允许用户从数据库中访问数据而不泄露具体请求的索引,从而保护用户隐私。然而,PIR方案往往面临巨大的计算负担,需要服务器遍历整个数据库以完成一次查询请求。通过运行离线预处理阶段,PIR方案可以实现亚线性的在线计算。此外,在如黑名单查询,密码泄露查询等许多PIR场景中,客户端有验证服务器返回的数据是否正确的需求。然而,目前的亚线性PIR协议不支持对与服务器答案的验证。

本文提出了一种简单且高效的亚线性PIR方案,能够确保在存在一个恶意服务器的情况下的数据隐私性和完整性。就在线计算效率而言,本文方案比最先进的两服务器方案快7倍,比单服务器亚线性PIR方案快6倍。相对于领先的可验证PIR方案,我们的方案展现了大约500倍的效率提升。

PIR协议通信不会泄露信息,可以以公开的信道传输。同时,对于请求的公开验证也有助于去中心化环境中服务提供者与用户声誉的形成。因此,本文的协议可以很好地与区块链场景结合。针对这一应用场景,本文提出了一套基于纠删码的分布式PIR查询框架,解决了PIR查询中内存瓶颈与分布式容错问题。更进一步地,利用智能合约,本文实现了相应的公开验证与激励机制,充分利用了区块链的不可篡改性质。实验表明,本文提出的PIR方案在区块链场景中具有较高的性能,在包含三百万黑名单的数据库中查询时,能够提供超过6000的QPS,适合大规模查询服务的需求。

\bigskip
\noindent \textbf{关键词}:隐匿查询、区块链、应用密码学、隐私保护
\cleardoublepage
\chapternonum{Abstract}