\section{解决方案}
针对上述问题,本体提出了一套与之相对应的解决方案。具体来说,本文提出了一套高性能的亚线性PIR协议。此协议可以以一个较低的代价生成计算证明,从而保证服务的正确性。同时,本文提出了一套可拓展的PIR框架,使得PIR服务能够容易地进行横向扩容与容灾。最后,本文探讨了这套框架PIR在区块链中的应用。
\subsection{高性能可验证的PIR协议}
本文分析了前人工作中“虚构查询”的问题,针对服务正确性问题,提出了一种无需虚构查询的亚线性复杂度PIR协议。这种协议基于亚线性复杂度PIR的框架,改进了在线查询的方式,优化了具体参数,使得在线查询的效率相较于目前行业领先的工作 \cite{TreePIR,Piano} 提高了6-7倍。本文提出的协议可以高效地生成计算证明,同时保护用户的隐私。本文提供了这种协议的多服务器构造与单服务器构造,同时给出了一种兼具多服务器构造与单服务器构造优点,在在线计算中无需多服务器参与的简便执行模型。
\subsection{可拓展的PIR框架}
针对PIR的存储与拓展性问题,本文提出了一种基于编码的PIR框架,允许服务提供者将数据库拆分成数个较小型的数据库,并将这些小型数据库分别部署于不通的物理设备上,充分利用多台设备的空间。同时,通过纠错码的纠错能力,这一框架允许服务提供者通过冗余保障PIR服务的可用性。本文同时分析了底层PIR协议的可验证性对于这一框架的影响,使用纠删码替代纠错码以进一步提高这种编码的效率。

\subsection{PIR于区块链中的应用}
本文将这一框架运用于“\projectname”这一课题中,对于黑名单匹配,用户隐私保护等问题提出了一种基于PIR的解决方案。本文同时提出了一种PIR的区块链激励模型,允许委托者,服务提供者,用户三方面协同,为用户提供高可用,高效,低延迟的PIR服务。在这套模型中,委托者首先与服务提供者通过链上沟通达成协议,用户通过区块链上的智能合约进行请求,服务提供者通过智能合约提供服务并获得激励。PIR公开,抗监听的特性使得这一模型与区块链天然地契合。通过智能合约,用户可以公开地验证服务的正确性,使请求完成的过程原子化,保护用户与服务提供者的利益。