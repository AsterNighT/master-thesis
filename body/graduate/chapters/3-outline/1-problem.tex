\section{PIR实际应用的难点}

在绪论中,我们提到了PIR在实际应用中所面临的性能、拓展性和可靠性难题。本章将详细讨论这三个问题。
\subsection{PIR的性能问题}
相较于传统数据库的查询,PIR的效率存在着显著差距。以目前最先进的线性复杂度PIR为例,SimplePIR \cite{SimplePIR} 在数据库大小为8GiB时,一次查询所需的通信量约为700 KiB,计算时间约为800ms。而非隐匿的普通数据库查询仅需要不到1 KiB 的通信和远低于1 ms 的计算时间(实际上,查询的瓶颈在于硬盘读写速度而非计算量)。同时,PIR的效率对于查询的数据库大小非常敏感,而传统数据库查询在进行类似PIR的单点查询时几乎不受到数据库大小的影响。主要原因在于PIR要求保护查询者的隐私,服务提供者不能获取查询的具体位置,因而无法像传统数据库查询那样利用索引加速查询。暂不考虑硬盘-内存的带宽瓶颈,可以认为,通过索引加速的普通数据库单点查询的复杂度总是 $O(\log\dbsize)$,而线性复杂度的PIR的复杂度为 $O(\dbsize)$。对数复杂度与线性复杂度之间的鸿沟造成了性能上的巨大差距。

这种巨大的性能差距使得PIR在实际应用中面临巨大挑战。一方面,长时间的计算使得查询延迟变得非常高,对于许多对实时性要求较高的应用,如UI渲染和DNS请求,这是不可接受的。另一方面,高通信量和高计算量也导致了服务提供商的成本大幅增加。综上所述,为了使得PIR能够在实际应用中发挥作用,迫切需要开发出更加高效的PIR算法。

\subsection{PIR的存储和拓展性问题}
在某些PIR协议、尤其是基于同态加密的PIR协议中,服务器会保存编码后的数据库。与原始数据库相比,编码后的数据库会出现一定程度的体积膨胀。SimplePIR\cite{SimplePIR}将 8GiB 大小的数据库编码后,其所需存储空间达到了 40GiB。这种体积膨胀导致了数据库存储成本的上升。PIR面临的另一个挑战加剧了这一问题:由于查询间的关联性会暴露隐私,PIR查询要求彻底消除所有查询的局部性。因此,缓存机制几乎对PIR效率的提升没有帮助。在线性复杂度的PIR协议中,服务器需要一次性将整个数据库从硬盘读入内存以进行计算,这会成为一个严重的瓶颈——以 AWS GP3 存储服务\cite{AWSEBSGP} 为例,其提供的硬盘最大吞吐量仅为 1000 MiB/s,因此,读入内存一个容量为 40GiB 的数据库需要 40s。而CPU计算的吞吐量超过 10000 MiB/s,远远超过了硬盘读取速度。因此,大部分时间都会花在等待硬盘I/O上。当前,大部分PIR协议并没有考虑这一问题,而是假设服务器可以在内存中存储整个数据库。然而,这一假设在实际应用中并不现实。随着数据库规模不断增大,服务器的内存容量会迅速耗尽。

在一些PIR研究工作中,存在关于是否应将“客户端直接存储整个数据库”作为一种解决方案的讨论。然而,在我们所研究的问题中,直接存储整个数据库是不切实际的。考虑到数据库的可扩展性,客户端有限的存储空间将成为首要瓶颈。需注意的是,通常客户端(如手机、个人电脑、物联网设备等)的内存和存储空间相对于服务器非常有限,PIR协议不应对客户端造成过大的计算和存储压力。实际应用中,拥有数 TiB 大小的数据库并不罕见,而现有的PIR协议无法处理这样的数据库规模。为了使PIR能够处理大规模数据库,我们需要一种能够横向扩展的PIR算法。

\subsection{云服务带来的可靠性问题}
讨论“云服务”等概念时,我们不可避免地要面对一个问题:如何保证服务的可靠性?服务的承接者可能并非诚实。一个不诚实的服务提供者可能会故意提供错误的结果以误导用户或节约计算量。即使是诚实的服务提供者,也可能因为硬件软件故障而导致结果出错。在传统的中心化服务中,这一问题相对较易解决:服务提供者可以通过监控、审计等手段来保证服务的可靠性。从另一个角度来看,服务的可靠性受到服务提供者信誉的保障。然而,在区块链等去中心化服务中,这一问题变得更加困难,服务提供者可能是匿名的,监控、审计等手段无法直接应用。

我们强调,在这样的环境中,保证可靠性不仅仅是用户的需求,也是委托者、服务提供者和用户三方的共同要求。对于委托者而言,可靠性是其委托的基础。如果没有可靠性,其与服务提供者的协议无法成立。对于服务提供者而言,可靠性是其获得报酬的凭证,服务提供者通过证明其提供了正确的服务来获取报酬。对于用户而言,可靠性保证使其能正常使用服务。因此,在去中心化服务中,保证服务的可靠性是一个迫切需求。

目前的项目中,这一问题通常通过可验证计算来解决。即,服务提供者向用户提供一个计算证明,证明其计算的可靠性。然而,这一方法往往需要额外计算,从而导致服务效率下降。朴素的证明,例如非交互式零知识证明\cite{10.1007/3-540-46766-1_35}(Non-interactive Zero-Knowledge Proof,NIZK),生成证明的过程往往比计算本身慢数个数量级。如何在保证服务可靠性的同时不影响服务效率是一个棘手的问题。

在前人的工作 \cite{APIR} 中,一种名为“选择失败攻击”的攻击方式被引入了PIR。这种攻击方式利用验证计算证明的过程来获取用户的隐私。服务提供者可能故意损坏数据库中的某些记录,以使用户只能在访问部分数据时获得正确结果。一旦服务提供者获取用户的验证结果,就能推断用户是否在访问完好数据。这种攻击使得传统的利用数字签名、消息认证码(Message Authentication Code,MAC)等方法来保证数据库记录可靠性的方式不再适用。

此外,我们意识到去中心化服务带来的分布式特性也凸显了服务可用性问题。在分布式服务中,保证服务的可用性至关重要。为了在硬件损坏、网络波动等情况下仍然保证用户能够持续访问服务,传统的分布式集群采用冗余备份等方式来保证服务的可用性,通过将服务切换至备用节点来屏蔽故障节点。然而,在去中心化环境中,缺乏中心化的协调与调度机制,我们需要更可靠的手段来确保服务的正常运行。