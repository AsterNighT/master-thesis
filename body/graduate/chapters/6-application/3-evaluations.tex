\section{区块链应用测试}

{问移动那边要点数据规模,然后把具体的性能指标填进去,可以在邦盛给的两台服务器上跑一下,再叠一下移动设备存储空间的buff。}
\subsection{实验设计}
基于区块链实际应用,我们对黑名单查询功能进行了测试。我们在一包含三百万条恶意网址的数据库上进行测试,数据库包含这些网址的256位哈希值。实验采用类似Checklist\cite{Checklist}的方案来将网址映射到数据库索引中。实验的平台与具体运行方式已经在 \ref{sec:evaluation} 中给出,此处不在重复。我们使用了第 \ref{sec:pir-framework} 节中提出的PIR框架作为基础,采用 \ref{fig:two-server-verify} 节中的协议作为PIR协议,并加入第 \ref{sec:optimized-model} 节中提到的优化。作为比较项,我们使用VeriSimplePIR\cite{VeriSimplePIR}作为底层PIR协议运行了另一组实验。VeriSimplePIR无法进行公开验证,我们忽略了这一步骤。我们取8个节点作为在线查询服务器,容错阈值为节点总数的一半,采用了Hyperchain\cite{hyperchain}作为区块链底层平台,并部署了智能合约。

\subsection{结果分析}
\begin{table}[]
    \caption{不同协议查询QPS对比}
    \centering
    \label{tab:blockchain-query-qps}
    \begin{tabular}{@{}c|c|c@{}}
    \toprule
    方案            & 吞吐量(qps) & 优化比例    \\ \midrule
    本文            & 6040.57  & 13042\% \\
    VeriSimplePIR & 45.97    & -       \\ \bottomrule
    \end{tabular}
\end{table}

能够明显对比看出,本文提出的亚线性PIR协议极大的提高了协议的吞吐量,使服务器能在相同时间内处理更多的查询。这一结果与我们的预期相符,说明了我们的优化方案的有效性。此外,实际测试表明,基于区块链设置的PIR协议在实际应用中具有较高的性能,能够满足实际使用需求。