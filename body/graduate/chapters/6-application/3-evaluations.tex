\section{区块链应用测试}
\subsection{实验设计}
针对区块链实际应用,本研究对黑名单查询功能进行了测试。实验使用了一个包含三百万条恶意网址的数据库,其中存储了这些网址对应的256位哈希值。在实验设计中,采用了类似Checklist\cite{USENIX:KogCor21}的方法,将网址映射到数据库索引中,但没有使用本地哈希表来过滤任何查询。本实验基于第 \ref{sec:pir-framework} 节提出的PIR框架,并采用了 \ref{fig:two-server-verify} 中的协议作为PIR协议。安全级别至少设定为 $\lambda=128$,支持至少 $\querycount=\sqrt{\dbsize}\log{\dbsize}$ 次查询(其中$\dbsize$为数据库中的记录数),根据参数$\dbsize$、$\lambda$ 和$\querycount$进行计算。所有测试在单线程上运行,并对10次运行结果取平均值。实验在配备16核Intel(R) Xeon(R) E-2288G CPU @ 3.70GHz的服务器上进行。由于计算瓶颈在服务提供者一方,实验通过代码批量运行查询,跳过了浏览器的调用部分。作为比较,VeriSimplePIR\cite{VeriSimplePIR}作为底层PIR协议进行了另一组实验。由于VeriSimplePIR无法进行公开验证,该步骤被省略。在在线查询服务器方面,选择了8个节点,容错阈值为节点总数的一半;底层平台采用了Hyperchain\cite{hyperchain},并部署了智能合约。

\subsection{结果分析}

\begin{table}[]
    \caption{不同协议查询QPS对比}
    \centering
    \label{tab:blockchain-query-qps}
    \begin{tabular}{@{}c|c|c@{}}
    \toprule
    方案            & 吞吐量(qps) & 优化比例    \\ \midrule
    本文            & 6040.57  & 13042\% \\
    VeriSimplePIR & 45.97    & -       \\ \bottomrule
    \end{tabular}
\end{table}

通过对比表\ref{tab:blockchain-query-qps}中的实验结果可以明显发现,本文提出的亚线性PIR协议显著提高了协议的吞吐量,使服务器能够在相同时间内处理更多的查询。这个结果与预期一致,验证了本文优化方案的有效性。实际测试表明,在吞吐量方面,该协议在实际应用中具备较高性能,能够满足实际的使用需求。

