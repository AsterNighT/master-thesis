\section{区块链的问题背景}

\subsection{课题内容}
{从课题大纲里摘取一点来}
推进国家治理体系和治理能力现代化,是党的十八届三中全会提出的全面深化改革总目标之一。近年来,新冠疫情、洗钱、电信诈骗等社会风险呈高发态势,健康码乱赋红码、电信诈骗、跨境洗钱等案件均反映出当前社会治理与风险防控在可信性、实时性、精准度、智能化等方面存在不足。如何基于区块链、大数据、人工智能等技术,实现高可信、高性能、高精准与智能化的社会治理和风险防控成为当前紧迫任务。

社会治理与风险防控需同时满足可信、实时、精准、智能的多目标需求,然而,区块链、大数据、人工智能等技术在实时性、可信性、智能化等方面分别存在不足。此外,面对不断出现的社会风险,现有系统平台存在数量众多、功能零散化的特点,缺少统一平台框架。如何深度融合区块链、大数据、人工智能、物联网等技术,构建多技术融合的社会治理与风险防控技术体系、面向数字化社会治理的可信服务支撑技术,研制支持快速构建社会风险防控应用的共性基础平台,突破社会治理与风险防控多目标需求同时满足的挑战,创建社会治理和风险防控新范式,是当前亟待解决的共性基础关键问题。

\subsection{具体应用}
{和后面具体的场景对应,比如黑名单查询的背景}
本文提出的技术主要用于“基于区块链的社会风险防控技术体系及可信服务技术”这一方向。具体地,我们将PIR技术应用于黑名单查询场景。黑名单查询是社会风险防控的重要环节,通过黑名单查询,可以有效地防范风险。然而,黑名单查询涉及大量的隐私数据,例如用户通话对象手机号,身份证号等。如何在保护隐私的前提下进行高效的黑名单查询是一个重要的问题。

针对查询内容需要保密但又需要允许进行查询的需求,通过链上部署智能合约,节点调用隐匿查询算法,实现对黑名单的查询。这一过程中,智能合约不会泄露查询的具体内容,保护了用户数据的隐私。
