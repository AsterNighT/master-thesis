\section{应用场景分析}

本文提出的技术主要应用于“基于区块链的社会风险防控技术体系及可信服务技术”这一领域。具体来说,我们将PIR技术应用于黑名单查询场景,黑名单查询是社会风险防控的重要环节。域名、手机号、身份证号等信息数据的黑名单查询可以在第一时间阻止有风险的数据访问与互动。然而,黑名单查询涉及大量的隐私数据,例如通话对象的手机号、用户身份证号等。在保护隐私的前提下进行高效的黑名单查询是一个重要的问题。

目前,黑名单查询主要是通过明文查询完成的。以常见的反电话诈骗为例,中国信通院联合电信企业建立了码号服务推进组\cite{e164},提供了号码标记查询服务。用户提供明文号码即可查询号码是否被标记为诈骗电话。许多个人手机设备厂商在操作系统内集成了相应功能,当用户接听电话时,系统会自动查询号码是否为诈骗电话。然而,厂商可以通过这些查询获取用户的来电情况,实质上构成了厂商对用户通话对象的监控。即便厂商并非恶意,用户也很难从技术上获得安全感。

苹果公司在2024年10月推出了Live Caller ID Lookup功能\cite{apple_live_caller_id},该功能允许开发者部署私有数据库,并通过代理服务、匿名ID和PIR等技术为用户提供完全隐私的号码查询服务。这一功能为PIR实际应用提供了一个很好的示例。然而,其运行效率受限于PIR协议的性能以及私有服务器的可靠性。根据官方文档提供的数据\cite{apple_parameter_tuning},在大小为60MiB的数据库上运行一次查询大约需要610毫秒的时间。这一性能对于实时查询而言不够理想。
