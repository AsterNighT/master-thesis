\section{解决方案}
针对黑名单匹配这一问题,本文提出了一套使用区块链节点委托计算的解决方案。在这一方案中,服务提供商使用本文提出于第\ref{sec:optimized-model}节中提出的优化多服务器模型提供服务。具体算法部分,双方采用Reed-solomon编码分片的,基于图\ref{fig:single-server}中算法的PIR协议。
\subsection{协议流程}
如图\ref{fig:blockchain-scheme}所示:
\begin{figure}
    \begin{mdframed}
        \centering
        \textbf{基于区块链的PIR协议}

        \raggedright
        初始化阶段:
        \begin{itemize}
            \item 服务提供商选择将数据库摘要公开,节点向服务提供商请求参与协议。服务提供商根据数据库大小和容灾需要选择合适的纠删码方案,并将数据库分片分发至节点。
            \item 服务提供商选择离散对数困难的群生成元$g$。
        \end{itemize}
        离线阶段:
        \begin{itemize}
            \item 客户端与服务提供商运行离线阶段PIR协议,获得所需的Hint与Crumb。
        \end{itemize}
        在线阶段:
        \begin{itemize}
            \item 客户端在区块链发起交易。
            \item 节点发现交易并开始计算PIR查询,将答案上传至区块链。
            \item 客户端验证答案的正确性。如果能够从有效的数据中恢复出正确的结果,则客户端接受答案。
        \end{itemize}
    \end{mdframed}
    \caption{基于区块链的PIR协议}
    \label{fig:blockchain-scheme}
\end{figure}

不难注意到,算法流程的核心是保障以下几点:
\begin{enumerate}
    \item 当且仅当节点给出的答案是正确的,客户端通过验证。
    \item 当且仅当通过验证的节点数量超过恢复阈值,客户端接受答案。
    \item 如果节点没有及时给出回复,应判断节点给出了不通过验证的答案。如果客户端没有及时给出回复,应认为客户端放弃了查询。
\end{enumerate}

这些要求都可以由智能合约实现。我们使用solidity语言编写了相应的逻辑,其核心在于,验证答案这一过程是可以公开进行的。具体地:
\begin{enumerate}
    \item 记图\ref{fig:single-server}算法中$Eval(\randomset_\hintidx, \blockidx)$的值为$e$,客户端将$e$Commit至交易中。同时,客户端也将$g^{\randomhint_\hintidx+r\cdot \crumbvalue_{\blockidx, \crumbidx}-e(\sumhint_\hintidx+\crumbvalue_{\blockidx, \crumbidx})}$的值上传至合约内。
    \item 节点提交$\sumanswer,\randomanswer$
    \item 确定答案后,客户端公开$e$的值,并验证是否有$g^{\randomhint_\hintidx+r\cdot \crumbvalue_{\blockidx, \crumbidx}-e(\sumhint_\hintidx+\crumbvalue_{\blockidx, \crumbidx})}=g^{\randomanswer-e\sumanswer}$
\end{enumerate}

这一过程的正确性是显然的。具体地算法代码如下:
\begin{lstlisting}[language=Solidity]
    // SPDX-License-Identifier: MIT
    pragma solidity ^0.8.0;
    
    contract CommitmentService {
        // 状态变量
        address public client;                 // 客户的地址
        uint256 public n;                      // 提交的数字和服务提供者的数量
        uint256 public t;                      // 允许失败的最大数量
        uint256 public commitEndTime;          // 提交阶段结束的时间
        uint256 public revealEndTime;          // 客户公开已提交元组的截止时间
        uint256 public lockedAmount;           // 客户锁定的资金
        bool public revealed;                  // 客户是否公开了已提交的元组
    
        uint256 public g;                      // 群的生成元
        uint256 public e;                      // 客户公开的e值
    
        // 承诺的元组:每个元组包含一个群元素h和一个承诺commit c
        struct Tuple {
            uint256 h;      // 群上的元素h
            bytes32 c;      // 承诺的commit(哈希值)
        }
    
        Tuple[] public tuples;                 // 客户承诺的n个元组
        address[] public servers;              // n个服务提供者的地址
    
        // 提交的ai, bi
        struct Submission {
            uint256 ai;
            uint256 bi;
            bool submitted;
        }
    
        mapping(address => Submission) public submissions; // 每个服务提供者提交的ai, bi
        uint256 public submissionCount;                    // 已提交的服务提供者数量
    
        // 事件
        event Commit(address indexed client, address[] servers, uint256 amountLocked);
        event Submit(address indexed server, uint256 ai, uint256 bi);
        event Reveal(address indexed client, uint256 e);
        event DistributeFunds(uint256 successCount, uint256 amountPerWinner);
        event RefundClient(uint256 amount);
        event TimeoutDistribute(uint256 amountPerServer);
        event RevealFailed(address indexed client, string message);
    
        // 修饰符
        modifier onlyClient() {
            require(msg.sender == client, unicode"不是客户");
            _;
        }
    
        modifier onlyDuringCommitPhase() {
            require(block.timestamp <= commitEndTime, unicode"提交阶段已结束");
            _;
        }
    
        modifier onlyDuringRevealPhase() {
            require(block.timestamp > commitEndTime && block.timestamp <= revealEndTime, unicode"不在公开阶段");
            _;
        }
    
        modifier afterRevealPhase() {
            require(block.timestamp > revealEndTime, unicode"公开阶段尚未结束");
            _;
        }
    
        constructor(
            uint256 _n,
            uint256 _t,
            uint256 _commitDuration,
            uint256 _revealDuration,
            uint256 _g,                 // 群的生成元g
            Tuple[] memory _tuples,     // Memory数组,含n个元组
            address[] memory _servers
        ) payable {
            require(_n == _tuples.length && _n == _servers.length, unicode"元组或服务提供者数量无效");
            require(msg.value > 0, unicode"必须锁定一些资金");
    
            client = msg.sender;
            n = _n;
            t = _t;
            g = _g;  // 设置群的生成元g
            commitEndTime = block.timestamp + _commitDuration;
            revealEndTime = commitEndTime + _revealDuration;
            
            // 将memory中的_tuples逐个复制到storage中的tuples
            for (uint256 i = 0; i < _tuples.length; i++) {
                tuples.push(Tuple({
                    h: _tuples[i].h,
                    c: _tuples[i].c
                }));
            }
    
            servers = _servers;
            lockedAmount = msg.value;
    
            emit Commit(client, servers, msg.value);
        }
    
        // 服务提供者提交他们的ai和bi
        function submit(uint256 _ai, uint256 _bi) external {
            require(block.timestamp <= commitEndTime, unicode"提交阶段已结束");
            require(submissions[msg.sender].submitted == false, unicode"已提交过");
    
            // 验证提交者是否是有效服务提供者
            bool isValidServer = false;
            for (uint256 i = 0; i < servers.length; i++) {
                if (servers[i] == msg.sender) {
                    isValidServer = true;
                    break;
                }
            }
            require(isValidServer, unicode"不是有效服务提供者");
    
            submissions[msg.sender] = Submission(_ai, _bi, true);
            submissionCount++;
    
            emit Submit(msg.sender, _ai, _bi);
        }
    
        // 客户公开之前承诺的e值
        function reveal(uint256 _e) external onlyClient onlyDuringRevealPhase {
            e = _e;
            revealed = true;
    
            emit Reveal(client, _e);
    
            // 评估提交情况并进行资金分配
            _evaluateSubmissions();
        }
    
        // 评估所有服务提供者的提交,并决定资金分配或退款
        function _evaluateSubmissions() internal {
            uint256 successCount = 0;
            uint256 failCount = 0;
    
            for (uint256 i = 0; i < n; i++) {
                address server = servers[i];
                Submission memory submission = submissions[server];
                Tuple memory tuple = tuples[i];
    
                // 如果服务提供者未提交,默认提交(0, 0)
                if (!submission.submitted) {
                    submission = Submission(0, 0, true);
                }
    
                // 验证g^(b_i - e * a_i) 是否等于 h_i
                if (_powMod(g, submission.bi - e * submission.ai) == tuple.h) {
                    successCount++;
                } else {
                    failCount++;
                }
            }
    
            // 分配资金或退款
            if (failCount <= t) {
                uint256 amountPerWinner = lockedAmount / successCount;
                for (uint256 i = 0; i < n; i++) {
                    address server = servers[i];
                    Submission memory submission = submissions[server];
                    Tuple memory tuple = tuples[i];
                    if (_powMod(g, submission.bi - e * submission.ai) == tuple.h) {
                        payable(server).transfer(amountPerWinner);
                    }
                }
                emit DistributeFunds(successCount, amountPerWinner);
            } else {
                payable(client).transfer(lockedAmount); // 如果失败数量超过阈值,退款给客户
                emit RefundClient(lockedAmount);
            }
        }
    
        // 用于计算群的指数运算(g^x mod p)
        function _powMod(uint256 base, uint256 exponent) internal pure returns (uint256) {
            return base ** exponent;  // 在此处假设使用一个安全的大整数库或实现
        }
    
        // 如果客户未能按时公开,平分资金给所有服务提供者
        function distributeIfNotRevealed() external afterRevealPhase {
            require(!revealed, unicode"已公开");
    
            uint256 amountPerServer = lockedAmount / n;
            for (uint256 i = 0; i < n; i++) {
                payable(servers[i]).transfer(amountPerServer);
            }
    
            emit TimeoutDistribute(amountPerServer);
        }
    }
\end{lstlisting}

由于区块链公开且不可篡改,恶意的客户端或是服务器行为都会被记录在区块链上。服务提供者与客户端都可以根据信用记录进行筛选,使恶意的节点与客户端无法参与到协议中。

\subsection{节点激励模型}
上文的协议要求客户端首先将一笔代币预支付于合约中。同时,合约也保证,只要失败的查询个数不超过阈值,正确回答问题的节点就能得到报酬。因此,节点通过回答客户端的请求获益,这激励了节点积极参与这一协议。