\section{单服务器模型}
本节旨在将之前的协议修改为适用于单服务器可验证PIR协议。我们假设数据所有者将数据库外包给单个服务器,并生成一个数据库摘要,客户端可以利用此摘要验证查询结果。为了实现这一修改,我们需要解决两个主要问题:
\begin{enumerate}
    \item \textbf{获取Hint:} 在双服务器设置中,Hint服务器负责向客户端提供Hint。然而在单服务器设定中,我们不能让查询服务器完成这一任务。如果在线查询的服务器同时也知道离线生成的Hint,它可以通过比对Hint和查询来推断客户端正在查询的索引。
    \item \textbf{更新Hint:} 为了让客户端能够执行多次查询,协议必须支持Hint更新。然而,在原始协议中,查询和Hint更新是由两台不同的服务器执行的,这同样会导致在线查询的服务器同时知道Hint。
\end{enumerate}

此外,还需要考虑客户端如何确保数据库交由不可信的服务器托管时的正确性。接下来的部分将专注于解决上述问题。

\subsection{修改为单服务器的协议}
\paragraph{使用备用Hint更新}
为了在在线阶段更新Hint,我们采用了备用Hint的做法 \cite{EC:CorHenKog22}。客户端在离线阶段获取备用的Hint。这些备用Hint用于更新在线阶段消耗的Hint。

我们将Hint分为两类:主Hint和备用Hint。在离线阶段,客户端获取$\hintcount_1$个主Hint。除此之外,客户端还为每个块获取$\hintcount_2$个备用Hint。块$\blockidx$的备用Hint在第$\blockidx$块中留空。换句话说,块$\blockidx$的备用Hint中不包含块$\blockidx$中的索引。客户端按照双服务器协议使用主Hint向服务器发出查询。一旦完成对块$\blockidx$中某个索引的查询,客户端获取块$\blockidx$的备用Hint,将查询的索引加入该备用Hint中,填补块$\blockidx$中缺失的部分并更新校验信息。这一过程完成后,该备用Hint转化为主Hint。

Crumb也遵循类似的模式。客户端在离线阶段为每个块保留$\hintcount_2$个备用Crumb,并在查询后从备用中获取一个对应块的新Crumb。

可以证明,若为每个块分配了$3\magicnumber$个备用Hint和Crumb,该协议能够让客户端在一个离线阶段后以极高的概率成功执行至少$\magictotal$次查询。在此之后,双方重新运行离线阶段以补充Hint与Crumb。

\paragraph{通过流式传输数据库获取Hint}
从现有工作来看,在单服务器环境下中,离线获取Hint可以通过两种方式实现:(i)  使用同态加密来计算Hint\cite{EC:CorHenKog22},(ii) 将整个数据库流式传输到客户端\cite{CCS:PatPerYeo18, Piano}。

二者中流式传输在实际运行中更为高效,因此我们采用了这一方式,二者的详细讨论见第\ref{sec:offline-discussion}节。数据库按前文所述方法划分$\sqrt{\dbsize}$个块。客户端首先生成所有需要的Hint集合,并从服务器请求$\sqrt{\dbsize}$个块中的一个,更新所有本地Hint。具体来说,当处理第$j$个块时,对于具有校验值$\hint_\hintidx$和PRF密钥$\setkey_\hintidx$的Hint,客户端计算$\hint_\hintidx \coloneqq \hint_\hintidx+\db_{Eval(\setkey_\hintidx,j)}$。如果Hint不应该包含该块中的任何索引(它是一个此块的备用Hint),则跳过该Hint。所有Hint更新完成后,客户端丢弃这个块并从服务器获取一个新块。

\paragraph{引入验证}
利用流式传输的方式获取Hint时,我们要求数据库所有者提供数据库的摘要,这与前人工作 \cite{APIR} 中的假设一致。该摘要可以通过签名发布在区块链等可信平台上。在执行在线查询之前,客户端需要根据此摘要验证流式传输的数据库,以确保数据库内容的真实性和完整性。

到目前为止,我们已经解决了将双服务器可验证PIR协议转换为单服务器协议所面临的主要问题。但在完整给出我们的单服务器协议之前,还有一些重要的辅助手段需要介绍。这些技巧在正确性上至关重要:
\begin{itemize}
    \item \textbf{支持$\querycount = \Omega(\sqrt{\dbsize})$次查询就能支持多项式数量的查询:} 在进行$\querycount$次查询后,协议双方可以重新执行离线阶段。只要协议能在$O_\lambda(\recordsize\dbsize)$时间内完成离线阶段,客户端将这一复杂度均摊到$\querycount$次查询上时,查询的复杂度就是亚线性的。这一离线-在线的循环过程可以支持多项式数量的查询。

    \item \textbf{查询不会重复:} 不失普遍性地,我们可以假设在$\querycount$ 次查询中没有重复查询。我们可以令客户端使用额外$\Theta(\recordsize \querycount)$的存储空间缓存最近的$\querycount$个查询结果。如果客户端需要提出一个重复查询,它可以随机查询一个未查询过的数据库记录,并从缓存结果中检索需要的信息。

    \item \textbf{查询的分布是均匀的:} 不失普遍性地,我们假设查询的索引在数据库中是均匀的。数据库所有者可以根据伪随机置换(PRP)密钥对数据库条目进行洗牌。我们还可以假设查询索引是在不知道这一置换密钥的情况下选择的。由于我们先前假设了查询中没有重复,每个查询索引将被映射到洗牌后数据库中的随机索引,并落入一个随机块。
\end{itemize}

总结上述内容,我们在下图中展示了我们的单服务器可验证PIR协议,其中验证过程以蓝色标记突出显示。

% \begin{figure*}
    \begin{mdframed}
    \centering
    \textbf{可验证单服务器协议}
        \raggedright
        \paragraph{符号约定:} 协议包含一个客户端 $\client$ 与一台Query服务器 $\queryserver$。单个Hint由元组:$\hint=(\setkey,x,\sumhint,\bluetext{\randomset,\randomhint})$构成。单个Crumb包含了一个偏移量和一条记录值 $\crumb=(\crumbvalue, \crumboffset)$。$f_{\setkey}: [\sqrt{\dbsize}]\to [\sqrt{\dbsize}]$ 是一个将块序号转化为偏移量的PRF,\bluetext{$fr_{\randomset}: [\sqrt{\dbsize}]\to \recordfield$ 是一个将块序号转化为 $\recordfield$上的随机元素的PRF。} $Eval(\setkey,\dbidx) \coloneqq f_{\setkey}(\dbidx)\oplus \shift + \dbidx\cdot\sqrt{\dbsize}$ 与 $Expand$ 密钥 $\setkey$ 标识计算密钥对应的集合内元素 $\{Eval(sk,j) \mid j\in[\sqrt{\dbsize}]\}$。 \bluetext{$Eval(\randomset,\dbidx) \coloneqq fr_{\randomset}(\dbidx)$ 与 $Expand$ 密钥 $\randomset$ 表示计算集合元素 $\{Eval(sr,j) \mid j\in[\sqrt{\dbsize}]\}$。} 假设 $\dbidx$ 是需要查询的索引。\bluetext{可信的数据库所有者 $\owner$ 计算数据库 $\db$ 的摘要 $\digest$ 并且将其发送给 $\client$。} 在离线阶段,生成 $\hintcount_1$ 个主Hint与每块 $\hintcount_2$ 个备用Hint。

        \paragraph{离线阶段:}
        \begin{itemize}
            \item \textbf{Setup:} $\client$ 生成主PRF密钥 $\setkey_j, j\in[\hintcount]$,备用PRF密钥 $\setkey_{k,j}, j\in[\lambda], k\in[\sqrt{\dbsize}]$ \bluetext{与 $\randomset_j, j\in[\hintcount]$ 及其备用密钥 $\randomset_{k,j}, j\in[\lambda], k\in[\sqrt{\dbsize}]$},将本地的主Hint存储初始化为 $\hint_j\coloneqq(\setkey_j,\bot,0,\bluetext{\randomset_j,0}), j\in[\hintcount]$,备用Hint初始化为 $\hint_{k,j}\coloneqq(\setkey_{k,j},\bot, 0,\bluetext{\randomset_{k,j},0}), j\in[\lambda], k\in[\sqrt{\dbsize}]$,Crumb存储初始化为$\crumb_{k,j} \coloneqq (\bot,\bot),j\in[\lambda], k\in[\sqrt{\dbsize}]$。
            \item \textbf{Hint:} 服务器 $\queryserver$ 将数据库 $\db$ 传输给 $\client$。\bluetext{$\client$ 初始化摘要 $\digest'$.} 当传输块 $l$ 时, $\client$ 按如下方法更新Hint:
                  \begin{itemize}
                      \item 更新主Hint: 对于所有 $j\in[\hintcount]$,$\sumhint_j \coloneqq \sumhint_j+\db_{Eval(\setkey_j, l)}$ \bluetext{以及 $\randomhint_j \coloneqq  \randomhint_j+Eval(\randomset_j, l)\cdot \db_{Eval(\setkey_j, l)}$}。
                      \item 更新不属于此块的备用Hint:对于所有 $j\in[\lambda], k\in[\sqrt{\dbsize}], k\neq l$ 的 $\sumhint_{k,j}$ \bluetext{和 $\randomhint_{k,j}$},$\sumhint_{k,j} \coloneqq  \sumhint_{k,j}+\db_{Eval(\setkey_{k,j}, l)} $ \bluetext{以及 $\randomhint_{k,j} \coloneqq  \randomhint_{k,j}+Eval(\randomset_{k,j}, l)\cdot \db_{Eval(\setkey_{k,j}, l)}$}。
                      \item 将块$l$的 Crumb $\crumb_{l,j}$ 更新为随机选择的记录值以及对应的偏移 $(\crumbvalue_{l,j}, \crumboffset_{l,j}), j\in[\lambda] $。
                      \item \bluetext{$\client$ 用块$l$的内容更新 $\digest'$。}
                      \item 完成之后,$\client$ 从储存中删除此块。
                    \end{itemize}
                \item \bluetext{$\client$ 检查是否有 $\digest = \digest'$。如果不成立, $\client$ 终止协议并输出 $\bot$。}
        \end{itemize}
        \paragraph{在线阶段:}
        \begin{itemize}
            \item \textbf{Query:}
                  \begin{itemize}
                      \item 记 $\dbidx$ 所在块为 $\blockidx\coloneqq \lfloor \dbidx/\sqrt{\dbsize}\rfloor$。 \redtext{\client 记录每一块被查询了多少次。如果块 $\blockidx$ 已经被查询了超过 $\hintcount_2$ 次, \client 随机选择一个被查询少于$\hintcount_2$ 的块中索引 $\dbidx'$ 作为查询对象,并重新运行在线阶段。} $\client$ 在存储中找到一个Hint $\hint_\hintidx = (\setkey_\hintidx,x_\hintidx,\sumhint_\hintidx,\bluetext{\randomset_\hintidx,\randomhint_\hintidx})$。这个Hint需要包含 $\dbidx$ ( $x_\hintidx=\dbidx$ 或是 $Eval(\setkey_\hintidx, \blockidx) + \sqrt{\dbsize}\cdot \blockidx = \dbidx $ 且有 $x_\hintidx = \bot \vee \lfloor x_\hintidx/\sqrt{\dbsize}\rfloor\neq \blockidx$). 如果没有这样的Hint,$\client$ 终止查询并输出 $\bot$。
                      \item $\client$ 将 $\setkey_\hintidx$ $Expand$ 为集合 $S$。如果$x_\hintidx\neq \bot$,将$S[\lfloor x_\hintidx/\sqrt{\dbsize}\rfloor]$ 替换为 $x_\hintidx$。$\client$找到一个块$\blockidx$中的Crumb $\crumb_{\blockidx, \crumbidx}=(\crumbvalue_{\blockidx, \crumbidx},\crumboffset_{\blockidx, \crumbidx})$ 并将 $S[\blockidx]$ 替换为  $\crumboffset_{\blockidx, \crumbidx}$。 \bluetext{$\client$ 将 $\randomset_\hintidx$ $Expand$ 为集合 $SR$,将 $SR[\blockidx]$ 替换为随机数 $r\leftarrow \recordfield$。} $\client$ 将 $S$ \bluetext{与 $SR$} 发送给 $\queryserver$。
                  \end{itemize}
            \item \textbf{Answer:} $\queryserver$ 计算校验值 $\sumanswer\coloneqq \sum_{k\in [\sqrt{\dbsize}]}\db_{S[k]}$ \bluetext{以及 $\randomanswer\coloneqq \sum_{k\in [\sqrt{\dbsize}]}\db_{S[k]}\cdot SR[k]$}。$\queryserver$ 将 $\sumanswer$ \bluetext{与 $\randomanswer$} 发送给 $\client$。
            \item \textbf{Reconstruct:} $\client$ 重构出记录 $\db_\dbidx \coloneqq  \sumhint_\hintidx-(\sumanswer-\crumbvalue_{\blockidx, \crumbidx})$。 \bluetext{$\client$ 验证是否有  $\randomhint_\hintidx-(\randomanswer-r\cdot \crumbvalue_{\blockidx, \crumbidx}) = Eval(\randomset_\hintidx, \blockidx)\cdot \db_\dbidx$。如果验证失败,$\client$ 输出 $\bot$。否则, } $\client$ 输出 $\db_\dbidx$。
            \item \textbf{Refresh:}
                  \begin{itemize}
                      \item $\client$ 找到一个块$\blockidx$中的未使用备用Hint $\hint_{\blockidx,j}=(\setkey_{\blockidx,j},\bot,\sumhint_{\blockidx,j},\bluetext{\randomset_{\blockidx,j},\randomhint_{\blockidx,j}})$。如果块中已经没有备用Hint了,$\client$ 直接结束$Refresh$算法。
                      \item $\client$ 更新Hint:$\hint_\hintidx \coloneqq (\setkey_{\blockidx,j}, \dbidx, \sumhint_{\blockidx,j} + \db_\dbidx, \bluetext{\randomset_{\blockidx,j},\randomhint_{\blockidx,j}+Eval(\randomset_{\blockidx,j},\blockidx)\cdot \db_\dbidx})$.
                  \end{itemize}
        \end{itemize}
    \end{mdframed}
    % \caption{单服务器PIR协议。蓝色部分是验证过程。红色与蓝色部分都可以在半诚实模型中移除。}
    \label{fig:single-server}
% \end{figure*}

\paragraph{“选择失败攻击”引入的问题}
为了应对选择性失败攻击,我们希望采用更强的正确性定义,尤其是允许敌手适应性地选择被查询的索引。然而,单服务器协议无法支持这种定义,因为敌手可能会刻意选择集中于同一个块中的索引进行查询,使一个块的备用Hint消耗殆尽,从而导致正确性失效。而正确性的缺陷又会反过来影响隐私性。为解决这一问题,我们在单服务器协议中引入了一些额外措施(以红色标记),即当客户端无法查询给定的索引时,会随机查询一个索引。

我们认为:(i) 这并不与正确性的定义相矛盾,因为在定义中索引是在进行置换之前选择的;(ii) 该解决方案不会影响实用性,因为协议的诚实执行不受影响,并且敌手总是可以通过拒绝响应来阻止客户端检索所需的索引。

\subsection{离线处理方案的可行性}
\label{sec:offline-discussion}
本文中的单服务器协议以及一些文献\cite{Piano, EC:CorHenKog22}中提出的协议引入了大量的离线通信。我们先前提到的另一种方案\cite{EC:CorHenKog22}使用同态加密计算Hint。该论文提出了一个基于LHE(线性同态加密)的可实现协议以及一个基于FHE(全同态加密)的理论协议。我们尚不清楚如何在实践中实现FHE协议。LHE协议需要$\softO(\recordsize\sqrt{\dbsize \querycount} + \dbsize)$的离线通信。将本文的参数$\querycount=\magictotal$代入,结果为$\softO(\recordsize\dbsize^{3/4} + \dbsize)$,且有较大的对数和常数因子。此外,该协议需要$\softO(\dbsize^{3/4}+\recordsize\sqrt{\dbsize})$的在线通信和计算。与本文$\recordsize\dbsize$离线通信相比,当数据库的记录数超过 $\lambda^6 \approx 2^{42}$ 时,可能会有一定的优势。此时若每条记录的大小为$\recordsize = \Theta(\lambda)$,数据库的大小至少为 64 TiB。并且,该协议在线查询的效率较低。

\subsection{针对在线查询的优化}
\label{sec:optimized-model}
使用备用Hint进行更新的做法也可以应用于双服务器协议。具体来说,双服务器协议中,客户端也可以从Hint服务器获取额外的备用Hint和Crumb,从而可以和单服务器场景一样,在线查询时仅与一台服务器交互。但与单服务器协议不同的是,客户端与服务器并不需要流式传输整个数据库以获取Hint和Crumb,而可以使用与双服务器协议中类似的方式,让客户端将PRF密钥发送给Hint服务器,采用与双服务器协议中的离线Hint算法即可。如此,客户端可以避免与半诚实服务器进行在线交互,同时最大限度地减少离线通信。这一方案为服务提供商提供了一个实用框架。数据库所有者可以利用空闲带宽提供离线Hint获取服务,并将实时查询委托给不可信的服务器。