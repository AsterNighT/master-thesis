\section{可验证的PIR}
\label{sec:verification}
在本节中,我们描述了一种将可验证性嵌入到PIR方案中的技术。我们首先正式提出可验证离线-在线PIR的定义,该概念扩展了先前定义的离线-在线PIR。

\begin{definition}[可验证离线-在线PIR]
    一个可验证离线-在线PIR方案$\Pi$允许客户端从数据库$\db$中检索记录$\db_\dbidx$,而不向$\servercount$个服务器中任意一个泄露索引$\dbidx$,同时确保结果的完整性。该方案包含算法组$\Pi = (Setup, Hint, Query, Answer, Reconstruct, Refresh)$:
    \begin{center}
        离线阶段:
        \begin{itemize}[leftmargin=*]
            \item $Setup(1^\lambda,\dbsize) \rightarrow \query_\hint$:给定数据库大小$\dbsize$和安全参数$\lambda$,生成Hint查询$\query_\hint$。
            \item $Hint(\db, \query_\hint) \rightarrow \hint$:根据数据库$\db$和Hint查询$\query_\hint$ 生成Hint $\hint$。
        \end{itemize}
        在线阶段:
        \begin{itemize}
            \item $Query(\hint, \dbidx) \rightarrow (\query, \clientstate)$:根据Hint~$\hint$和要查询的索引$\dbidx$生成查询$\query$。注意,查询$\query$可能由多个子查询组成。客户端生成并保存一个私有状态$\clientstate$。
            \item $Answer(\db, \query) \rightarrow \answer$:根据查询$\query$生成响应$\answer$。
            \item $Reconstruct(\clientstate, \hint, \answer) \rightarrow \{\db_\dbidx, \bot\}$:根据响应$\answer$,借助Hint~$\hint$和私有状态$\clientstate$重建记录$\db_\dbidx$,或拒绝响应并输出 $\bot$。
            \item $Refresh(\clientstate, \hint, \answer) \rightarrow \hint$:根据响应$\answer$和私有状态$\clientstate$更新Hint~$\hint$。
        \end{itemize}
    \end{center}
    在两服务器($\servercount$=2)时,我们假设Query服务器是恶意的,而Hint服务器是半诚实的。先前的工作 \cite{APIR} 采用了类似的设置,但它们不要求客户端知道哪个服务器是半诚实的。正确的数据库定义为Hint算法 $Hint$ 接受的数据库。在单服务器设置($\servercount=1$)中,正确的数据库由数据所有者提供的摘要定义。该方案应满足以下属性:

    \begin{itemize}
        \item \textbf{正确性}:给定安全参数$\lambda$,任意数据库$\db$和索引$\dbidx$以及任意多项式数量的查询索引序列$\{\dbidx_0,\dbidx_1, \dots\}$,如果服务器和客户端诚实地遵循$\Pi$,则客户端输出$\{\db_{\dbidx_0},\db_{\dbidx_1}, \dots\}$的概率为$1-negl(\lambda)$。
        \item \textbf{完整性}:给定安全参数$\lambda$,任意数据库$\db$和索引$\dbidx$,任意概率多项式时间敌手 $\adversary$作为服务器之一与客户端交互时,客户端重构出错误记录$\widehat{\db_\dbidx} \neq \db_\dbidx$的概率为 $negl(\lambda)$。
        \item \textbf{隐私性(针对选择性失败攻击)}:该属性在定义 \ref{def:privacy-sfa} 中给出。
    \end{itemize}
\end{definition}
\subsection{一种常用于多方安全计算的验证方案}
在多方安全计算协议中,函数 $m = f(x)$ 的计算通常是由一个相关量 $n = g(x) = \alpha f(x)$ 的验证的。其中, $\alpha \in \recordfield$ 代表一个隐藏的系数。验证过程检查是否有 $m\alpha = n$。多方安全计算的性质保障了参与者无法知晓他们计算出的值。诚实的参与者总是能通过此验证,而如果$\recordfield$足够大,不诚实的计算通过验证的概率非常小。

\subsection{验证方案在PIR中的应用}

在我们的协议中,我们采用类似的方法来验证$\queryanswer$的真实性。

在第 \ref{sec:construction} 节中的构造中,客户端最初从服务器获取了一些Hint。我们选择在每个Hint中包含一个额外的校验值,以使客户端能够进行后续的答案验证。升级后的Hint中的校验值包含两个部分:和校验$\sumhint$和随机校验$\randomhint$。对于一个特定集合$S = \{x_j \mid j\in [\setsize]\}$,一组随机数 $SR\coloneqq \{r_j \mid j\in [\setsize]\}$,$\sumhint, \randomhint$ 定义为如下:

$$
    \begin{array}{l}
        \sumhint \coloneqq  \sum_{j \in [\setsize]} \db_{x_j}, \\
        \randomhint \coloneqq  \sum_{j \in [\setsize]} r_j\cdot \db_{x_j}
    \end{array}
$$

查询被分为两个不同的部分:和查询$\sumquery$和随机查询$\randomquery$。假设$x_\dbidx$是查询的索引。$\sumquery$和$\randomquery$的定义如下:

$$
    \begin{array}{l}
        \sumquery \coloneqq  S \setminus \{x_\dbidx\}, \\
        \randomquery \coloneqq  SR \setminus \{r_\dbidx\}
    \end{array}
$$

在接收到这两个查询后,Query服务器计算答案$\sumanswer$和$\randomanswer$:

$$
    \begin{array}{l}
        \sumanswer \coloneqq  \sum_{j \in [s-1]} \db_{\sumquery_j}, \\
        \randomanswer \coloneqq  \sum_{j \in [s-1]} \randomquery_j\cdot \db_{\sumquery_j}
    \end{array}
$$

Query服务器将答案发送给客户端,客户端重构记录$\db_{x_\dbidx} \coloneqq \sumhint-\sumanswer$,通过以下条件来验证答案:

$$
    \randomhint-\randomanswer = r_\dbidx \cdot (\sumhint-\sumanswer)
$$

如果Query服务器如实作答,则等式的左边等于$r_{\dbidx} \cdot \db_{x_\dbidx}$。由于$\sumhint - \sumanswer = \db_{x_\dbidx}$,等式应成立。如果验证未通过,客户端输出 $\bot$。

在我们的框架中,我们进一步采用第 \ref{sec:construction} 节中讨论的利用PRF压缩集合的策略来提高空间效率。集合 $S$ 可以使用密钥$\setkey$通过一个PRF展开。集合$SR$ 可以通过另一个PRF展开:$fr_{\randomset}: [\sqrt{\dbsize}] \to \recordfield$。因此,集合 SR 也可以用一个短密钥$\randomset$来表示。

$$
    \randomset \to SR = \{fr_{\randomset}(j) \mid j\in[\sqrt{\dbsize}]\}
$$

稍稍滥用符号,我们使用 $Eval(\randomset, j)$ 作为 $fr_{\randomset}(j)$ 的简写。

与索引集合一致,这一随机数组成的集合$SR$中被删去元素的位置必然在包含查询索引$\dbidx$的块中。在半诚实方案中,一个随机Crumb($\crumb$)替换了被删去的索引,从而隐藏了$\dbidx$所在的块。同样,在验证方案中,一个随机元素$r \leftarrow \recordfield$可以用于隐藏$SR$中被删去删除的元素。具体来说,令$\blockidx \coloneqq \lfloor \dbidx /\sqrt{\dbsize} \rfloor$为包含查询索引$\dbidx$的块,客户端从$\recordfield$中采样$r \leftarrow \recordfield$,并将$SR[\blockidx]$替换为$r$。在客户端接收到响应$\randomanswer$后,它可以移除Crumb值的贡献($\crumbvalue$)并更新$\randomanswer$,即:
$$\randomanswer \coloneqq \randomanswer - r\cdot \crumbvalue$$
随后,客户端通过检查
$$\randomhint-\randomanswer = Eval(\randomset,\blockidx)\cdot \db_{\dbidx}$$
来执行验证。如果验证失败,客户端输出$\bot$。

我们在下图中展示了我们的双服务器可验证PIR方案,其中验证过程以蓝色突出显示。

% \begin{figure*}
    \begin{mdframed}
        \centering
        \textbf{可验证两服务器协议}

        \raggedright
        \paragraph{符号约定:} 协议包含一个客户端 $\client$,一台Query服务器 $\queryserver$ 与一台Hint服务器 $\hintserver$。单个Hint由元组:$\hint=(\setkey,\sumhint,\bluetext{\randomset,\randomhint})$构成。单个Crumb包含了一个偏移量和一条记录值 $\crumb=(\crumboffset,\crumbvalue)$。$fk_{key}:\{0,1\}^\lambda \to \{0,1\}^\lambda$ 是一个能将单一PRF密钥映射为多个PRF密钥的PRF函数, $f_{\setkey}: [\sqrt{\dbsize}]\to [\sqrt{\dbsize}]$ 是一个将块序号转化为偏移量的PRF, \bluetext{$fr_{\randomset}: [\sqrt{\dbsize}]\to \recordfield$ 是一个将块序号转化为 $\recordfield$上的随机元素的PRF。} $Eval(\setkey,\dbidx) \coloneqq f_{\setkey}(\dbidx)\oplus \shift + \dbidx\cdot\sqrt{\dbsize}$ 与 $Expand$ 密钥 $\setkey$ 标识计算密钥对应的集合内元素 $\{Eval(sk,j) \mid j\in[\sqrt{\dbsize}]\}$。 \bluetext{$Eval(\randomset,\dbidx) \coloneqq fr_{\randomset}(\dbidx)$ 与 $Expand$ 密钥 $\randomset$ 表示计算集合元素 $\{Eval(sr,j) \mid j\in[\sqrt{\dbsize}]\}$。} 假设 $\dbidx$ 是需要查询的索引。 在离线阶段,总共生成 $\hintcount$ 个 Hint。

        \paragraph{离线阶段:}
        \begin{itemize}
            \item \textbf{Setup:} $\client$ 生成一个PRF密钥 $mk\in\{0,1\}^\lambda$,将本地的Hint存储初始化为 $\hint_j\coloneqq(fk_{mk}(j),0,\bluetext{fk_{mk}(j+\hintcount),0}), j\in [\hintcount]$,Crumb存储初始化为 $\crumb_j\coloneqq (\bot,\bot), j\in [\sqrt{\dbsize}]$。 $\client$ 将 $mk$ 发送给 $\hintserver$.
            \item \textbf{Hint:}
                  \begin{itemize}
                      \item $\hintserver$ 将 $mk$ 展开为PRF密钥 $\setkey_j=fk_{mk}(j), j\in[\hintcount]$ \bluetext{与 $\randomset_j=fk_{mk}(j+\hintcount), j\in[\hintcount]$}。它将这些密钥进一步$Expand$为集合 $S_j, j\in[\hintcount]$ \bluetext{与 $SR_j, j\in[\hintcount]$}.
                      \item $\hintserver$ 为每个集合计算校验值 $\sumhint_j\coloneqq \sum_{k\in [\sqrt{\dbsize}]}\db_{S_{j}[k]}, j\in[\hintcount]$ \bluetext{与 $\randomhint_j\coloneqq \sum_{k\in [\sqrt{\dbsize}]}\db_{S_{j}[k]}\cdot SR_{j}[k], j\in[\hintcount]$}。$\hintserver$将这些校验值发送给 $\client$.
                      \item 在第 $j$ 个 $\sqrt{\dbsize}$ 大小的块 ($j\in[\sqrt{N}]$) 内, $\hintserver$ 随机选取一个偏移量 $\crumboffset_j\leftarrow [\sqrt{N}]$ 及其对应的记录值 $\crumbvalue_j$ 作为Crumb。它将这些Crumb $\crumb_j \coloneqq  (\crumboffset_j, \crumbvalue_j), j\in [\sqrt{\dbsize}]$ 发送给 $\client$。 $\client$ 接受并储存这些Crumb。
                  \end{itemize}
        \end{itemize}
        \paragraph{在线阶段:}
        \begin{itemize}
            \item \textbf{Query ($\queryserver$):}
                  \begin{itemize}
                      \item 记 $\dbidx$ 所在的块为 $\blockidx\coloneqq \lfloor \dbidx/\sqrt{\dbsize}\rfloor$。 $\client$ 在存储中找到一个Hint $\hint_\hintidx = (\setkey_\hintidx,\sumhint_\hintidx,\bluetext{\randomset_\hintidx,\randomhint_\hintidx})$,满足条件 $f_{\setkey_\hintidx}(\blockidx) = \dbidx-\sqrt{\dbsize}\cdot \blockidx$。如果没有这样的Hint,$\client$ 终止查询并输出 $\bot$。
                      \item $\client$ 将 $\setkey_\hintidx$ $Expand$ 为集合 $S$。$\client$ 找到 $\blockidx$ 的 Crumb $\crumb_\blockidx=(\crumboffset_\blockidx, \crumbvalue_\blockidx)$,将 $S[\blockidx]$ 替换为Crumb偏移量 $\crumboffset_\blockidx$。 \bluetext{$\client$ 将 $\randomset_\hintidx$ $Expand$ 为集合 $SR$,并将 $SR[\blockidx]$ 替换为一随机元素 $r\leftarrow \recordfield$。} $\client$ 将 $S$ \bluetext{与 $SR$} 发送给 $\queryserver$。
                  \end{itemize}
            \item \textbf{Answer ($\queryserver$):} $\queryserver$ 计算校验值 $\sumanswer\coloneqq \sum_{k\in [\sqrt{\dbsize}]}\db_{S[k]}$ \bluetext{与 $\randomanswer\coloneqq \sum_{k\in [\sqrt{\dbsize}]}\db_{S[k]}\cdot SR[k]$}。$\queryserver$ 将 $\sumanswer$ \bluetext{与 $\randomanswer$} 发送给 $\client$.
            \item \textbf{Reconstruct:} $\client$ 重构出记录 $\db_\dbidx \coloneqq  \sumhint_\hintidx-(\sumanswer-\crumbvalue_\blockidx)$。 \bluetext{$\client$ 验证是否有 $\randomhint_\hintidx-(\randomanswer-r\cdot \crumbvalue_\blockidx) = Eval(\randomset_\hintidx, \blockidx)\cdot \db_\dbidx$。如果验证失败,$\client$ 输出 $\bot$。否则, } $\client$ 输出 $\db_\dbidx$。
            \item \textbf{Query ($\hintserver$):}
                  \begin{itemize}
                      \item $\client$ 生成PRF密钥 $\setkey'$ 使得 $f_{\setkey'}(\blockidx) = \dbidx-\sqrt{\dbsize}\cdot \blockidx$。 \bluetext{$\client$ 生成PRF密钥 $\randomset'$}.
                      \item $\client$ 将 $\setkey'$ $Expand$ 为集合 $S'$, \bluetext{将 $\randomset'$ $Expand$ 为集合 $SR'$}。$\client$ 将 $S'[\blockidx]$ 替换为随机偏移 $\crumboffset' \leftarrow [\sqrt{\dbsize}]$. $\client$ 将 $S'$ 发送给 $\hintserver$.
                  \end{itemize}
            \item \textbf{Answer ($\hintserver$):} $\hintserver$ 直接将 $\sqrt{\dbsize}$ 条记录 $\db_{S'[j]}, j\in [\sqrt{\dbsize}]$ 发送给 $\client$。
            \item \textbf{Refresh:} $\client$ 将Crumb更新为 $\crumb_\blockidx \coloneqq  (\crumboffset', \db_{S'[\blockidx]})$,将Hint更新为 $\hint_\hintidx \coloneqq  (\setkey',\db_\dbidx + \sum_{k\in [\sqrt{\dbsize}],k\neq \blockidx}\db_{S'[k]}, \bluetext{\randomset', SR[l]\cdot \db_\dbidx + \sum_{k\in [\sqrt{\dbsize}],k\neq \blockidx}SR[k]\cdot \db_{S'[k]}})$
        \end{itemize}
    \end{mdframed}
    \label{fig:two-server-verify}
% \end{figure*}

% 我们提出了以下关于安全性和效率的定理:

% \begin{theorem}
%     \label{thm:1}
%     图 \ref{fig:two-server-verify} 中描述的方案是一个双服务器可验证的离线-在线PIR方案。设 $\alpha(\lambda)$ 为任意超常数函数,即$\alpha(\lambda)$ = $\omega(1)$。假设数据库索引大小为$O(1)$,PRF密钥大小为$O_\lambda(1)$。在一个包含$\dbsize$条记录、每条大小为$\recordsize$的数据库上,存储$\hintcount = \magictotal$个Hint,该方案的效率如下:
    
%     \noindent 在离线阶段:
%     \begin{itemize}[itemsep=0em]
%         \item 通信量为$O_\lambda(1)$上传,$2\recordsize\hintcount + \recordsize\sqrt{\dbsize} + O(\sqrt{\dbsize})$下载。
%         \item 客户端计算复杂度为$O_\lambda(\hintcount)$,服务器计算复杂度为$O_\lambda(\hintcount) + O(\recordsize\sqrt{\dbsize}\hintcount)$。
%     \end{itemize}
%     在在线阶段:
%     \begin{itemize}[itemsep=0em]
%         \item 向Query服务器通信量为:$\recordsize\sqrt{\dbsize} + O(\sqrt{\dbsize})$上传,$2\recordsize$下载。
%         \item 向Hint服务器通信量为:$O(\sqrt{\dbsize})$上传,$\recordsize\sqrt{\dbsize}$下载。
%         \item 客户端计算复杂度为$O_\lambda(\sqrt{\dbsize}) + O(\recordsize\sqrt{\dbsize})$,服务器计算复杂度为$O(\recordsize\sqrt{\dbsize})$。
%         \item 支持多项式数量的查询。
%     \end{itemize}
% \end{theorem}

% \paragraph{恶意的Hint服务器}
% 当Hint服务器是恶意的时,该协议是不安全的。在附录 \ref{appendix:hint-server} 中提供了详细的解释和示例。