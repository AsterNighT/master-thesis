\section{协议分析与安全性证明}
\label{sec:analysis}

在证明中,为了简化处理,我们假设方案中的PRF是完全随机的。

\subsection{扩展至多个查询}
\label{appendix:extension-to-multiple-queries}
要证明双服务器方案可以处理多项式数量的查询,只需要证明每次查询后Hint的分布保持不变。

\begin{lemma}
\label{lemma:extension}
在离线阶段结束时,以及每次查询后,客户端的Hint$\hint$遵循相同的分布,与$\hintcount$个每个块包含一条随机记录的随机集合分布无法区分。
\end{lemma}

\begin{proof}
在离线阶段结束时,该引理的正确性直接由方案和PRF的安全性推出。在在线阶段,只需证明在每次查询过程中,消耗的Hint和更新的Hint信息遵循相同的分布。对于落在第$\blockidx$个块的查询$\dbidx$,找到的查询集合$\setkey_\hintidx$和新采样的集合$\setkey'$在除$\blockidx$之外的每个块中都是均匀随机的,而在块$\blockidx$中必然为$\dbidx$。
\end{proof}

由引理 \ref{lemma:extension},针对第一次查询的正确性证明可以扩展到后续的查询。在接下来的证明中,除非特别说明,我们假设Hint信息的分布在查询发生时不会改变,并省略了Hint的初始化、消耗和更新过程,以简化表述。

\subsection{正确性}

我们首先给出正确性的形式化定义。

\begin{definition}[双服务器方案的正确性]
给定安全参数 $\lambda$,对于任意数据库$\db$和索引$\dbidx$以及任意多项式长度的查询索引序列$\{\dbidx_0,\dbidx_1, \dots\}$,如果两台服务器和客户端都诚实地遵循协议,客户端以概率$PC_\dbidx \geq 1-\negl(\lambda)$ 输出$\db_\dbidx$。
\end{definition}

\begin{theorem}
    当$\hintcount=\magictotal$时,第 \ref{fig:two-server} 节中的协议满足正确性。
\end{theorem}

\begin{proof}
    当且仅当 $Query$ 算法中客户端找不到包含查询索引的Hint时,客户端无法检索到记录。我们证明这一事件的概率是可忽略的。首先,我们证明以下引理:

    \begin{lemma}
    \label{lemma:hint-prob}
        每个Hint包含特定索引$\dbidx$的概率为$\frac{1}{\sqrt{\dbsize}}$。
    \end{lemma}

    \begin{proof}
    每个Hint在大小为$\sqrt{\dbsize}$的每个块中包含一条随机记录。任一特定索引必然落在某个块中,并以$\frac{1}{\sqrt{\dbsize}}$的概率被选中。
    \end{proof}

    根据引理 \ref{lemma:hint-prob},客户端找不到包含查询索引的Hint的概率至多为$(1-\frac{1}{\sqrt{\dbsize}})^\hintcount$。当$\hintcount=\magictotal$时,$(1-\frac{1}{\sqrt{\dbsize}})^{\magictotal}\le \lambda^{-\alpha(\lambda)} \le \negl(\lambda)$。
\end{proof}

\begin{definition}[单服务器方案的正确性]
给定安全参数 $\lambda$,对于任意数据库$\db$和索引$\dbidx$以及任意长度不超过$\querycount=\magictotal$的查询索引序列$\{\dbidx_0,\dbidx_1, \dots\}$。如果服务器和客户端诚实地遵循协议,在一次离线阶段之后,客户端至少能以概率$PC_\dbidx \geq 1-\negl(\lambda)$输出$\db_\dbidx$。
\end{definition}

\begin{theorem}
    当每个块有$\hintcount_1=\magictotal$个主Hint和$\hintcount_2 = \backuphintcount$个备用Hint时,第 \ref{fig:single-server} 节中的协议满足正确性。
\end{theorem}

\begin{proof}
有两种情况下客户端无法获取$\dbidx$。第一种情况与双服务器方案相同。第二种情况是某个块内有超过$\hintcount_2$次查询。我们证明以下引理:

\begin{lemma}
\label{lemma:crumb-prob}
    对于$\querycount=\magictotal$次查询,落在某个块内的查询超过$\backuphintcount$次的概率关于$\lambda$是可忽略的。
\end{lemma}

\begin{proof}
    在一个块中期望的查询次数是$\magicnumber$。该概率可以通过切尔诺夫界限来限定。令$X$表示落在第一个块中的查询次数,取$\delta=2$。
$$ Pr(X \ge (1+2)\magicnumber) \le e^{-2^2\magicnumber/(2+2)} = \lambda^{-\alpha(\lambda)} $$ 
对所有块使用联合界限,该概率最多为
$$\sqrt{\dbsize}\lambda^{-\alpha(\lambda)} \le \negl(\lambda)$$
\end{proof}

单服务器的正确性直接来自于双服务器正确性的论证和引理 \ref{lemma:crumb-prob}。
\end{proof}

从现在起,我们假设 $Query$ 算法永远不会失败。这对于双服务器方案是显而易见的,因为正确性证明表明客户端可以以压倒性概率找到Hint。对于单服务器方案,我们在第 \ref{fig:single-server} 节中的协议内引入了一种额外的解决方法,以确保这一假设成立。

\subsection{完整性}
\label{appendix:integrity}

我们首先给出完整性的形式化定义。

\begin{definition}[双服务器方案的完整性]
给定安全参数 $\lambda$,对于所有多项式时间的概率性对手$\adversary_{\mathrm{query}}$,假设该对手控制了查询服务器,定义如下概率为$PI_\dbidx(\adversary_{\mathrm{query}})$。
            $$ Pr\left[
            c \notin \{\db_\dbidx,\bot\}:
            \begin{array}{rcl}
                \query_\hint                          & \leftarrow & Setup(1^\lambda, \dbsize)                       \\
                \hint                                 & \leftarrow & Hint(\db, q_h)                                  \\
                (\hintquery,\queryquery,\clientstate) & \leftarrow & Query(\hint, \dbidx)                            \\
                \queryanswer'                         & \leftarrow & \adversary_{\mathrm{query}}(\db, \queryquery)   \\
                c                                     & \leftarrow & Reconstruct(\clientstate, \hint, \queryanswer') \\
            \end{array}
            \right]$$
对于所有的$\dbidx$,$PI_\dbidx(\adversary_{\mathrm{query}}) \leq \negl(\lambda)$。
\end{definition}

\begin{theorem}
    第 \ref{fig:two-server} 节中的协议满足完整性。
\end{theorem}

\begin{proof}

我们证明一个有用的引理。

\begin{lemma}
\label{lemma:integrity}
对于任何非零偏移量$\Delta=(\Delta^+,\Delta^\times)\in \recordfield^2, \Delta \neq (0, 0)$,以下概率表示客户端接受查询服务器提供的错误答案 $a'=a + \Delta$ 的可能性。用$P_{\Delta, \dbidx}$表示该概率。
$$ Pr\left[
    c\neq \bot:
    \begin{array}{rcl}
        \query_\hint                          & \leftarrow & Setup(1^\lambda, \dbsize)                               \\
        \hint                                 & \leftarrow & Hint(\db, q_h)                                          \\
        (\hintquery,\queryquery,\clientstate) & \leftarrow & Query(\hint, \dbidx)                                    \\
        \queryanswer                          & \leftarrow & Answer(\db, \queryquery)                           \\
        c                                     & \leftarrow & Reconstruct(\clientstate, \hint, \queryanswer + \Delta) \\
    \end{array}
    \right]$$
对于所有的$\Delta, \dbidx$,$P_{\Delta, \dbidx} \leq \frac{1}{2^\recordsize-1}$。
\end{lemma}

\begin{proof}
设 $r$ 为从随机集合中移除的随机元素。简而言之,验证通过的概率可以表示为:

$$ Pr\left[ r\cdot(\sumhint-\sumanswer - \Delta^+) = \randomhint - \randomanswer -\Delta^\times \right]$$
其中$\sumanswer,\randomanswer$是正确答案,$\sumhint, \randomhint$是Hint的校验值。由于正确答案总能通过验证,我们有

$$ r\cdot(\sumhint-\sumanswer)=\randomhint - \randomanswer $$
因此,该概率可以简化为

$$ Pr\left[ r\Delta^+ - \Delta^\times = 0 \right]$$

$r$是一个$\recordfield$上的随机元素。该方程是一个非零一次多项式在随机点$r$上的求值。由于一次多项式在$\recordfield$中至多有一个根,因此该概率至多为$\frac{1}{2^\recordsize-1}$。由于$\recordsize\ge\lambda$。因此,该概率关于$\lambda$是可忽略的。
\end{proof}

引理 \ref{lemma:integrity}直接证明了完整性。
\end{proof}

单服务器方案的完整性定义类似。

\begin{definition}[单服务器方案的完整性]
    给定安全参数 $\lambda$,对于所有多项式时间的概率性对手$\adversary$,假设该对手控制了服务器,定义如下概率为$PI_\dbidx(\adversary)$。
$$ Pr\left[
            c \notin \{\db_\dbidx,\bot\}:
            \begin{array}{rcl}
                \digest        & \leftarrow & Digest(1^\lambda, \dbsize, \db)              \\
                \query_\hint   & \leftarrow & Setup(1^\lambda, \dbsize)                    \\
                \{\hint,\bot\} & \leftarrow & Hint(\adversary(\db), \query_\hint, \digest) \\
                \query         & \leftarrow & Query(\hint, \dbidx)                         \\
                \answer'       & \leftarrow & \adversary(\db, \query)                      \\
                c              & \leftarrow & Reconstruct(\hint, \answer')                 \\
            \end{array}
            \right]$$
对于所有$\dbidx$,$PI_\dbidx(\adversary) \leq \negl(\lambda)$。
\end{definition}

\begin{theorem}
第 \ref{fig:single-server} 节中的协议满足完整性。
\end{theorem}

单服务器版本的证明与双服务器相同,唯一的区别是我们需要为离线阶段引入一个额外机制,确保Hint的正确性。该证明可以通过哈希函数的基本性质完成,这里我们将其省略了。

\subsection{隐私性}

我们首先证明该方案对选择失败攻击的隐私性,然后自然地推导出一种更弱的隐私性形式,即不涉及选择失败的隐私性。正如完整性部分所述,我们假设查询总是成功的,并忽略Hint的初始化、消耗和更新过程。

首先,我们证明客户端的查询与对随机索引的查询无法区分。

我们将证明以下引理:

\begin{lemma}
\label{lemma:privacy}
    在双服务器方案中,对于任意由两个服务器之一接收到的查询$\query$和任意两个查询索引$\dbidx,\dbidx'$,有
    $$Pr[\query|\dbidx] = Pr[\query|\dbidx']$$
\end{lemma}

\begin{proof}
我们证明查询$\dbidx$或$\dbidx'$的请求与$\sqrt{\dbsize}$个在$[\sqrt{\dbsize}]$上的随机数加上$\sqrt{\dbsize}$个在$\recordfield$中的随机元素无法区分。

在该方案中,$\query$由两部分组成$\query = (\sumquery, \randomquery)$。$\sumquery$是一个包含$\sqrt{\dbsize}$个偏移量的集合。设$\blockidx$为包含$\dbidx$或$\dbidx'$的块。$\sumquery$无法区分于$[\sqrt{\dbsize}]$中的$\sqrt{\dbsize}$个随机数:
\begin{itemize}
    \item 对于除$\blockidx$以外的块,偏移量是随机选择的,并且之前没有被服务器看到。
    \item 对于块$\blockidx$,在$Query$算法中,它被替换为一个随机偏移量(一个crumb)。
\end{itemize}
$\randomquery$仅仅是$\recordfield$中的$\sqrt{\dbsize}$个随机元素,其内容与查询的索引$\dbidx$或$\dbidx'$无关。由此可得该引理成立。
\end{proof}

借助引理 \ref{lemma:privacy},我们可以证明该方案的隐私性。首先,我们证明其对$\queryserver$的隐私性。在离线阶段,$\queryserver$与$\client$没有交互。因此,我们只需考虑在线阶段。

\begin{definition}[选择失败隐私性 - $\queryserver$]
给定安全参数$\lambda$,如果存在一个多项式时间的概率模拟器$\simulator(1^\lambda, \dbsize)$,使得对于控制服务器$\queryserver$的任何多项式时间的概率性对手$\adversary$,与诚实的$\hintserver$进行交互时,$\adversary$无法以关于$\lambda$不可忽略的概率区分以下两个世界,则该双服务器方案对$\queryserver$满足选择失败隐私性。

\begin{itemize}
    \item \textbf{世界 0}: 在在线阶段的每个步骤$t$,$\adversary$自适应地选择下一个索引$\dbidx_t$,$\client$使用$\dbidx_t$作为其查询索引。在每次查询后,如果答案被接受,$\client$向$\adversary$输出1,否则输出0。
    \item \textbf{世界 1}: 在在线阶段的每个步骤$t$,$\adversary$自适应地选择下一个索引$\dbidx_t$。$\simulator$在不知道$\dbidx_t$的情况下运行,但在$\adversary$给出正确答案时接收一个比特1,否则为0。$\simulator$在每次查询后将该比特输出给$\adversary$。
\end{itemize}$\adversary$被允许随意偏离协议。
\end{definition}

为了更清晰地说明,我们假设$\adversary$在每个步骤$t$接受一个额外输入$o_t$。如果$o_t$为1,$\adversary$诚实回答;如果$o_t$为0,$\adversary$给出一个错误答案。两个世界可以被重写如下(忽略不相关的操作):

$$Ok(c)=\left\{\begin{array}{ll}
        1, & c\neq \bot \\
        0, & c = \bot
    \end{array}\right.$$
\begin{mdframed}
    \begin{multicols}{2}
        \textbf{World 0}:
        \begin{itemize}
            \item For $t=1,2,\dots$:
                  \begin{itemize}
                      \item $\dbidx_t \leftarrow \adversary(\db)$
                      \item $(\query_t,\clientstate_t) \leftarrow Query(\dbidx_t)$
                      \item $\answer_t \leftarrow \adversary(\db, \query_t, o_t)$
                      \item $c_t \leftarrow Reconstruct(\clientstate_t, \answer_t)$
                      \item $Ok(c_t) \rightarrow \adversary$
                  \end{itemize}
        \end{itemize}
        \textbf{World 1}:
        \begin{itemize}
            \item For $t=1,2,\dots$:
                  \begin{itemize}
                      \item $\dbidx_t \leftarrow \adversary(\db)$
                      \item $(\query_t,\clientstate_t) \leftarrow \simulator$
                      \item $\answer_t \leftarrow \adversary(\db, \query_t, o_t)$
                      \item $b \leftarrow \simulator(o_t)$
                      \item $b \rightarrow \adversary$
                  \end{itemize}
        \end{itemize}
    \end{multicols}
\end{mdframed}

该定义等价于一个真实/理想世界范式。

\begin{theorem}
    第 \ref{fig:two-server} 节中的协议对$\queryserver$具有选择失败隐私性。
\end{theorem}
    
\begin{proof}
    
    在离线阶段,$\queryserver$不与$\client$交互。因此,我们只需要考虑在线阶段。考虑以下混合(hybrids):
    
    \begin{itemize}
        \item \textbf{Hyb 0}: 这是现实世界的执行。在这个混合中,\simulator 的执行方式与 \client 在 \textbf{World 0} 中的完全相同。
        \item \textbf{Hyb 1}: 在这个混合中,$\simulator$放弃了$\dbidx_t$并随机选择一个索引$\dbidx_t'$作为其查询索引。根据引理 \ref{lemma:privacy},此混合与 \textbf{Hyb 0} 是不可区分的。
        \item \textbf{Hyb 2}: 在这个混合中,$\simulator$收到一个比特$b$。如果$\adversary$给出正确答案,$b$为$1$,否则为$0$。$\simulator$将$b$输出给$\adversary$,而不是验证答案。这个混合与 \textbf{Hyb 1} 是不可区分的。根据引理 \ref{lemma:integrity},除了可忽略的概率外,比特$b$与$Ok(c_t)$的输出相同。
    \end{itemize}
    $\simulator$在 \textbf{Hyb 2} 中的执行与 \textbf{World 1} 完全相同。
\end{proof}
    
对$\hintserver$的隐私性单独定义。
    
\begin{definition}[选择失败隐私性 - $\hintserver$]
    给定安全参数$\lambda$,如果存在一个概率性多项式时间的模拟器$\simulator(1^\lambda, \dbsize)$,使得对于任何控制服务器$\hintserver$的概率多项式时间对手$\adversary$,当同不和$\adversary$串通的恶意$\queryserver$交互时,$\adversary$无法以关于$\lambda$不可忽略的概率区分以下世界,则两服务器方案对$\hintserver$具有选择失败隐私性。
    
    \begin{itemize}
        \item \textbf{World 0}: 在在线阶段的每一步$t$,$\adversary$自适应地选择下一个索引$\dbidx_t$,而$\client$使用$\dbidx_t$作为查询索引。若答案被接受,客户端会在每次查询后输出$1$给$\adversary$,否则输出$0$。
        \item \textbf{World 1}: 在在线阶段的每一步$t$,$\adversary$自适应地选择下一个索引$\dbidx_t$。$\simulator$在没有$\dbidx_t$信息的情况下运行,但在$\queryserver$给出正确答案时收到比特$1$,否则为$0$。$\simulator$在每次查询后将比特输出给$\adversary$。
    \end{itemize}
    
    我们强调$\adversary$必须诚实地遵循协议。
\end{definition}
    
\begin{theorem}
    第 \ref{fig:two-server} 节中的两服务器协议对$\hintserver$具有选择性失败隐私性。
\end{theorem}
    
\begin{proof}
    对$\hintserver$的隐私性在离线阶段直接由以下事实得出:离线阶段发生在任何查询之前。在线阶段的证明与对$\queryserver$的证明相同。
\end{proof}
    
单服务器环境中的隐私性定义与对$\queryserver$的两服务器隐私性类似。
    
\begin{definition}[选择失败隐私性 - 单服务器]
    给定安全参数$\lambda$,如果存在一个概率性多项式时间的模拟器$\simulator(1^\lambda, \dbsize)$,使得对于任何控制$\server$的概率多项式时间对手$\adversary$,$\adversary$无法以关于$\lambda$不可忽略的概率区分以下世界,则单服务器方案对$\server$具有选择性失败隐私性。
    
    \begin{itemize}
        \item \textbf{World 0}: 在在线阶段的每一步$t$,$\adversary$自适应地选择下一个索引$\dbidx_t$,而$\client$使用$\dbidx_t$作为查询索引。若答案被接受,客户端会在每次查询后输出$1$给$\adversary$,否则输出$0$。
        \item \textbf{World 1}: 在在线阶段的每一步$t$,$\adversary$自适应地选择下一个索引$\dbidx_t$。$\simulator$在没有$\dbidx_t$信息的情况下运行,但在$\adversary$给出正确答案时收到比特$1$,否则为$0$。$\simulator$在每次查询后将比特输出给$\adversary$。
    \end{itemize}$\adversary$被允许任意偏离协议。
\end{definition}
    
\begin{theorem}
    第 \ref{fig:single-server} 节中的单服务器协议对$\server$具有选择性失败隐私性。
\end{theorem}
    
\begin{proof}
    该证明需要一些额外的论证。由于$\adversary$可以自适应地选择下一个索引$\dbidx_t$,这可能导致$\client$无法找到包含$\dbidx_t$的提示。单服务器协议中的解决方法是通过选择另一个索引进行查询。考虑以下混合:
    
    \begin{itemize}
        \item \textbf{Hyb 0}: 这是现实世界的执行。在这个混合中,\simulator 的执行方式与 \client 在 \textbf{World 0} 中的完全相同。
        \item \textbf{Hyb 1}: 在这个混合中,$\simulator$放弃了$\dbidx_t$并随机选择一个索引$\dbidx_t'$作为其查询索引。这里有两种情况需要讨论:
        \begin{enumerate}
            \item 在 \textbf{Hyb 0} 中,\simulator 找到了$\dbidx_t$的合适提示并正常继续查询。根据引理 \ref{lemma:privacy},\textbf{Hyb 1} 与 \textbf{Hyb 0} 是不可区分的。
            \item 在 \textbf{Hyb 0} 中,\simulator 无法找到$\dbidx_t$的提示并查询了$\dbidx_t''$。根据引理 \ref{lemma:privacy},对$\dbidx_t'$和$\dbidx_t''$的查询仍然是不可区分的。
        \end{enumerate}
        因此,\textbf{Hyb 1} 与 \textbf{Hyb 0} 是不可区分的。
        \item \textbf{Hyb 2}: 在这个混合中,$\simulator$收到一个比特$b$。如果$\adversary$给出正确答案,$b$为$1$,否则为$0$。$\simulator$将$b$输出给$\adversary$,而不是验证答案。这个混合与 \textbf{Hyb 1} 是不可区分的。根据引理 \ref{lemma:integrity},除了可忽略的概率外,比特$b$与$Ok(c_t)$的输出相同。
    \end{itemize}
\end{proof}
% \subsection{安全分析}
% {这里可以写好几页,也可以写一句话,看情况}
% 在本节中,我们聚焦于图 \ref{fig:two-server-verify} 中描述的双服务器方案,并尝试对协议的安全性证明提供直观的解释。总体而言,单服务器协议与双服务器方案类似,但只包含了Query服务器。详细安全性证明在附录 \ref{appendix:security} 中提供。

% \paragraph{正确性}
% 仅当客户端无法找到包含查询索引的集合时,它才无法获得查询的记录。由于客户端保有$\hintcount = \magictotal$个Hint,这种情况发生的概率可忽略不计。来自服务器的诚实答案始终会被接受,所以正确性的缺陷是可忽略的。

% \paragraph{完整性}
% 设 $\blockidx = \lfloor \dbidx / \sqrt{\dbsize} \rfloor$ 为包含查询索引的块,$r = Eval(\randomset, \blockidx)$ 是从查询中移除的随机元素。在$Reconstruct$算法中,客户端检查是否有$\randomhint-\randomanswer = r \cdot (\sumhint-\sumanswer)$。恶意查询服务器提供的错误答案可以表示为偏移量$(\sumanswer+\Delta^+, \randomanswer+\Delta^\times)$,其中$(\Delta^+, \Delta^\times) \neq (0, 0)$。该检查可以化简为 $r\Delta^+ - \Delta^\times = 0$。在不知道$r$的情况下,此式为真的概率为$2^{-\recordsize}$。

% \paragraph{隐私性}
% 在离线阶段,生成的的Hint和集合与任何特定索引无关。在线阶段,客户端向任何服务器的每个查询都是$[\sqrt{\dbsize}]$上的$\sqrt{\dbsize}$个随机数加上$\recordfield$中的$\sqrt{\dbsize}$个随机元素,与查询的索引也没有关系。

% 本文的方案能够抵抗选择失败攻击。不难发现,验证结果不会向Query服务器提供任何信息。如果服务器如实作答,根据正确性,客户端通过验证并接受答案。如果服务器的答案是错误的,根据完整性分析,客户端只有$2^{-\recordsize}$的概率接受这一答案。因此,敌手的优势不超过$2^{-\recordsize}$。由于 $\recordsize \ge \lambda$,这是一个关于$\lambda$的可忽略函数。

\subsection{性能分析}
在离线阶段,客户端向Hint服务器发送一个大小为$O_\lambda(1)$的PRF密钥。Hint服务器将该密钥扩展为$\hintcount$个Hint密钥,并计算这些Hint。随后,服务器向客户端发送$\hintcount$个校验值和$\sqrt{\dbsize}$个Crumb。校验值的大小为$2\recordsize\hintcount$。$\sqrt{\dbsize}$个面包屑的大小为$\recordsize\sqrt{\dbsize} + O(\sqrt{\dbsize})$。因此,离线通信的总量为$2\recordsize\hintcount + \recordsize\sqrt{\dbsize} + O(\sqrt{\dbsize})$,离线计算的开销为$O_\lambda(\hintcount) + O(\recordsize\sqrt{\dbsize}\hintcount)$。

在在线阶段,客户端搜索包含查询索引$\dbidx$的Hint,预期计算开销为$O_\lambda(\sqrt{\dbsize})$。随后的Hint处理,包括展开集合、用Crumb替换索引以及生成新集合,也需要$O_\lambda(\sqrt{\dbsize})$的计算量。客户端向Query服务器发送的查询大小为$\recordsize\sqrt{\dbsize} + O(\sqrt{\dbsize})$,向Hint服务器发送的查询大小为$O(\sqrt{\dbsize})$。每个服务器在$O(\recordsize\sqrt{\dbsize})$时间内计算校验值。Hint服务器回复$\sqrt{\dbsize}$条记录,Query服务器回复2个校验值。然后,客户端在$O(\recordsize\sqrt{\dbsize})$时间内验证答案并刷新Hint。
