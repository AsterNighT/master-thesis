\section{协议分析}

\subsection{安全分析}
\todo{这里可以写好几页,也可以写一句话,看情况}
在本节中,我们聚焦于图 \ref{fig:two-server-verify} 中描述的双服务器方案,并尝试对协议的安全性证明提供直观的解释。总体而言,单服务器协议与双服务器方案类似,但只包含了Query服务器。详细安全性证明在附录 \ref{appendix:security} 中提供。

\paragraph{正确性}
仅当客户端无法找到包含查询索引的集合时,它才无法获得查询的记录。由于客户端保有$\hintcount = \magictotal$个Hint,这种情况发生的概率可忽略不计。来自服务器的诚实答案始终会被接受,所以正确性的缺陷是可忽略的。

\paragraph{完整性}
设 $\blockidx = \lfloor \dbidx / \sqrt{\dbsize} \rfloor$ 为包含查询索引的块,$r = Eval(\randomset, \blockidx)$ 是从查询中移除的随机元素。在$Reconstruct$算法中,客户端检查是否有$\randomhint-\randomanswer = r \cdot (\sumhint-\sumanswer)$。恶意查询服务器提供的错误答案可以表示为偏移量$(\sumanswer+\Delta^+, \randomanswer+\Delta^\times)$,其中$(\Delta^+, \Delta^\times) \neq (0, 0)$。该检查可以化简为 $r\Delta^+ - \Delta^\times = 0$。在不知道$r$的情况下,此式为真的概率为$2^{-\recordsize}$。

\paragraph{隐私性}
在离线阶段,生成的的Hint和集合与任何特定索引无关。在线阶段,客户端向任何服务器的每个查询都是$[\sqrt{\dbsize}]$上的$\sqrt{\dbsize}$个随机数加上$\recordfield$中的$\sqrt{\dbsize}$个随机元素,与查询的索引也没有关系。

本文的方案能够抵抗选择失败攻击。不难发现,验证结果不会向Query服务器提供任何信息。如果服务器如实作答,根据正确性,客户端通过验证并接受答案。如果服务器的答案是错误的,根据完整性分析,客户端只有$2^{-\recordsize}$的概率接受这一答案。因此,敌手的优势不超过$2^{-\recordsize}$。由于 $\recordsize \ge \lambda$,这是一个关于$\lambda$的可忽略函数。

\subsection{性能分析}
在离线阶段,客户端向Hint服务器发送一个大小为$O_\lambda(1)$的PRF密钥。Hint服务器将该密钥扩展为$\hintcount$个Hint密钥,并计算这些Hint。随后,服务器向客户端发送$\hintcount$个校验值和$\sqrt{\dbsize}$个Crumb。校验值的大小为$2\recordsize\hintcount$。$\sqrt{\dbsize}$个面包屑的大小为$\recordsize\sqrt{\dbsize} + O(\sqrt{\dbsize})$。因此,离线通信的总量为$2\recordsize\hintcount + \recordsize\sqrt{\dbsize} + O(\sqrt{\dbsize})$,离线计算的开销为$O_\lambda(\hintcount) + O(\recordsize\sqrt{\dbsize}\hintcount)$。

在在线阶段,客户端搜索包含查询索引$\dbidx$的Hint,预期计算开销为$O_\lambda(\sqrt{\dbsize})$。随后的Hint处理,包括展开集合、用Crumb替换索引以及生成新集合,也需要$O_\lambda(\sqrt{\dbsize})$的计算量。客户端向Query服务器发送的查询大小为$\recordsize\sqrt{\dbsize} + O(\sqrt{\dbsize})$,向Hint服务器发送的查询大小为$O(\sqrt{\dbsize})$。每个服务器在$O(\recordsize\sqrt{\dbsize})$时间内计算校验值。Hint服务器回复$\sqrt{\dbsize}$条记录,Query服务器回复2个校验值。然后,客户端在$O(\recordsize\sqrt{\dbsize})$时间内验证答案并刷新Hint。
