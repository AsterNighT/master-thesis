\section{协议分析}

\label{sec:analysis}


\subsection{安全分析}
% {这里可以写好几页,也可以写一句话,看情况}
本节将聚焦于图\ref{fig:two-server-verify}中描述的双服务器协议,提供对该协议安全性证明的直观解释。单服务器协议类似于仅包含查询服务器的双服务器协议的在线阶段。详细的安全性证明可在附录\ref{appendix:security}中找到。

\paragraph{正确性}
只有当客户端无法找到包含查询索引的集合时,它才无法获取查询的记录。由于客户端持有$\hintcount = \magictotal$个Hint,这种情况发生的概率可以被忽略。由于来自服务器的诚实答案始终被接受,因此正确性的缺陷是可忽略的。

\paragraph{完整性}
设 $\blockidx = \lfloor \dbidx / \sqrt{\dbsize} \rfloor$表示包含查询索引的块,$r = Eval(\randomset, \blockidx)$是从查询中移除的随机元素。在$Reconstruct$算法中,客户端检查是否满足$\randomhint-\randomanswer = r \cdot (\sumhint-\sumanswer)$。恶意查询服务器提供的错误答案可以表示为$(\sumanswer+\Delta^+, \randomanswer+\Delta^\times)$,其中$(\Delta^+, \Delta^\times) \neq (0, 0)$。此检查可以简化为 $r\Delta^+ - \Delta^\times = 0$。在不知道$r$的情况下,此方程成立的概率为$2^{-\recordsize}$。

\paragraph{隐私性}
在离线阶段,生成的Hint和集合与任何特定的索引都无关。在在线阶段,客户端针对任何服务器的每个查询都会提供$[\sqrt{\dbsize}]$上的$\sqrt{\dbsize}$个随机数,以及$\recordfield$中的$\sqrt{\dbsize}$个随机元素,与查询的索引无关。

本文的协议能够抵抗“选择失败攻击”。值得注意的是,验证结果不会向Query服务器透露任何信息。如果服务器如实回答,由协议的正确性保证,客户端将通过验证并接受答案。如果服务器提供错误答案,根据完整性分析,客户端只有$2^{-\recordsize}$的概率接受该答案。因此,敌手的优势不会超过$2^{-\recordsize}$。因为 $\recordsize \ge \lambda$,这可以被视作一个关于 $\lambda$ 的可忽略函数。

\subsection{性能分析}
\paragraph{双服务器协议}
在离线阶段,客户端向Hint服务器发送一个大小为$O_\lambda(1)$的PRF密钥。Hint服务器将该密钥扩展为$\hintcount$个Hint密钥,并计算这些Hint。随后,服务器向客户端发送$\hintcount$个校验值和$\sqrt{\dbsize}$个Crumb。校验值的大小为$2\recordsize\hintcount$,$\sqrt{\dbsize}$个Crumb的大小为$\recordsize\sqrt{\dbsize} + O(\sqrt{\dbsize})$。因此,离线通信总量为$2\recordsize\hintcount + \recordsize\sqrt{\dbsize} + O(\sqrt{\dbsize})$,离线计算开销为$O_\lambda(\hintcount) + O(\recordsize\sqrt{\dbsize}\hintcount)$。

在在线阶段,客户端搜索包含查询索引$\dbidx$的Hint,预期计算开销为$O_\lambda(\sqrt{\dbsize})$。随后的Hint处理,包括展开集合、用Crumb替换索引以及生成新集合,同样需要$O_\lambda(\sqrt{\dbsize})$的计算量。客户端向Query服务器发送的查询大小为$\recordsize\sqrt{\dbsize} + O(\sqrt{\dbsize})$,向Hint服务器发送的查询大小为$O(\sqrt{\dbsize})$。每个服务器在$O(\recordsize\sqrt{\dbsize})$时间内计算校验值。Hint服务器回复$\sqrt{\dbsize}$条记录,Query服务器回复2个校验值。然后,客户端在$O(\recordsize\sqrt{\dbsize})$时间内验证答案并刷新Hint。

\paragraph{单服务器协议}
设$\hintcount = \hintcount_1 + \sqrt{\dbsize}\hintcount_2$。在离线阶段,将数据库流式传输到客户端需要$\recordsize\dbsize$的通信量,并且计算量为$O_\lambda(\sqrt{\dbsize}\hintcount) + O(\recordsize\sqrt{\dbsize}\hintcount)$。这与当前最先进的单服务器亚线性PIR协议\cite{Piano}一致。在在线阶段,成本与双服务器协议的Query服务器部分类似。因为不需要从$\sqrt{\dbsize}$条记录中重新计算Hint,客户端可以在$O(\recordsize)$时间更新Hint。单服务器协议与双服务器协议不同,必须在$\querycount$次查询后重新运行离线阶段。