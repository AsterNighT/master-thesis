\section{主要工作总结}
PIR协议受限于性能问题,在实际应用场景中使用受限。对查询请求的验证问题也是这一类服务外包场景的主要痛点之一。本文介绍了国内外研究现状,详细分析了现有的PIR协议的优势与局限性,提出了一种具有亚线性复杂度,可验证的PIR协议。在此基础上,本文提出了一种基于纠删码的分布式PIR协议框架,进一步提高了PIR协议的性能与内存限制问题。本文最后给出了这些工作在区块链中的实际应用场景,并通过实验验证了这一方案的可行性与性能。本文的主要研究成果如下:
\begin{enumerate}
    \item 提出了一种亚线性复杂度的可验证PIR协议。本文提出了一种基于离线-在线模型的亚线性PIR协议,并通过一种新颖的查询构造解决了目前研究中的虚构查询问题,使得协议能够抵抗选择错误攻击。实验表明,本文提出的协议不论在性能表现与运行成本上,都在一定情况下优于现有的PIR协议。
    \item 提出了一种基于纠删码的分布式PIR协议框架。本文提出了一种用于将数据库划分后放至于多台服务器上,进行分布式查询的,基于纠删码的PIR框架,能够解决目前PIR协议面临的内存、性能瓶颈问题。同时,该框架充分利用了底层协议的可验证性,大幅提高了编码的性能,使用较低的代价完成了分布式架构中的容错需求。
    \item 提出了一种基于区块链的PIR协议应用方案。本文提出了一种基于区块链的PIR协议应用方案,通过智能合约实现了PIR协议的验证过程,使得验证与支付过程原子化且公开透明,PIR协议能够与区块链结合应用。实验表明,本文提出的方案在黑名单查询场景中有着较高的性能,能够满足实际应用需求。
\end{enumerate}