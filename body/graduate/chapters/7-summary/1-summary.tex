\section{主要工作总结}
PIR协议因性能问题受到限制,在实际应用中受到一定的限制。同时,查询请求的验证问题也是服务外包场景下一个主要挑战。本文综述了国内外研究现状,详细分析了现有PIR协议的优点和局限性,并提出了一种具有亚线性复杂度且可验证的PIR协议。基于此,本文还提出了基于编码的分布式PIR协议框架,以进一步改善PIR协议的性能和内存限制问题。最后,本文探讨了这些工作在区块链中的实际应用,并通过实验验证了该方案的可行性与性能。本文的主要研究成果如下:
\begin{enumerate}
    \item 提出了一种亚线性复杂度的可验证PIR协议,通过一种新颖的查询构造解决了当前研究中的虚构查询问题,从而增强了协议对选择错误攻击的抵抗能力。实验结果表明,本文提出的协议在性能和运行成本方面在一定情况下优于现有PIR协议。
    \item 提出了一种基于编码的分布式PIR协议框架,通过将数据库划分并分布在多台服务器上进行分布式查询,解决了目前PIR协议面临的内存和性能瓶颈问题。该框架充分利用了底层协议的可验证性,显著提高了编码的性能,以较低的代价完成了分布式架构中的容错需求。
    \item 提出了一种基于区块链的PIR应用方案,通过智能合约实现了PIR协议的验证过程,使验证过程变得原子化和公开透明,从而实现PIR协议与区块链的结合应用。实验显示,本文提出的方案在黑名单查询场景中具有较高的性能,能够满足实际应用需求。
\end{enumerate}