\section{不足与展望}
本文提出的PIR协议迈出了可验证亚线性PIR的第一步。然而,本文所提出的亚线性PIR框架未能有效保护服务器数据库的隐私。在某些需要对称隐私的情境中,特别是需要同时保护客户端和数据库隐私的情况下,本文提出的方案并不能满足需求。因此,如何在亚线性PIR框架中保护数据库隐私仍然是学术界尚未解决的难题。
% \begin{enumerate}
%     \item 本文提出的单服务器PIR协议需要服务器将整个数据库传输至客户端。在大部分场景中,这种要求都是不现实的。然而,本文分析得出,以目前的技术水平,没有更好的解决方案。如何使单服务器亚线性PIR的离线阶段更较高效,摆脱传输数据库的需求,是未来的重要研究方向。
%     \item 本文提出的亚线性PIR框架无法保护服务器数据库的隐私,在一部分需要对称隐私的场景中,即需要保护客户端与数据库双方的隐私时,本文提出的方案无法满足需求。如何在亚线性PIR框架中保护数据库隐私,目前仍是学界的一个未解难题。
% \end{enumerate}