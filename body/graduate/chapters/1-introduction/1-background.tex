\section{研究背景及意义}

\todo{主要讲PIR技术的发展和应用场景,在这基础上展开PIR技术的重要性与局限性(内存占用,速度等等)。其次是区块链的一些特点,过渡到PIR技术在区块链上的应用(其实这里结合的不会很紧密,也就是本文提出的技术不会和区块链紧耦合,只是说可能比较适合区块链的场景,这里应该怎么说?)。其次需要强调一下可验证性与鲁棒性的意义,包括其处理恶意参与方的能力,以及处理系统故障地能力,强调这一点在分布式系统中是必不可少的。然后提一下选择失败攻击(selective failure)。最后,讨论一下数据库更新的必要性。}

随着互联网技术的发展,越来越多的数据被存储在云服务器中而非本地。用户在需要时访问服务器以获取他们需要的数据。这极大地拓展了数据的可用性与易用性,但在另一方面也带来了一些问题:用户的访问模式会泄露他们的隐私信息。例如,当一个用户进行DNS查询时,他的请求会暴露他想要访问的域名。另一个例子是密码泄露查询服务。当用户输入他们的密码时,服务提供者会告诉用户这个密码是否在以往的数据泄露事件中出现过。如果用户直接用明文密码进行查询,服务提供者就会知道用户的密码。如果我们考虑更为敏感的领域,如军工,政府,医疗等行业,这一问题会更加严重——现代互联网的数据放置模式要求我们使用更安全的方式访问数据。

在这一大前提下,密码学领域发明了PIR这一原语。PIR允许用户向服务器请求一个公开的数据库中的特定数据,但同时不泄露具体访问的是哪一条数据。这立刻为DNS,密码泄露查询等服务提供了可靠的解决方案。但是,PIR技术也有一些局限性。首先,PIR的效率相较于非PIR大幅下降,每次查询都需要双方进行大量的计算,这使得在现实中PIR的应用局限于较小的数据规模上。其次,PIR无法有效地利用传统数据库中的索引功能,这使得服务器往往需要将整个数据库加载进内存中,对服务器的内存容量提出了挑战。尽管自1995年PIR被提出以来 \cite{FOCS:CGKS95},PIR已经经历了长久的发展,对于大多数应用场景而言,目前的PIR协议仍然在通信量和计算量上过于昂贵。

另一方面,人们也逐渐意识到,保证服务的可靠性和正确性是非常必要的。如果我们将数据交付给云服务器,而后供他人进行查询,那么必须有某种措施来保障查询结果的正确性和服务的可用性。尤其在区块链这一类零信任的环境中,新型的DePIN等基础设施模型为服务的可验证性与可用性提出了挑战。为了将PIR拓展到区块链网络上,我们迫切地需要可验证,高可用的隐私计算服务。

本文期望提出了一套可用于区块链中使用的PIR协议框架,能够 (i) 降低PIR服务的计算量 (ii) 验证查询结果 (iii) 降低服务器存储压力,提高服务可用性,从而提高PIR在实际应用中的可用性。