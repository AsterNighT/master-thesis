\section{研究背景及意义}

随着互联网技术的发展,许多数据不再存储于本地,而被放置于在云服务器中。用户在需要时,向服务器请求数据。这极大地拓展了数据的可用性与易用性,但同时也带来了问题:用户的访问模式会泄露隐私。例如,当用户进行DNS查询时,DNS请求会暴露用户正在访问的域名。又以密码泄露查询服务为例,用户提供他们的密码,服务告知用户密码是否在以往的数据泄露事件中出现过。如果用户直接以明文密码进行查询,密码就泄露给了服务提供者。在隐私更为敏感的领域中,如军工,政府,医疗等行业,这一系列问题则更加严重——现代互联网的数据逻辑要求我们使用更安全的方式访问数据。

在这一大前提下,密码学领域发明了隐匿查询(PIR)这一原语。PIR允许用户向服务器请求数据库中的数据,但不泄露具体访问的是哪一条。这为DNS,密码泄露查询等服务提供了可能的解决方案。但是,PIR技术也有一些局限性。首先,相较于普通的数据库查询,PIR的效率要低得多,每次查询都需要双方进行大量的计算于通信。因此,现实中PIR的应用局限于较小的数据规模上。其次,PIR无法有效利用传统数据库中的索引功能,这使得服务器往往需要将整个数据库加载进内存中,对服务器的内存容量提出了挑战。自1995年PIR被提出以来 \cite{FOCS:CGKS95},这一原语已经经历了长久的发展,但对于许多应用场景而言,目前的PIR协议仍然在成本上过于昂贵。

从在线服务的另一个角度来看,人们也逐渐意识到,保证服务的可靠性和正确性是十分必要的。如果我们将数据交付给云服务器,而后供他人进行查询,那么必须有措施来保障查询结果的正确性和服务的可用性。尤其是在区块链这一类零信任的环境中,新兴的DePIN等基础设施模型性提出了挑战。一半时间宕机,另一半时间出错的服务是毫无价值的。为了将PIR拓展到更广泛的应用上,我们迫切地需要可验证,高可用的PIR协议。

本文期望提出一套可用于区块链中使用的PIR协议框架,能够 (i) 降低PIR服务的计算量 (ii) 验证查询结果 (iii) 降低服务器存储压力,提高服务可用性,从而提高PIR在实际应用中的可用性。