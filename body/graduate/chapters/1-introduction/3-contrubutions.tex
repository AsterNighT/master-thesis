\section{研究目标内容}

\subsection{研究目标}
结合前人研究成果,本文希望实现一种高效的PIR框架,能够在降低PIR计算量的同时,允许其在恶意安全模型下工作。同时,本文提出的协议旨在实际于区块链中应用,所以其应当具有鲁棒性,能够满足高可用的需求。本文提出的协议是切实可用的,其部署与使用具有灵活性,允许服务提供者适应自身需求以不同的方式部署并提供服务。同时,本文也提出一种基于区块链的服务激励模式,使得委托者可以容易地将数据库委托给链上的服务提供者以提供PIR查询服务。这种去中心化的服务模式更好地利用了算力资源,同时有利于委托者和用户双方。

\subsection{研究内容}

基于前文分析,目前尚没有针对亚线性——可验证PIR协议的研究,也缺少对于实践中PIR遇到的扩容难,落地难问题的研究。本文主要填补亚线性——可验证这一目前的技术空缺。同时,本文也提出一种与这一技术融合的PIR扩容与容灾方法,能够降低服务器的存储需求,并提供高可用的服务。本文的主要研究内容包括:
\begin{enumerate}
    \item 本文分析了现有的PIR技术的局限性,尤其目前许多PIR构造基于“虚构查询”。这种“虚构查询”需要使用者在查询时生成许多虚假的查询,以保证服务的隐私性。“虚构查询”的存在一方面使得服务提供者进行了大量无用计算,使查询的效率降低;另一方面也使得服务的正确性难以保证,给攻击者可乘之机。本文详细分析了“虚构查询”与“选择失败攻击”之间的联系,阐明了目前基于“虚构查询”的PIR协议的缺陷与为何它们易受“选择失败攻击”。
    \item 针对现有PIR技术的局限性,本文提出一种不依赖“虚构查询”的亚线性复杂度PIR协议。本文提出的协议包括两个版本,其一需要至少两台不共谋的服务器运行,其二只需要一台服务器。本文提出的协议是第一个可验证的,均摊计算与通信复杂度达到 $O_\lambda(\dbsize)$ 的PIR协议。协议在设计过程中着重防范了选择失败攻击,使得协议能够在保障正确性的同时不泄露用户查询的隐私。本文提出了一种插件形式的验证方法,在不改变渐近复杂度的前提下提供了高效的验证方法。对这两个版本的协议,本文给出了具体的协议流程,实现与测试。相较于领先的单服务器与多服务器协议,本文提出的协议在计算效率上6-7倍提升。
    \item 本文讨论了纠错码和纠删码于PIR中的应用,可验证性与纠错码之间的联系,提出了一种通用的,通过分片将可验证PIR算法变为高可用PIR算法,同时降低服务器负载的框架。将这一框架与本文提出的PIR协议结合,提出了一种高可用,高效的PIR服务方案。本文给出了这一框架的具体实现,测试与分析。利用这一框架与$\servercount$台服务器,服务提供者可以近似得到$\servercount$倍的性能与容量提升。
    \item 本文将这一框架与区块链技术相结合,提出了一种基于区块链的PIR服务激励模型。这一模型允许委托者通过区块链与服务提供者达成协议,通过区块链向服务提供者提供激励。本文将这一模型运用于“\projectname”项目中,用于提供高效的黑名单匹配服务。\todo{补数据}
\end{enumerate}
