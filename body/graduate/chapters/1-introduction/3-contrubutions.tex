\section{研究目标和内容}

\subsection{研究目标}
本文旨在设计一种高效的PIR协议,在降低PIR计算复杂度的同时,在恶意安全模型下实现其功能。基于此协议设计的应用框架应具备鲁棒性,以满足高可用性的要求,同时解决PIR协议在内存使用方面的挑战;该框架还应具有实用性,支持灵活的部署和使用,并允许服务提供者根据不同需求采用不同部署方式提供服务。在区块链环境中,该框架应支持公开的验证,利用智能合约提供服务。这种去中心化的服务模式更好地利用了计算资源,同时有利于委托者和用户双方。
\subsection{研究内容}

本文的主要研究内容包括:
\begin{enumerate}
    \item 本文分析了现有PIR技术的局限性,并提出了一种不依赖于“虚构查询”的亚线性PIR协议。该协议在亚线性PIR的基本框架\cite{EC:CorKog20}基础上,采用了“划分与采样”、伪随机函数以及额外的Hint技术,重新设计了具体的查询流程,从而极大提高了服务器的计算效率。本文还提出了一种采用外包计算中常见的隐藏系数进行验证的插件形式验证方法,旨在提供可验证性而不增加渐近复杂度。所提出的协议共包括两个版本,一个需要至少两台不共谋的服务器运行,另一个仅需要一台服务器。对这两个版本的协议,本文提供了详细的协议流程、实现和测试。本文提出的协议是第一个可验证的、均摊计算与通信复杂度达到 $O_\lambda(\dbsize)$ 的PIR协议。协议在设计过程中着重防范了“选择失败攻击”,使得协议能够在保障正确性的同时不泄露用户查询的隐私。相较于目前最优的单服务器与多服务器协议,本文提出协议的在线计算效率有所提升。
    \item 本文讨论了编码技术于PIR中的应用,以及可验证性与纠错码之间的联系,提出了一种基于编码的框架,能够通过编解码将数据库分片,为PIR协议提供高可用性,同时降低服务器负载。本文将这一框架与提出的PIR协议结合,给出了这一框架的具体实现、测试与分析。利用这一框架与$\servercount$台服务器,服务提供者可以近似得到$\servercount$倍的性能与容量提升。
    \item 本文将这一框架与区块链应用相结合,提出了一种基于智能合约的公开验证方式。这一验证方式允许委托者通过智能合约中的群运算完成答案验证,以此为基础构造区块链应用。本文将这一模型运用于“\projectname”国家重点研发计划项目中,用于提供高效的黑名单匹配服务。实践中,本文的协议可以提供6000左右的QPS,具备较高性能。
\end{enumerate}
