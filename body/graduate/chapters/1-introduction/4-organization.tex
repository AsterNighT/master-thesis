\section{本文组织结构}

本文围绕PIR的构造、优化与应用展开,共分为七章。具体的章节组织如下:
\begin{enumerate}
    \item 第一章概述了PIR作为密码学原语的现实意义,提出了本文研究的核心问题,并介绍了国内外PIR研究的现状。此外,阐述了本文的研究目标与内容,并最后概括介绍了本文的组织结构。
    \item 第二章介绍了本文所依赖的背景知识、密码学原语等,并提供了具体的形式化定义与符号约定以便读者查阅。
    \item 第三章以问题——答案的形式详细描述了PIR在实际应用中所面临的性能、扩展性和可靠性三个难点。通过回答这三个问题,揭示了本文总体技术路线:通过高效的亚线性复杂度协议提升PIR在应用中的性能;通过分片和纠错码提高PIR的扩展性和可靠性;通过可验证计算确保服务的可靠性。
    \item 第四章介绍了本文提出的高性能PIR协议。本章首先介绍了亚线性PIR的基础协议,并提出了其中的三个基本问题。随后,介绍了主流研究对这三个问题的解决方案,并指出了其中存在的问题。接着介绍了本文的解决方案,解释了与其他方案相比的优缺点。在此基础上,阐述了如何将其转化为一个可验证的计算方案,使该协议能够扩展至恶意安全模型。随后,介绍了如何将该协议转化为单服务器PIR协议,并进行了安全性分析,探讨数据库更新的处理方法。最后,给出了该协议的实验分析。
    \item 第五章提出了一种基于编码的PIR框架,并探讨了PIR协议的可验证性与编码之间的联系,指出PIR的可验证性可以显著提升编码效率,随后对该协议进行了安全性和效率方面的分析,并提供了实验分析结果。
    \item 第六章介绍了本文提出的框架在“\projectname”国家重点研发计划项目中的应用,包括项目背景、解决方案、框架在项目中的具体实施方案以及达到的效果。
    \item 第七章对全文内容进行了总结概括,指出了本文的创新点和不足之处,并展望了未来的研究方向。
\end{enumerate}