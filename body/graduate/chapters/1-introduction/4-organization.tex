\section{本文组织结构}
{修改和补完内容}

本文围绕PIR的构造,优化与应用展开,主要分为七章。章节的组织形式如下:
\begin{enumerate}
    \item 第一章绪论,概述了PIR这种密码学原语的现实意义,提出了本文的核心研究问题,介绍了国内外PIR的研究现状,阐述了本文的研究目标与内容,最后介绍了本文的组织结构。
    \item 第二章相关技术分析,介绍了本文所依赖的背景知识、密码学原语等,同时提供了具体的形式化定义与符号约定以供读者查询。
    \item 第三章总体需求与框架设计,以问题——答案的形式,详细描述了PIR在实际应用中所面临的性能、拓展、可靠三个难点,通过回答这三个问题,展开本文的总技术路线:通过高效的亚线性复杂度协议,提高PIR在应用中的性能;通过分片与纠错码提高PIR的拓展性、可用性;通过可验证计算保障服务的可靠性。
    \item 第四章亚线性复杂度的PIR,介绍本文提出的高性能PIR协议。本章首先介绍了亚线性PIR协议的基本框架和模型,提出了这一模型中的三个基本问题,介绍了目前主流研究对这三个问题的解决方案,并指出其中存在的问题,随后给出了本文的方案,解释了其相较于其他方案的优缺点。在这一方案的基础上,本章阐述了如何将其转化为一个可验证的计算方案,使得这一协议可以拓展到恶意安全模型上。在这之后,本章介绍了如何将这一协议转化为单服务器PIR协议,并给出了安全分析,讨论了如何处理数据库更新。最后,本章给出了这一协议的实验分析。
    \item 第五章基于编码的PIR框架,提出了一种基于分片与编码的PIR框架,讨论了PIR协议的可验证性与编码之间的联系,指出PIR的可验证性可以大幅优化编码效率,随后对这一协议给出了安全与效率上的分析。最后,本章讨论了这一框架如何处理数据库更新,以及给出了实验分析。
    \item 第六章PIR于区块链中的应用,介绍了本文提出的框架于“\projectname”项目中的应用,给出了项目的背景,本文提出的框架在项目中的具体实施方案,以及达成的效果。
    \item 第七章总结与展望,对全文的内容进行了总结概括,指出了本文的创新点与不足之处,并对未来的研究方向进行了展望。
\end{enumerate}