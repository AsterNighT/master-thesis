\section{符号约定}
{确定这一章放哪里}
在本文中,我们假设服务器\server 存储了一份数据库 \db。如果服务器与数据库超过一个,我们使用上标进行标注区分,如$\db^{1},\db^{2}$等。我们假设数据库是一个长度为 \dbsize 的向量,其中每个位置存储了大小为 \recordsize 比特的记录。换言之,我们假设每个位置上的值是\recordfield 中的一个元素。我们使用 $\db = (\db_0,\dots,\db_{\dbsize-1})$ 来索引这些元素。我们假设 \dbsize 是一个完全平方数。如果它不是,我们总可以向数据库中填充不超过 $\sqrt{\dbsize}$ 个元素以达成这一要求。我们使用 $\lambda$ 表示安全参数。我们假设$\recordsize \ge \lambda$,如果这一条件不能满足,服务器可以将数个较小的记录合并为一个较大的记录。我们使用 $\negl(\cdot)$ 表示可忽略函数。

我们使用形如 $[k]$ 的记号来表示集合 $\{0,1,\dots,k-1\}$。如果这一记号被用在一个向量$S = (S_0, S_1, \dots)$上,$S[j]=S_j$ 表示向量中下表为$j$的元素。$x\leftarrow S$ 表示从集合$S$中随机选择元素$x$。我们使用``$\coloneqq$''来表示赋值。为了更加准确地描述效率,我们在后续的复杂度分析中明确地将 $\recordsize$ 包含在渐进复杂度和具体复杂度中。
