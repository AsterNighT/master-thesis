\section{伪随机函数}
伪随机函数(Pseudorandom Function,PRF)是一类特殊的函数。我们首先给出PRF的定义\cite{GolGolMic86}:
\begin{definition}[伪随机函数族]
    一个函数族 $\{f_{key}\}$ 接受输入 $x\in\{0,1\}^{I(n)}$ 并输出 $y\in\{0,1\}^{O(n)}$,其中 $key\in S_k, n \coloneqq |S_k|$。即:
    $$ f_{key}: \{0,1\}^{I(n)} \rightarrow \{0,1\}^{O(n)}$$
    如果如下三个条件满足,则称 $\{f_{key}\}$ 是一个伪随机函数族:
    \begin{itemize}
        \item \textbf{可索引性}:$S_k$与$\{f_{key}\}$的映射关系构成双射。
        \item \textbf{可计算性}:对于任意 $key\in S_k$ 和 $x$,$f_{key}(x)$ 可以在多项式时间内计算。
        \item \textbf{伪随机性}:定义$RF$为从所有$\{0,1\}^{I(n)} \rightarrow \{0,1\}^{O(n)}$的函数集合中均匀采样得到的分布,$F$为$f_{key}$的分布(其中$key$从$S_k$中均匀采样)。对于任意多项式时间的区分器 $\adversary$,存在一个可忽略函数 $\negl(\cdot)$,使得:
        $$\left| \Pr[\adversary^{F}(1^n) = 1] - \Pr[\adversary^{FR}(1^n) = 1] \right| \le \negl(n)$$
    \end{itemize}
\end{definition}

简言之,一个PRF是一组函数之一,它将输入映射到输出。对于给定的输入,PRF的输出是确定的。然而,攻击者难以通过某个输入的输出来预测未知输入的输出。这种性质使得PRF在密码学中得到广泛应用。将映射关系压缩为大小为$n$的$key$,大大减少了映射关系的传输和存储成本。

在实践中,PRF常使用AES加密\cite{1250456}实现。AES是一种对称加密算法,接受一个128位输入和一个128位密钥,并输出一个128位结果。我们可视AES为PRF,其中输入、输出和$key$长度均为128位。AES的加密安全性能保证了PRF的伪随机性。由于AES得到广泛硬件支持\cite{1190589},其加密和解密速度在大多数硬件平台上都非常快。