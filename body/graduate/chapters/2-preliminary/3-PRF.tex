\section{伪随机函数}
{讲一下PRF,如何用PRF实现PRS,有什么用}
伪随机函数(Pseudorandom Function,PRF)是一类特殊的函数。我们首先给出PRF的定义:
\begin{definition}[伪随机函数族]
    一个函数族 $\{f_{key}\}$ 接受输入 $x\in\{0,1\}^{I(n)}$ 并输出 $y\in\{0,1\}^{O(n)}$,其中 $n \coloneqq |key|$。即:
    $$ f_{key}: \{0,1\}^{I(n)} \rightarrow \{0,1\}^{O(n)}$$
    如果如下两个条件满足,则称 $\{f_{key}\}$ 是一个伪随机函数族:
    \begin{itemize}
        \item \textbf{可计算性}。对于任意 $key$ 和 $x$,$f_{key}(x)$ 可以在多项式时间内计算。
        \item \textbf{伪随机性}。定义$RF$为从所有$\{0,1\}^{I(n)} \rightarrow \{0,1\}^{O(n)}$的函数集合中均匀采样得到的分布,$F$为$f_{key}$的分布(其中$key$从$\{0,1\}^{n}$中均匀采样)。对于任意多项式时间的区分器 $\adversary$,存在一个可忽略函数 $\negl(\cdot)$,使得:
        $$\left| \Pr[\adversary^{F}(1^n) = 1] - \Pr[\adversary^{FR}(1^n) = 1] \right| \le \negl(n)$$
    \end{itemize}
\end{definition}

以更简洁的方式来描述,一个PRF是一系列函数中的一个,它将一个输入映射到一个输出。对于一个给定的输入,PRF的输出是确定的。然而,攻击者难以通过一个PRF对于某些输入的输出来预测某个未知输入对应的输出。这一性质使得PRF在密码学中有着广泛的应用。我们可以将一个映射关系压缩为大小为$O(n)$的$key$,大大压缩了映射关系的传输和存储成本。

实践中,PRF可以使用AES加密来实现。AES是一种对称加密算法,它接受一个128位的输入和一个128位的密钥,并输出一个128位的输出。因此,我们可以将AES看作是一个PRF,其中输入、输出和$key$的长度都是128位。AES的加密安全性能够保证PRF的伪随机性。由于AES有广泛的硬件支持,其加密与解密速度在大部分硬件平台上非常快。