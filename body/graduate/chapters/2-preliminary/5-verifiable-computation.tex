\section{可验证计算}
外包计算(Outsource Computation,OC) \cite{USENIX:GreHohWat11, C:ChuKalVad10} 是一种客户端-服务器模型,允许客户端将计算过程委托给服务器,而客户端仅提供输入并接受输出。一般我们认为,相较于服务器而言,客户端通常计算能力有限。这一模型使得客户端可以利用服务器的计算能力来完成复杂的任务。在这一过程中,一个重要的问题是如何验证服务器的计算结果。可验证计算(Verifiable Computation,VC)是解决这一问题的一种方案。在VC中,服务器需要提供一个证明,证明其计算结果的正确性。客户端可以通过验证这一证明来确保服务器的计算结果的正确性。显然,验证这一证明的过程必须是高效的,验证效率需要高于计算效率,否则客户端没有必要将计算外包,可以直接计算结果。

PIR可以看作一种特殊的外包计算。其特殊之处在于,客户端外包给服务器的是数据库的存储而非计算。同时,客户端还要保护己方输入,不能将明文请求直接传输给服务器。在传统的PIR协议中,往往不考虑服务器给出的查询结果是否可靠。换言之,传统PIR协议假设服务器是半诚实的。然而如前文所述,验证计算结果一直是外包计算的一个重要构成部分。如 \cite{VeriSimplePIR, APIR, SVPIR18} 等工作拓展了这一场景,讨论了如何与恶意的服务器进行PIR查询。我们首先给出一个基于离线-在线PIR框架的可验证PIR定义:

\begin{definition}[可验证PIR]
    一个\textit{可验证PIR}方案$\Pi$允许客户端从数据库$\db$中检索记录$\db_\dbidx$,而不向$\servercount$个服务器中任何一个泄露索引$\dbidx$。该方案由算法元组$\Pi = (Setup, Hint, Query,\\ Answer, Reconstruct, Refresh)$组成:

    离线部分:
    \begin{itemize}[leftmargin=*]
        \item $Setup(1^\lambda,\dbsize) \rightarrow \query_\hint$。给定数据库大小$\dbsize$和安全参数$\lambda$,生成Hint查询$\query_\hint$。
        \item $Hint(\db, \query_\hint) \rightarrow \hint$。给定数据库$\db$和Hint查询$\query_\hint$,生成Hint $\hint$。
    \end{itemize}
    在线部分:
    \begin{itemize}[leftmargin=*]
        \item $Query(\hint, \dbidx) \rightarrow (\query, \clientstate)$。给定Hint $\hint$和要查询的索引$\dbidx$,生成查询$\query$。注意,查询$\query$可能包含多个子查询。客户端生成并保存一个私有状态$\clientstate$。
        \item $Answer(\db, \query) \rightarrow \answer$。给定查询$\query$,生成回答$\answer$。
        \item $Reconstruct(\clientstate, \hint, \answer) \rightarrow \{\db_\dbidx, \bot\}$。给定回答$\answer$,使用Hint $\hint$和私有状态$\clientstate$重构出记录$\db_\dbidx$,或是拒绝并输出 $\bot$。
        \item $Refresh(\clientstate, \hint, \answer) \rightarrow \hint$。给定回答$\answer$和私有状态$\clientstate$,更新Hint $\hint$。
    \end{itemize}
    可验证PIR算法除了满足\textbf{正确性}与\textbf{隐私性}之外,还需要额外满足\textbf{完整性}:
    \begin{itemize}
        \item \textbf{完整性}:对于任意安全参数$\lambda$,任意数据库$\db$和索引$\dbidx$,任意概率多项式时间敌手 $\adversary$作为服务器之一与客户端交互时,客户端输出一个错误答案 $\db_\dbidx' \neq \db_\dbidx$ 的概率小于 $\negl(\lambda)$。
    \end{itemize}
\end{definition}

\subsection{选择失败攻击}
在外包计算的过程中,客户端如果发现服务器的不诚实行为,可以提出证明并拒绝为服务付费 \cite{chen2012efficient, carbunar2011payments}。然而,在包含PIR在内的一部分场景中,这种证明可能是危险的。服务器可以通过一种称作“选择失败”的攻击方式,利用这种证明收集客户端的查询信息。

选择失败攻击的概念最早提出于2006年 \cite{Kiraz2006API}。这种攻击利用了“证明验证失败”这一知识。为展开选择失败攻击,恶意的服务器刻意构造一些错误的答案,以期从客户端的反应中推测客户端的输入。举例而言,假设客户端采用数字签名分别保护数据库中每一条记录的可靠性。服务器针对性地修改了第一条记录,将其改为了某些不能通过数字签名验证的错误值。那么,如果客户端请求的正是第一条记录,它就会拒绝服务器的答案。相反,如果客户端请求的不是第一条记录,客户端就会接受服务器的答案。由此,服务器可以确认客户端是否在请求第一条记录。这与PIR的要求相违背。

因此,在本文中,为了突出这种攻击的重要性,我们采用了一种更严苛的隐私性定义:

\begin{definition}[抗选择失败攻击的隐私性]
    \label{def:privacy-sfa}
    对于任意计算安全参数$\lambda$,一个可验证PIR协议 $\Pi$ 是\textit{抗选择失败攻击隐私}的,当且仅当对于任意数据库$\db$,存在一概率多项式时间模拟器 $\simulator(1^\lambda, \dbsize)$,使得对于控制不超过$\servercount$台服务器中$\threshold$台的任意概率多项式时间敌手 $\adversary$,$\adversary$无法以超过 $\negl(\lambda)$的优势区分下述两个世界:
    \begin{itemize}
        \item \textbf{世界 0}: 在每个查询时刻 $t$,$\adversary$ 选择下一次查询的索引 $\dbidx_t$, $\client$ 将 $\dbidx_t$ 作为查询的索引。查询完成后,如果验证通过,客户端向 $\adversary$ 输出 $1$ ,否则输出 $0$.
        \item \textbf{世界 1}: 在每个查询时刻 $t$,$\adversary$ 选择下一次查询的索引 $\dbidx_t$。$\simulator$ 在不知道  $\dbidx_t$ 的情况下执行协议, 但是如果 $\adversary$ 给出了正确的答案,$\simulator$会收到一个比特 $1$,否则收到 $0$。 $\simulator$ 在查询完成后将这个比特输出给 $\adversary$。
    \end{itemize}

    $\adversary$ 可以以任意方式背离原协议。
\end{definition}

对于没有验证环节的PIR而言,这一定义与原先的定义没有区别。由于所有的答案都会被接受,无验证的PIR显然是抗选择失败攻击的。在可验证的PIR中,协议必须仔细分析由验证带来的隐私泄露。一些先前的工作如\cite{APIR, VeriSimplePIR}将数据库与信息摘要绑定,并进行线性运算以检查整个数据库,以此防御选择失败攻击。然而,在这些协议中,服务器处理一次查询的时间至少是线性的,无法用于亚线性复杂度的PIR协议。