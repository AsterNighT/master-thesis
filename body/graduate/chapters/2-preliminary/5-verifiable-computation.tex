\section{可验证计算}
外包计算(Outsource Computation,OC)\cite{USENIX:GreHohWat11, C:ChuKalVad10} 是一种客户端-服务器模型,允许客户端将计算过程委托给服务器,客户端仅需提供输入并接受输出。一般来说,客户端的计算能力通常受限,故客户端可以利用服务器的计算资源来执行复杂的任务。在这一过程中,一个关键问题是如何验证服务器的计算结果。可验证计算(Verifiable Computation,VC)为解决这一问题提供了一种解决方案。在VC中,服务器必须生成一个证明以验证其计算结果的正确性,客户端通过验证该证明来确保服务器计算的正确性。显然,证明的验证过程应高效,且其效率应高于计算效率,否则客户端将不会选择将计算外包,而会自行完成计算。

PIR可视为一种特殊形式的外包计算。其独特之处在于,客户端将数据库存储的任务外包给服务器,而非计算任务。同时,客户端还需保护其输入,以防止将明文查询直接传输给服务器。传统PIR协议通常并未考虑服务器提供的查询结果是否可信,即假定服务器半诚实。然而,根据前文所述,验证计算结果一直是外包计算的关键组成部分。一系列工作\cite{VeriSimplePIR, APIR, SVPIR18}拓展了这一场景,探讨了如何与恶意服务器进行PIR查询。我们首先给出一个基于离线-在线PIR框架的可验证PIR定义:

\begin{definition}[可验证PIR]
    一个\textit{可验证PIR}协议$\Pi$允许客户端从数据库$\db$中检索记录$\db_\dbidx$,而不向$\servercount$个服务器中任何一个泄露索引$\dbidx$。该协议由算法元组$\Pi = (Setup, Hint, Query,\\ Answer, Reconstruct, Refresh)$组成:

    \noindent \textbf{离线阶段:}
    \begin{itemize}
        \item $Setup(1^\lambda,\dbsize) \rightarrow \query_\hint$。给定数据库大小$\dbsize$和安全参数$\lambda$,生成Hint查询$\query_\hint$。
        \item $Hint(\db, \query_\hint) \rightarrow \hint$。给定数据库$\db$和Hint查询$\query_\hint$,生成Hint $\hint$。
    \end{itemize}
    \noindent \textbf{在线阶段:}
    \begin{itemize}
        \item $Query(\hint, \dbidx) \rightarrow (\query, \clientstate)$。给定Hint $\hint$和要查询的索引$\dbidx$,生成查询$\query$。注意,查询$\query$可能包含多个子查询。客户端生成并保存一个私有状态$\clientstate$。
        \item $Answer(\db, \query) \rightarrow \answer$。给定查询$\query$,生成回答$\answer$。
        \item $Reconstruct(\clientstate, \hint, \answer) \rightarrow \{\db_\dbidx, \bot\}$。给定回答$\answer$,使用Hint $\hint$和私有状态$\clientstate$重构出记录$\db_\dbidx$,或是拒绝并输出 $\bot$。
        \item $Refresh(\clientstate, \hint, \answer) \rightarrow \hint$。给定回答$\answer$和私有状态$\clientstate$,更新Hint $\hint$。
    \end{itemize}
    可验证PIR算法除了满足\textbf{正确性}与\textbf{隐私性}之外,还需要额外满足\textbf{完整性}:
    \begin{itemize}
        \item \textbf{完整性}:对于任意安全参数$\lambda$,任意数据库$\db$和索引$\dbidx$,任意概率多项式时间敌手 $\adversary$作为服务器之一与客户端交互时,客户端输出一个错误答案 $\db_\dbidx' \neq \db_\dbidx$ 的概率小于 $\negl(\lambda)$。
    \end{itemize}
\end{definition}

\subsection{选择失败攻击}
在外包计算的过程中,如果客户端发现服务器存在不诚实行为,便可提出证据并拒绝为服务付费 \cite{chen2012efficient, carbunar2011payments}。然而,在包含PIR的场景中,这种证据可能存在危险性。服务器可以利用一种称为“选择失败”的攻击方式,通过这种手段来收集客户端的查询信息。

“选择失败攻击”的概念最早于2006年提出 \cite{Kiraz2006API}。这种攻击利用了验证失败提供的信息。在进行“选择失败攻击”时,恶意服务器会故意制造一些错误的答案,试图从客户端的反应中推断客户端的输入。比如,假设客户端使用数字签名来确保数据库中每条记录的可靠性,服务器就会有针对性地修改第一条记录,将其改为无法通过数字签名验证的错误数值。当客户端请求第一条记录时,它将会拒绝服务器的答案;而如果请求的不是第一条记录,就会接受服务器的答案。这使得服务器能够确认客户端是否请求了第一条记录,与PIR的要求相悖。

因此,在本文中,为了突出这种攻击的重要性,我们采用了一种更严苛的隐私性定义:

\begin{definition}[抗“选择失败攻击”的隐私性]
    \label{def:privacy-sfa}
    对于任意计算安全参数$\lambda$,一个可验证PIR协议 $\Pi$ 是\textit{抗“选择失败攻击”隐私}的,当且仅当对于任意数据库$\db$,存在一概率多项式时间模拟器 $\simulator(1^\lambda, \dbsize)$,使得对于控制不超过协议$\Pi$定义内$\servercount$台服务器中$\threshold$台的任意概率多项式时间敌手 $\adversary$,$\adversary$无法以超过 $\negl(\lambda)$的优势区分下述两个世界:
    \begin{itemize}
        \item \textbf{世界 0}: 在每个查询时刻 $t$,$\adversary$ 选择下一次查询的索引 $\dbidx_t$, $\client$ 将 $\dbidx_t$ 作为查询的索引。查询完成后,如果验证通过,客户端向 $\adversary$ 输出 $1$ ,否则输出 $0$.
        \item \textbf{世界 1}: 在每个查询时刻 $t$,$\adversary$ 选择下一次查询的索引 $\dbidx_t$。$\simulator$ 在不知道  $\dbidx_t$ 的情况下执行协议, 但是如果 $\adversary$ 给出了正确的答案,$\simulator$会收到一个比特 $1$,否则收到 $0$。 $\simulator$ 在查询完成后将这个比特输出给 $\adversary$。
    \end{itemize}

    $\adversary$ 可以以任意方式背离原协议。
\end{definition}

对于没有验证环节的PIR而言,这一定义与原先的定义没有区别。由于所有的答案都会被接受,无验证的PIR显然是抗“选择失败攻击”的。在可验证的PIR中,协议必须仔细分析由验证带来的隐私泄露。一些先前的工作 \cite{APIR, VeriSimplePIR} 将数据库与信息摘要绑定,并进行线性运算以检查整个数据库,以此防御“选择失败攻击”。然而,在这些协议中,服务器处理一次查询的时间至少是线性的,无法用于亚线性复杂度的PIR协议。