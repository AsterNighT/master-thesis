\section{纠错码}
\label{sec:preliminary-code}
纠错码(Error correcting code)是一类用于在消息传输中检测错误的前向纠错技术(Forward error correction, FEC)。在传输过程中,消息可能会受到干扰,导致接收方接收到的消息与发送方发送的消息不一致。纠错码通过在消息中添加冗余信息,使得接收方可以检测并纠正错误。典型的例子包括汉明码(Hamming code)。以下首先给出一个简化版的纠错码定义\cite{6772729}:

\begin{definition}[纠错码]
    记$w(x)$为向量$x$的汉明重量,$(n,m,e)$编码协议$\Pi$允许将消息从$\mathbb{F}^n$编码至$\mathbb{F}^m$,并且进行反向解码。其由算法元组$\Pi = (Encode, Decode)$组成:
    \begin{itemize}
        \item $Encode(x) \rightarrow c$。给定消息向量$x\in \mathbb{F}^n$,生成编码向量$c\in \mathbb{F}^m$。
        \item $Decode(c) \rightarrow \{x,\bot\}$。给定编码向量$c\in \mathbb{F}^m$,生成消息向量$x\in \mathbb{F}^n$或是解码失败。 
    \end{itemize}
    如果对于任意消息向量$x\in \mathbb{F}^n$,对于任意错误向量$\Delta\in \mathbb{F}^m, w(\Delta)\le e$,进行编码$c = Encode(x)$后,有$Decode(c+e) = x$。那么称$\Pi$是一个$(n,m,e)$纠错码。
\end{definition}

由基本的信息论要求可知,始终有 $n\le m$。在此基础上,一旦 $n$ 和 $e$ 确定,不同的编码协议将需要不同大小的 $m$,且在编码的时间效率上也各有优劣。我们将 $n/m$ 称为编码的利用率。对于给定的 $n$ 和 $e$ 值,利用率越高,编码占用的空间越小,传输和存储方面的效率就越高。基于此概念,有一类编码协议被称为纠删码(Erasure code)\cite{10.1145/263876.263881}。纠删码是一种特殊的纠错码,其编码利用率高于一般的纠错码,但在解码时要求知道错误的位置。换言之,纠删码可以处理数据丢失,但无法纠正数据误差。以下是纠删码的定义:

\begin{definition}[纠删码]
    记$w(x)$为向量$x$的汉明重量,$(n,m,e)$编码协议$\Pi$允许将消息从$\mathbb{F}^n$编码至$\mathbb{F}^m$,并且进行反向解码。其由算法元组$\Pi = (Encode, Decode)$组成:
    \begin{itemize}
        \item $Encode(x) \rightarrow c$。给定消息向量$x\in \mathbb{F}^n$,生成编码向量$c\in \mathbb{F}^m$。
        \item $Decode(c, \delta) \rightarrow \{x,\bot\}$。给定编码向量$c\in \mathbb{F}^m$与错误位置向量$\delta\in \{0,1\}^m$,生成消息向量$x\in \mathbb{F}^n$或是解码失败。
    \end{itemize}
    如果对于任意消息向量$x\in \mathbb{F}^n$,对于任意错误向量与错误位置向量$\Delta\in \mathbb{F}^m, \delta\in \{0,1\}^m, w(\Delta)=w(\delta)\le e$,且满足 $\Delta$ 与 $\delta$ 非零位置一致。进行编码$c = Encode(x)$后,有$Decode(c, \delta) = x$。那么称$\Pi$是一个$(n,m,e)$纠删码。
\end{definition}

显然,纠删码在功能上不及纠错码。然而,在利用率上,纠删码可以达到信息论中的理论上界:即$(n,m,e)$ 中有$m=n+e$。在这种情况下,可以从编码后向量中的任意$n$个位置中还原原始消息。这便是所谓的“最优纠删码”。Reed-Solomon编码是一种常见的纠删码方案,也是最佳纠删码的一种实现,即可实现$m=n+e$。本文采用Reed-Solomon编码作为主要的纠删码实现。Reed-Solomon编码的核心原则在于多项式插值和求值,与本文关注的纠删码构建方法及底层原理无关,因此不再深入探讨。
