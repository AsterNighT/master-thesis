\section{协议分析}
本节对上一节提出的PIR协议进行安全性和性能分析。

\subsection{安全分析}
对于客户端而言,其相当于进行了$\servercount$次单服务器PIR协议$\Pi$。因此,只要$\Pi$满足隐私性要求,整个协议也满足隐私性。由于我们要求$\Pi$是一个可验证的PIR协议,因此每台服务器给出的答案都是真实的码点或是$\bot$。在出错或恶意的服务器不超过$\threshold$时,至少有$\servercount-\threshold$个正确答案。由纠删码的性质可以知道,$Decode$算法必然能恢复出正确的记录。

\subsection{性能分析}
由于数据库划分与编码,原数据库$\db$被转化为$\servercount$个$\frac{\dbsize}{\servercount-\threshold}$大小的子数据库。对于每个子数据库,其PIR协议的渐进复杂度是不变的。如果原PIR协议的复杂度为$T(\dbsize)$,则每个子数据库对应的复杂度为$T(\frac{\dbsize}{\servercount-\threshold})$,总复杂度为$\servercount \cdot T(\frac{\dbsize}{\servercount-\threshold})$。由于前述的优化,客户端计算复杂度保持不变。