\section{协议分析}
本节对上一节提出的PIR框架进行安全性和性能分析。

\subsection{安全分析}
\paragraph{正确性}
框架的正确性可以简单地由PIR协议$\Pi$的正确性与编码协议$\Pi_c$的正确性所保证。

\paragraph{完整性}
由于我们要求$\Pi$是一个可验证的PIR协议,因此每台服务器提供的答案都是真实的码点或$\bot$。在出错或恶意的服务器不超过$\threshold$个时,至少有$\servercount-\threshold$个正确答案。根据编码的性质,$Decode$算法必然能恢复出正确的记录。

\paragraph{隐私性}
对于客户端而言,其相当于执行了$\servercount$次单服务器PIR协议$\Pi$。然而,从服务器的联合视角来看,客户端发送的消息是完全相同的。相对于原始协议$\Pi$,服务器并未获得任何额外的信息。因此,只要$\Pi$满足隐私性要求,整个框架也满足隐私性。

\subsection{性能分析}
由于数据库的划分和编码,原数据库$\db$被转化为$\servercount$个大小为$\frac{\dbsize}{\servercount-\threshold}$的子数据库。对于每个子数据库,其PIR协议的渐进复杂度是不变的。如果原PIR协议的通信和计算复杂度为$T(\dbsize)$,则每个子数据库对应的复杂度为$T(\frac{\dbsize}{\servercount-\threshold})$,总复杂度为$\servercount \cdot T(\frac{\dbsize}{\servercount-\threshold})$。