\section{基于编码的PIR框架测试}
\subsection{实验设计}
本文使用Rust语言实现了基于编码划分的PIR协议框架。我们选择了前文提出的单服务器亚线性PIR协议作为框架的PIR协议。根据数据库规模$\dbsize$、安全参数$\lambda$和查询次数$\querycount$计算参数,确保安全级别至少为 $\lambda=128$,并支持至少$\querycount=\sqrt{\dbsize}\log{\dbsize}$次查询($\dbsize$表示数据库中的记录数)。每台服务器都以单线程运行PIR协议,并取10次运行结果的平均值。数据传输时间不包含在时间计算中,但总通信量会明确给出。实验在装备有Intel(R) Xeon(R) E-2288G CPU @ 3.70GHz的多台服务器上进行。我们对一个大小为32GiB的数据库进行编码后,分发给8、32和128台服务器,且纠错阈值均设置为服务器总数的一半。我们对比了 (i) 直接运行单服务器PIR协议 和 (ii) 在不同分片后运行基于编码划分的PIR协议。由于离线阶段可以采用与先前完全一致的方法,无需在编码后的数据库上运行,我们只测试了在线阶段的性能。在实验中,我们假定一半的数据由于服务器恶意行为或网络故障而丢失,因此只有一半的服务器回答可用。

\begin{table}[]
    \caption{不同分片数量的PIR性能比较}
    \label{tab:sharded-scheme}
    \centering
    \begin{tabular}{@{}c|cc|cccc@{}}
    \toprule
    分片数量 & \multicolumn{2}{c|}{通信量(KiB)} & \multicolumn{4}{c}{时间(ms)}         \\ \midrule
         & 下载      & 上传        & Query & Answer & Reconstruct & 共计 \\ \midrule
    单机   & 0.06          & 1152.02       & 1.17  & 3.29   & 0.00        & 4.46  \\
    8    & 0.48          & 5,120.16      & 0.52  & 1.46   & 0.00        & 1.98  \\
    32   & 1.92          & 10240.64      & 0.22  & 0.52   & 0.00        & 0.73  \\
    128  & 7.68          & 20482.56      & 0.08  & 0.15   & 0.00        & 0.23  \\ \bottomrule
    \end{tabular}
\end{table}
\subsection{性能结果分析}

实验结果如表\ref{tab:sharded-scheme}所示:随着分片数量的增加,计算时间逐渐减少,编码和解码带来的开销非常小,在实验中难以体现。可以看出,在数据库规模较大时,采用纠删码划分的PIR协议可以进一步降低计算时间,提高性能。

相对于计算速度的提升,通信量却有所增加。因为虽然单个服务器需要处理的数据库规模减小了,但传输的请求和回复总量却随着服务器数量的增加而线性增加。总通信量中占主导地位的是上传数据量。结合前文对两种分片方式的分析,可以看出,采用保持记录大小不变的分片方式,以降低上传通信量,是更符合实际情况的选择。

\paragraph{转发带来的进一步优化}
\begin{table}[]
    \caption{优化后不同分片数量的PIR性能比较}
    \label{tab:sharded-scheme-optimized}
    \centering
    \begin{tabular}{@{}c|cc@{}}
    \toprule
    分片数量 & \multicolumn{2}{c}{通信量(KiB)} \\ \midrule
         & 下载      & 上传       \\ \midrule
    单机   & 0.06          & 1152.02      \\
    8    & 0.48          & 576.02       \\
    32   & 1.92          & 288.02       \\
    128  & 7.68          & 144.02       \\ \bottomrule
    \end{tabular}
\end{table}
在先前的讨论中,我们提到客户端查询可以由服务器复用。在实际应用中,使用的服务器常常集中在数个数据中心内。常见的服务器运营商计费方式\cite{gcp}指出,数据中心内、数据中心间网络传输的费用远低于数据中心与用户间的传输费用。如果我们委托每个数据中心内的一台服务器对请求与回答进行转发,可以进一步降低通信的开销。具体而言,假设所有的服务器平均分布于四个数据中心内,每个数据中心有一台服务器负责向其余服务器转发客户端的查询,忽略数据中心内转发产生的通信,只考虑客户端到数据中心的通信量。在计算量不变的情况下,通信量可以优化至如表\ref{tab:sharded-scheme-optimized}所示。

在这种情况下,客户端上传量随着服务器数量的增加而减少,进一步降低了通信开销。

\subsection{内存使用结果分析}
\begin{table}[]
    \centering
    \caption{不同分片数量的PIR内存使用比较}
    \label{sharded-scheme-memory}
    \begin{tabular}{c|c}
    \hline
    分片数量 & 内存使用(GiB) \\ \hline
    单机   & 32.03     \\
    8    & 8.03      \\
    32   & 2.03      \\
    128  & 0.53      \\ \hline
    \end{tabular}
\end{table}

不同分片数量下单台服务器内存使用情况如表\ref{sharded-scheme-memory}所示。由于需要将整个数据库缓存在内存中,运行PIR所需的内存与数据库大小接近。可以看出,分片后单台服务器只需存储较小的子数据库,单台服务器的内存使用随分片数量增加而减少。这充分说明了这一框架的高扩展性。