\chapter{亚线性复杂度的PIR}

\todo{这一整段从Crust抄过来即可}
\todo{把security proof放什么地方?}

我们首先提出一种在半诚实环境下的两服务器PIR方案的构造。我们的构造基于CK20\cite{EC:CorKog20}框架。我们采用了一种新颖的方法来构建Hint和Query。该方法旨在促进高效的查询,同时确保该构造不会阻碍数据可靠性的集成。

\section{亚线性复杂度PIR基础协议}
本文设计的可验证PIR基于现有的亚线性PIR协议\cite{EC:CorKog20, C:LazPap23}。我们从如图\ref{fig:CK20}所示的两服务器PIR基础协议起步,分析其中的问题,讨论如何修正该基础协议的局限性。该协议是一个离线-在线PIR协议,我们使用$\hintserver$ 与 $\queryserver$ 两个记号来区分协议中涉及到的两台服务器。这些记号是根据两台服务器的功能定义的,其中$\hintserver$负责处理Hint请求,而$\queryserver$处理真正的查询:

\subsection{基础协议}

\noindent \textbf{离线阶段:}
\begin{itemize}
\item \textbf{Setup:} 客户端$\client$ 生成$\hintcount$ 组集合$\setkey_j, j\in[\hintcount]$。每个集合包含$\setsize$个在范围[$\dbsize$]内的随机索引。
\item \textbf{Hint:} $\client$将这些集合转发给$Hint$服务器$\hintserver$。$\hintserver$计算这些集合对应数据库中记录的和,表示为$\hint_j \coloneqq \sum_{k\in \setkey_j} \db_k, j \in [\hintcount]$。$\client$存储这些集合及其对应的记录之和作为校验值。
\end{itemize}

\noindent \textbf{在线阶段:}
\begin{itemize}
\item \textbf{Query:} $\client$首先找出一个包含目标索引$\dbidx$的集合$\setkey_\hintidx$,并从该集合中去掉$\dbidx$,$\queryquery \coloneqq \setkey_\hintidx \setminus \{\dbidx\}$。此外,$\client$生成一个包含索引$\dbidx$的新集合$\setkey'$,并令$\hintquery \coloneqq \setkey' \setminus \{\dbidx\}$。$\client$将集合$\hintquery$发送给$Hint$服务器$\hintserver$,将$\queryquery$发送给$Query$服务器$\queryserver$。
\item \textbf{Answer:} 收到这些集合后,服务器分别为两个集合计算校验值,$\queryanswer\coloneqq \sum_{j\in \queryquery}\db_j$,$\hintanswer\coloneqq \sum_{j\in \hintquery}\db_j$,将这些校验值发送给$\client$。
\item \textbf{Reconstruct:} $\client$ 计算出所需的记录:$\db_\dbidx \coloneqq \hint_\hintidx - \queryanswer$。
\item \textbf{Refresh:} $\client$ 更新$Hint$: $\setkey_\hintidx \coloneqq \setkey'$,$\hint_\hintidx \coloneqq \hintanswer+\db_\dbidx$。
\end{itemize}


\subsection{基础协议的缺陷}
不妨先假设客户端生成的集合足够多,多到在$Query$阶段能以足够大的概率找到包含目标索引的集合。在此前提下,客户端总能得到想要查询的记录$\db_\dbidx$,这使得该协议的正确性得到保证。然而,该协议存在三个关键问题:

\begin{enumerate}
\item \textbf{低效的集合隶属测试:} 搜索包含查询索引$\dbidx$的集合$\setkey_t$的过程计算量很大。客户端需要遍历所有集合,同时枚举集合中的元素以确定某个索引是否隶属于这个集合。尽管可以通过排序和二分查找优化在集合内枚举的过程,但一次搜索仍然需要$O(\sqrt{\dbsize}\log \dbsize)$的在线计算复杂度。
\item \textbf{低下的空间和通信效率:} 以明文形式生成、存储和发送这些集合的效率很低,粗略估计,为了确保正确性,客户端需要大约$\hintcount = O(\lambda\sqrt{\dbsize})$个集合,每个集合的大小为$\setsize = \sqrt{\dbsize}$。相应的存储和离线通信复杂度为$O(\lambda\dbsize)$。
\item \textbf{有缺陷的隐私性:} 集合$\queryquery$和$\hintquery$中均不会包含查询的索引$\dbidx$。从信息论角度来看,“必然不包含$\dbidx$”这一事实泄露了大约$1/(\sqrt{\dbsize}\ln 2)$比特$\dbidx$的信息。
\end{enumerate}

\subsection{“虚设查询”及其引入的新问题}
在现有工作中,研究者已经提出了多种方法来解决上述问题。例如,TreePIR \cite{C:LazPap23} 利用了一种称为弱可穿孔伪随机函数(Weakly Puncturable Pseudorandom Function)的原语,并引入了一种新的亚线性PIR协议设计。 Piano \cite{Piano} 提出了一个类似的算法,称为$PossibleParities$。然而,当尝试向这些协议中引入可验证性时,这些方法遇到了类似的困境,本文总结为“虚设查询”。

\begin{figure}
    \centering
    \includegraphics[width=1\linewidth]{figure/dummy.png}
    \caption{关于“虚设查询”的图示}
    \label{fig:dummy}
\end{figure}

这些“虚设查询”是随机生成的假查询,其目的是与真实查询相混淆。在这些协议中,客户端发送的单个查询会泄露关于查询索引$\dbidx$的信息。为解决这一问题,客户端会在一次查询时发送多个请求给服务器,将真实查询混入“虚设查询”中,从而隐藏索引$\dbidx$。服务器无法区分“虚设查询”与真实查询,所以必须对所有查询作出回应,但客户端会忽略对“虚设查询”的回应。

在半诚实环境下,这些协议的隐私性要求服务器无法区分真实查询和“虚设查询”。然而,在可验证PIR的模型中,“选择失败攻击”会使“虚设查询”失效。在验证过程中,由于客户端没有在离线阶段获得这些“虚设查询”的校验信息,因此无法计算和验证“虚设查询”的答案,只能忽略它们。于是,无论一轮查询中“虚设查询”的答案是否正确,只要真实查询的答案通过了验证,客户端都会接受。利用这一点,服务器可以选择性地篡改答案中的一部分,通过客户端的验证结果推断被篡改的答案是否对应了真实查询。这一漏洞使得恶意服务器有可能区分真实查询和“虚设查询”,进而获得客户端正在查询的索引信息。这导致我们无法在现有亚线性PIR协议中引入可验证性。图\ref{fig:dummy}展示了该问题的一个示例。客户端向服务器发送了3个查询,其中第一个查询是真实查询,另外两个是“选择失败攻击”。由于客户端只拒绝对真实查询的错误答案,接受对“虚设查询”的错误答案,因此服务器可以操纵答案,从客户端的验证结果中推断信息。除了隐私问题外,计算“虚设查询”的答案也增加了在线计算的成本。为了提高效率,应对“选择失败攻击”,我们的协议\textbf{消除了“虚设查询”}。


\section{一种新的亚线性PIR构造方案}

\subsection{对数据库进行划分}

\subsection{优化存储方案}

\subsection{优化查询方案}
\section{可验证的PIR}
\label{sec:verification}
在本节中,我们描述了一种将可验证性嵌入到PIR方案中的技术。我们首先正式提出可验证离线-在线PIR的定义,该概念扩展了先前定义的离线-在线PIR。

\begin{definition}[可验证离线-在线PIR]
    一个可验证离线-在线PIR方案$\Pi$允许客户端从数据库$\db$中检索记录$\db_\dbidx$,而不向$\servercount$个服务器中任意一个泄露索引$\dbidx$,同时确保结果的完整性。该方案包含算法组$\Pi = (Setup, Hint, Query, Answer, Reconstruct, Refresh)$:
    \begin{center}
        离线阶段:
        \begin{itemize}[leftmargin=*]
            \item $Setup(1^\lambda,\dbsize) \rightarrow \query_\hint$:给定数据库大小$\dbsize$和安全参数$\lambda$,生成Hint查询$\query_\hint$。
            \item $Hint(\db, \query_\hint) \rightarrow \hint$:根据数据库$\db$和Hint查询$\query_\hint$ 生成Hint $\hint$。
        \end{itemize}
        在线阶段:
        \begin{itemize}
            \item $Query(\hint, \dbidx) \rightarrow (\query, \clientstate)$:根据Hint~$\hint$和要查询的索引$\dbidx$生成查询$\query$。注意,查询$\query$可能由多个子查询组成。客户端生成并保存一个私有状态$\clientstate$。
            \item $Answer(\db, \query) \rightarrow \answer$:根据查询$\query$生成响应$\answer$。
            \item $Reconstruct(\clientstate, \hint, \answer) \rightarrow \{\db_\dbidx, \bot\}$:根据响应$\answer$,借助Hint~$\hint$和私有状态$\clientstate$重建记录$\db_\dbidx$,或拒绝响应并输出 $\bot$。
            \item $Refresh(\clientstate, \hint, \answer) \rightarrow \hint$:根据响应$\answer$和私有状态$\clientstate$更新Hint~$\hint$。
        \end{itemize}
    \end{center}
    在两服务器($\servercount$=2)时,我们假设Query服务器是恶意的,而Hint服务器是半诚实的。先前的工作 \cite{APIR} 采用了类似的设置,但它们不要求客户端知道哪个服务器是半诚实的。正确的数据库定义为Hint算法 $Hint$ 接受的数据库。在单服务器设置($\servercount=1$)中,正确的数据库由数据所有者提供的摘要定义。该方案应满足以下属性:

    \begin{itemize}
        \item \textbf{正确性}:给定安全参数$\lambda$,任意数据库$\db$和索引$\dbidx$以及任意多项式数量的查询索引序列$\{\dbidx_0,\dbidx_1, \dots\}$,如果服务器和客户端诚实地遵循$\Pi$,则客户端输出$\{\db_{\dbidx_0},\db_{\dbidx_1}, \dots\}$的概率为$1-negl(\lambda)$。
        \item \textbf{完整性}:给定安全参数$\lambda$,任意数据库$\db$和索引$\dbidx$,任意概率多项式时间敌手 $\adversary$作为服务器之一与客户端交互时,客户端重构出错误记录$\widehat{\db_\dbidx} \neq \db_\dbidx$的概率为 $negl(\lambda)$。
        \item \textbf{隐私性(针对选择性失败攻击)}:该属性在定义 \ref{def:privacy-sfa} 中给出。
    \end{itemize}
\end{definition}
\subsection{一种常用于多方安全计算的验证方案}
在多方安全计算协议中,函数 $m = f(x)$ 的计算通常是由一个相关量 $n = g(x) = \alpha f(x)$ 的验证的。其中, $\alpha \in \recordfield$ 代表一个隐藏的系数。验证过程检查是否有 $m\alpha = n$。多方安全计算的性质保障了参与者无法知晓他们计算出的值。诚实的参与者总是能通过此验证,而如果$\recordfield$足够大,不诚实的计算通过验证的概率非常小。

\subsection{验证方案在PIR中的应用}

在我们的协议中,我们采用类似的方法来验证$\queryanswer$的真实性。

在第 \ref{sec:construction} 节中的构造中,客户端最初从服务器获取了一些Hint。我们选择在每个Hint中包含一个额外的校验值,以使客户端能够进行后续的答案验证。升级后的Hint中的校验值包含两个部分:和校验$\sumhint$和随机校验$\randomhint$。对于一个特定集合$S = \{x_j \mid j\in [\setsize]\}$,一组随机数 $SR\coloneqq \{r_j \mid j\in [\setsize]\}$,$\sumhint, \randomhint$ 定义为如下:

$$
    \begin{array}{l}
        \sumhint \coloneqq  \sum_{j \in [\setsize]} \db_{x_j}, \\
        \randomhint \coloneqq  \sum_{j \in [\setsize]} r_j\cdot \db_{x_j}
    \end{array}
$$

查询被分为两个不同的部分:和查询$\sumquery$和随机查询$\randomquery$。假设$x_\dbidx$是查询的索引。$\sumquery$和$\randomquery$的定义如下:

$$
    \begin{array}{l}
        \sumquery \coloneqq  S \setminus \{x_\dbidx\}, \\
        \randomquery \coloneqq  SR \setminus \{r_\dbidx\}
    \end{array}
$$

在接收到这两个查询后,Query服务器计算答案$\sumanswer$和$\randomanswer$:

$$
    \begin{array}{l}
        \sumanswer \coloneqq  \sum_{j \in [s-1]} \db_{\sumquery_j}, \\
        \randomanswer \coloneqq  \sum_{j \in [s-1]} \randomquery_j\cdot \db_{\sumquery_j}
    \end{array}
$$

Query服务器将答案发送给客户端,客户端重构记录$\db_{x_\dbidx} \coloneqq \sumhint-\sumanswer$,通过以下条件来验证答案:

$$
    \randomhint-\randomanswer = r_\dbidx \cdot (\sumhint-\sumanswer)
$$

如果Query服务器如实作答,则等式的左边等于$r_{\dbidx} \cdot \db_{x_\dbidx}$。由于$\sumhint - \sumanswer = \db_{x_\dbidx}$,等式应成立。如果验证未通过,客户端输出 $\bot$。

在我们的框架中,我们进一步采用第 \ref{sec:construction} 节中讨论的利用PRF压缩集合的策略来提高空间效率。集合 $S$ 可以使用密钥$\setkey$通过一个PRF展开。集合$SR$ 可以通过另一个PRF展开:$fr_{\randomset}: [\sqrt{\dbsize}] \to \recordfield$。因此,集合 SR 也可以用一个短密钥$\randomset$来表示。

$$
    \randomset \to SR = \{fr_{\randomset}(j) \mid j\in[\sqrt{\dbsize}]\}
$$

稍稍滥用符号,我们使用 $Eval(\randomset, j)$ 作为 $fr_{\randomset}(j)$ 的简写。

与索引集合一致,这一随机数组成的集合$SR$中被删去元素的位置必然在包含查询索引$\dbidx$的块中。在半诚实方案中,一个随机Crumb($\crumb$)替换了被删去的索引,从而隐藏了$\dbidx$所在的块。同样,在验证方案中,一个随机元素$r \leftarrow \recordfield$可以用于隐藏$SR$中被删去删除的元素。具体来说,令$\blockidx \coloneqq \lfloor \dbidx /\sqrt{\dbsize} \rfloor$为包含查询索引$\dbidx$的块,客户端从$\recordfield$中采样$r \leftarrow \recordfield$,并将$SR[\blockidx]$替换为$r$。在客户端接收到响应$\randomanswer$后,它可以移除Crumb值的贡献($\crumbvalue$)并更新$\randomanswer$,即:
$$\randomanswer \coloneqq \randomanswer - r\cdot \crumbvalue$$
随后,客户端通过检查
$$\randomhint-\randomanswer = Eval(\randomset,\blockidx)\cdot \db_{\dbidx}$$
来执行验证。如果验证失败,客户端输出$\bot$。

我们在下图中展示了我们的双服务器可验证PIR方案,其中验证过程以蓝色突出显示。

% \begin{figure*}
    \begin{mdframed}
        \centering
        \textbf{可验证两服务器协议}

        \raggedright
        \paragraph{符号约定:} 协议包含一个客户端 $\client$,一台Query服务器 $\queryserver$ 与一台Hint服务器 $\hintserver$。单个Hint由元组:$\hint=(\setkey,\sumhint,\bluetext{\randomset,\randomhint})$构成。单个Crumb包含了一个偏移量和一条记录值 $\crumb=(\crumboffset,\crumbvalue)$。$fk_{key}:\{0,1\}^\lambda \to \{0,1\}^\lambda$ 是一个能将单一PRF密钥映射为多个PRF密钥的PRF函数, $f_{\setkey}: [\sqrt{\dbsize}]\to [\sqrt{\dbsize}]$ 是一个将块序号转化为偏移量的PRF, \bluetext{$fr_{\randomset}: [\sqrt{\dbsize}]\to \recordfield$ 是一个将块序号转化为 $\recordfield$上的随机元素的PRF。} $Eval(\setkey,\dbidx) \coloneqq f_{\setkey}(\dbidx)\oplus \shift + \dbidx\cdot\sqrt{\dbsize}$ 与 $Expand$ 密钥 $\setkey$ 标识计算密钥对应的集合内元素 $\{Eval(sk,j) \mid j\in[\sqrt{\dbsize}]\}$。 \bluetext{$Eval(\randomset,\dbidx) \coloneqq fr_{\randomset}(\dbidx)$ 与 $Expand$ 密钥 $\randomset$ 表示计算集合元素 $\{Eval(sr,j) \mid j\in[\sqrt{\dbsize}]\}$。} 假设 $\dbidx$ 是需要查询的索引。 在离线阶段,总共生成 $\hintcount$ 个 Hint。

        \paragraph{离线阶段:}
        \begin{itemize}
            \item \textbf{Setup:} $\client$ 生成一个PRF密钥 $mk\in\{0,1\}^\lambda$,将本地的Hint存储初始化为 $\hint_j\coloneqq(fk_{mk}(j),0,\bluetext{fk_{mk}(j+\hintcount),0}), j\in [\hintcount]$,Crumb存储初始化为 $\crumb_j\coloneqq (\bot,\bot), j\in [\sqrt{\dbsize}]$。 $\client$ 将 $mk$ 发送给 $\hintserver$.
            \item \textbf{Hint:}
                  \begin{itemize}
                      \item $\hintserver$ 将 $mk$ 展开为PRF密钥 $\setkey_j=fk_{mk}(j), j\in[\hintcount]$ \bluetext{与 $\randomset_j=fk_{mk}(j+\hintcount), j\in[\hintcount]$}。它将这些密钥进一步$Expand$为集合 $S_j, j\in[\hintcount]$ \bluetext{与 $SR_j, j\in[\hintcount]$}.
                      \item $\hintserver$ 为每个集合计算校验值 $\sumhint_j\coloneqq \sum_{k\in [\sqrt{\dbsize}]}\db_{S_{j}[k]}, j\in[\hintcount]$ \bluetext{与 $\randomhint_j\coloneqq \sum_{k\in [\sqrt{\dbsize}]}\db_{S_{j}[k]}\cdot SR_{j}[k], j\in[\hintcount]$}。$\hintserver$将这些校验值发送给 $\client$.
                      \item 在第 $j$ 个 $\sqrt{\dbsize}$ 大小的块 ($j\in[\sqrt{N}]$) 内, $\hintserver$ 随机选取一个偏移量 $\crumboffset_j\leftarrow [\sqrt{N}]$ 及其对应的记录值 $\crumbvalue_j$ 作为Crumb。它将这些Crumb $\crumb_j \coloneqq  (\crumboffset_j, \crumbvalue_j), j\in [\sqrt{\dbsize}]$ 发送给 $\client$。 $\client$ 接受并储存这些Crumb。
                  \end{itemize}
        \end{itemize}
        \paragraph{在线阶段:}
        \begin{itemize}
            \item \textbf{Query ($\queryserver$):}
                  \begin{itemize}
                      \item 记 $\dbidx$ 所在的块为 $\blockidx\coloneqq \lfloor \dbidx/\sqrt{\dbsize}\rfloor$。 $\client$ 在存储中找到一个Hint $\hint_\hintidx = (\setkey_\hintidx,\sumhint_\hintidx,\bluetext{\randomset_\hintidx,\randomhint_\hintidx})$,满足条件 $f_{\setkey_\hintidx}(\blockidx) = \dbidx-\sqrt{\dbsize}\cdot \blockidx$。如果没有这样的Hint,$\client$ 终止查询并输出 $\bot$。
                      \item $\client$ 将 $\setkey_\hintidx$ $Expand$ 为集合 $S$。$\client$ 找到 $\blockidx$ 的 Crumb $\crumb_\blockidx=(\crumboffset_\blockidx, \crumbvalue_\blockidx)$,将 $S[\blockidx]$ 替换为Crumb偏移量 $\crumboffset_\blockidx$。 \bluetext{$\client$ 将 $\randomset_\hintidx$ $Expand$ 为集合 $SR$,并将 $SR[\blockidx]$ 替换为一随机元素 $r\leftarrow \recordfield$。} $\client$ 将 $S$ \bluetext{与 $SR$} 发送给 $\queryserver$。
                  \end{itemize}
            \item \textbf{Answer ($\queryserver$):} $\queryserver$ 计算校验值 $\sumanswer\coloneqq \sum_{k\in [\sqrt{\dbsize}]}\db_{S[k]}$ \bluetext{与 $\randomanswer\coloneqq \sum_{k\in [\sqrt{\dbsize}]}\db_{S[k]}\cdot SR[k]$}。$\queryserver$ 将 $\sumanswer$ \bluetext{与 $\randomanswer$} 发送给 $\client$.
            \item \textbf{Reconstruct:} $\client$ 重构出记录 $\db_\dbidx \coloneqq  \sumhint_\hintidx-(\sumanswer-\crumbvalue_\blockidx)$。 \bluetext{$\client$ 验证是否有 $\randomhint_\hintidx-(\randomanswer-r\cdot \crumbvalue_\blockidx) = Eval(\randomset_\hintidx, \blockidx)\cdot \db_\dbidx$。如果验证失败,$\client$ 输出 $\bot$。否则, } $\client$ 输出 $\db_\dbidx$。
            \item \textbf{Query ($\hintserver$):}
                  \begin{itemize}
                      \item $\client$ 生成PRF密钥 $\setkey'$ 使得 $f_{\setkey'}(\blockidx) = \dbidx-\sqrt{\dbsize}\cdot \blockidx$。 \bluetext{$\client$ 生成PRF密钥 $\randomset'$}.
                      \item $\client$ 将 $\setkey'$ $Expand$ 为集合 $S'$, \bluetext{将 $\randomset'$ $Expand$ 为集合 $SR'$}。$\client$ 将 $S'[\blockidx]$ 替换为随机偏移 $\crumboffset' \leftarrow [\sqrt{\dbsize}]$. $\client$ 将 $S'$ 发送给 $\hintserver$.
                  \end{itemize}
            \item \textbf{Answer ($\hintserver$):} $\hintserver$ 直接将 $\sqrt{\dbsize}$ 条记录 $\db_{S'[j]}, j\in [\sqrt{\dbsize}]$ 发送给 $\client$。
            \item \textbf{Refresh:} $\client$ 将Crumb更新为 $\crumb_\blockidx \coloneqq  (\crumboffset', \db_{S'[\blockidx]})$,将Hint更新为 $\hint_\hintidx \coloneqq  (\setkey',\db_\dbidx + \sum_{k\in [\sqrt{\dbsize}],k\neq \blockidx}\db_{S'[k]}, \bluetext{\randomset', SR[l]\cdot \db_\dbidx + \sum_{k\in [\sqrt{\dbsize}],k\neq \blockidx}SR[k]\cdot \db_{S'[k]}})$
        \end{itemize}
    \end{mdframed}
    \label{fig:two-server-verify}
% \end{figure*}

% 我们提出了以下关于安全性和效率的定理:

% \begin{theorem}
%     \label{thm:1}
%     图 \ref{fig:two-server-verify} 中描述的方案是一个双服务器可验证的离线-在线PIR方案。设 $\alpha(\lambda)$ 为任意超常数函数,即$\alpha(\lambda)$ = $\omega(1)$。假设数据库索引大小为$O(1)$,PRF密钥大小为$O_\lambda(1)$。在一个包含$\dbsize$条记录、每条大小为$\recordsize$的数据库上,存储$\hintcount = \magictotal$个Hint,该方案的效率如下:
    
%     \noindent 在离线阶段:
%     \begin{itemize}[itemsep=0em]
%         \item 通信量为$O_\lambda(1)$上传,$2\recordsize\hintcount + \recordsize\sqrt{\dbsize} + O(\sqrt{\dbsize})$下载。
%         \item 客户端计算复杂度为$O_\lambda(\hintcount)$,服务器计算复杂度为$O_\lambda(\hintcount) + O(\recordsize\sqrt{\dbsize}\hintcount)$。
%     \end{itemize}
%     在在线阶段:
%     \begin{itemize}[itemsep=0em]
%         \item 向Query服务器通信量为:$\recordsize\sqrt{\dbsize} + O(\sqrt{\dbsize})$上传,$2\recordsize$下载。
%         \item 向Hint服务器通信量为:$O(\sqrt{\dbsize})$上传,$\recordsize\sqrt{\dbsize}$下载。
%         \item 客户端计算复杂度为$O_\lambda(\sqrt{\dbsize}) + O(\recordsize\sqrt{\dbsize})$,服务器计算复杂度为$O(\recordsize\sqrt{\dbsize})$。
%         \item 支持多项式数量的查询。
%     \end{itemize}
% \end{theorem}

% \paragraph{恶意的Hint服务器}
% 当Hint服务器是恶意的时,该协议是不安全的。在附录 \ref{appendix:hint-server} 中提供了详细的解释和示例。
\section{单服务器模型}
这一节中,我们将前文方案适配为单服务器可验证的PIR。在这种场景下,我们假设数据所有者将数据库外包给单个服务器,并生成一个数据库摘要。客户端可以使用此摘要对查询结果进行验证。为了实现这一适配,有两个主要问题需要解决:
\begin{enumerate}
    \item \textbf{更新Hint:} 为了让客户端能够执行多个查询,方案必须支持Hint更新。然而,原本的方案中,Hint更新不能由Query服务器执行。
    \item \textbf{获取hints.} 在双服务器设置中,Hint服务器负责向客户端提供Hint。在单服务器设定中,我们不能让Query服务器来完成这一任务。如果在线查询的服务器同时也知道离线生成的Hint,它可以通过比对Hint和查询来推断客户端正在查询的索引。
\end{enumerate}

此外,还需要考虑客户端如何确保托管给不可信服务器的数据库的正确性。我们将在以下部分中解决上述问题。

\subsection{一种通用的协议编译器}

\paragraph{使用备用更新Hint}
为了在在线阶段更新Hint,我们采用了备用Hint的做法 \cite{EC:CorHenKog22}。在这种方法中,客户端在离线阶段获取额外的Hint作为“备用”。这些备用Hint用于更新在线阶段消耗的Hint。

在这种方法中,Hint分为两类:主Hint和备用Hint。客户端获取$\hintcount_1$个主Hint。这些主Hint与双服务器协议中的Hint是相同的。除此之外,客户端还为每个块获取$\hintcount_2$个备用Hint。块$\blockidx$的备用Hint在第$\blockidx$块中留空。换句话说,块$\blockidx$的备用Hint集合中不包含块$\blockidx$中的索引。客户端按照双服务器方案使用主Hint向服务器发出查询。一旦完成对块$\blockidx$中某个索引的查询,客户端会获取该块的备用Hint,将查询的索引纳入备用Hint并更新校验信息,以此将备用Hint转换为主Hint。转换需要在Hint中加入一个额外的索引。在实际操作中,客户端使用元组$(\setkey, x)$表示一个Hint。客户端首先将$\setkey$展开为集合$S$。如果$x\neq \bot$,客户端将$S[\lfloor x/\sqrt{\dbsize}\rfloor]$替换为$x$。

Crumb也遵循类似的模式。客户端在离线阶段为每个块保留$\hintcount_2$个备用Crumb,并在查询后从备用中获取一个对应块的新Crumb。

可以证明,若为每个块分配了$3\magicnumber$个备用Hint和Crumb,该方案能够让客户端在一个离线阶段后以极高的概率成功执行最多$\magictotal$次查询。然而,在此之后,必须重新启动离线阶段以补充这些资源。

\paragraph{通过流式传输数据库获取Hint}
从现有工作的角度来看,在单服务器环境下中,离线获取Hint可以通过两种方式实现:(i)  使用同态加密来检索Hint\cite{EC:CorHenKog22},(ii) 将整个数据库流式传输到客户端\cite{CCS:PatPerYeo18, Piano}。

我们采用了后者的方法,因为在实际操作中它更为高效。Hint生成可以以块状流式传输的方式进行。客户端一次从服务器请求$\sqrt{\dbsize}$个块中的一个,并更新所有本地Hint。具体来说,当处理第$j$个块时,对于具有校验值$\hint_\hintidx$和PRF密钥$\setkey_\hintidx$的Hint,客户端通过$\hint_\hintidx \coloneqq \hint_\hintidx+\db_{Eval(\setkey_\hintidx,j)}$更新$\hint_\hintidx$。如果Hint不应该包含该块中的任何索引(它是一个此块的备用Hint),则跳过该Hint。所有Hint更新完成后,该块将被服务器获取的新块替换。这样可以确保客户端的存储保持在低于数据库大小的规模。

\paragraph{引入验证}
将可验证性整合到我们的方案中并不不需要额外的工作。唯一的不同点是,利用流式传输的方式获取Hint时,除了附加的校验值外,我们还要求受信任的数据库所有者提供数据库的摘要,这与前任的工作 \cite{APIR23} 中的假设一致。该摘要可以通过签名发布在区块链等可信平台上。在执行在线查询之前,客户端需要根据此摘要验证流式传输的数据库,以确保数据库内容的真实性和完整性。

到目前为止,我们已经解决了将双服务器可验证PIR方案转换为单服务器环境所面临的主要问题。在完整给出我们的单服务器构造之前,有必要强调一些这种构造的重要性质:
\begin{itemize}
    \item \textbf{支持$\querycount = \Omega(\sqrt{\dbsize})$次查询就能支持多项式数量的查询:} 在进行$\querycount$次查询后,协议双方可以重新执行离线阶段。只要方案能在$O_\lambda(\recordsize\dbsize)$时间内完成离线阶段(这在亚线性PIR方案中很容易做到),客户端就可以将这一复杂度均摊到$\querycount$次查询上。这一离线-在线的循环过程就可以支持多项式数量的查询。

    \item \textbf{查询不会重复:} 不失普遍性地,我们可以假设在$\querycount$ 次查询中没有重复查询。我们可以令客户端使用额外的$\Theta(\recordsize \querycount)$存储空间缓存最近的$\querycount$个查询结果。如果客户端需要提出一个重复查询,它可以随机查询一个未查询过的数据库记录,并从缓存结果中检索需要的信息。

    \item \textbf{查询的分布是均匀的:} 不失普遍性地,我们假设查询的索引在数据库中是均匀的。数据库所有者可以根据伪随机置换(PRP)密钥对数据库条目进行洗牌。我们还可以假设查询索引是在不知道这一置换密钥的情况下选择的。由于我们先前假设了查询中没有重复,每个查询索引将被映射到洗牌后数据库中的随机索引,并落入一个随机块。
\end{itemize}

总结上述内容,我们在下图中展示了我们的单服务器可验证PIR方案,其中验证过程以蓝色标记突出显示。

% \begin{figure*}
    \begin{mdframed}
    \centering
    \textbf{可验证单服务器协议}
        \raggedright
        \paragraph{符号约定:} 协议包含一个客户端 $\client$ 与一台Query服务器 $\queryserver$。单个Hint由元组:$\hint=(\setkey,x,\sumhint,\bluetext{\randomset,\randomhint})$构成。单个Crumb包含了一个偏移量和一条记录值 $\crumb=(\crumbvalue, \crumboffset)$。$f_{\setkey}: [\sqrt{\dbsize}]\to [\sqrt{\dbsize}]$ 是一个将块序号转化为偏移量的PRF,\bluetext{$fr_{\randomset}: [\sqrt{\dbsize}]\to \recordfield$ 是一个将块序号转化为 $\recordfield$上的随机元素的PRF。} $Eval(\setkey,\dbidx) \coloneqq f_{\setkey}(\dbidx)\oplus \shift + \dbidx\cdot\sqrt{\dbsize}$ 与 $Expand$ 密钥 $\setkey$ 标识计算密钥对应的集合内元素 $\{Eval(sk,j) \mid j\in[\sqrt{\dbsize}]\}$。 \bluetext{$Eval(\randomset,\dbidx) \coloneqq fr_{\randomset}(\dbidx)$ 与 $Expand$ 密钥 $\randomset$ 表示计算集合元素 $\{Eval(sr,j) \mid j\in[\sqrt{\dbsize}]\}$。} 假设 $\dbidx$ 是需要查询的索引。\bluetext{可信的数据库所有者 $\owner$ 计算数据库 $\db$ 的摘要 $\digest$ 并且将其发送给 $\client$。} 在离线阶段,生成 $\hintcount_1$ 个主Hint与每块 $\hintcount_2$ 个备用Hint。

        \paragraph{离线阶段:}
        \begin{itemize}
            \item \textbf{Setup:} $\client$ 生成主PRF密钥 $\setkey_j, j\in[\hintcount]$,备用PRF密钥 $\setkey_{k,j}, j\in[\lambda], k\in[\sqrt{\dbsize}]$ \bluetext{与 $\randomset_j, j\in[\hintcount]$ 及其备用密钥 $\randomset_{k,j}, j\in[\lambda], k\in[\sqrt{\dbsize}]$},将本地的主Hint存储初始化为 $\hint_j\coloneqq(\setkey_j,\bot,0,\bluetext{\randomset_j,0}), j\in[\hintcount]$,备用Hint初始化为 $\hint_{k,j}\coloneqq(\setkey_{k,j},\bot, 0,\bluetext{\randomset_{k,j},0}), j\in[\lambda], k\in[\sqrt{\dbsize}]$,Crumb存储初始化为$\crumb_{k,j} \coloneqq (\bot,\bot),j\in[\lambda], k\in[\sqrt{\dbsize}]$。
            \item \textbf{Hint:} 服务器 $\queryserver$ 将数据库 $\db$ 传输给 $\client$。\bluetext{$\client$ 初始化摘要 $\digest'$.} 当传输块 $l$ 时, $\client$ 按如下方法更新Hint:
                  \begin{itemize}
                      \item 更新主Hint: 对于所有 $j\in[\hintcount]$,$\sumhint_j \coloneqq \sumhint_j+\db_{Eval(\setkey_j, l)}$ \bluetext{以及 $\randomhint_j \coloneqq  \randomhint_j+Eval(\randomset_j, l)\cdot \db_{Eval(\setkey_j, l)}$}。
                      \item 更新不属于此块的备用Hint:对于所有 $j\in[\lambda], k\in[\sqrt{\dbsize}], k\neq l$ 的 $\sumhint_{k,j}$ \bluetext{和 $\randomhint_{k,j}$},$\sumhint_{k,j} \coloneqq  \sumhint_{k,j}+\db_{Eval(\setkey_{k,j}, l)} $ \bluetext{以及 $\randomhint_{k,j} \coloneqq  \randomhint_{k,j}+Eval(\randomset_{k,j}, l)\cdot \db_{Eval(\setkey_{k,j}, l)}$}。
                      \item 将块$l$的 Crumb $\crumb_{l,j}$ 更新为随机选择的记录值以及对应的偏移 $(\crumbvalue_{l,j}, \crumboffset_{l,j}), j\in[\lambda] $。
                      \item \bluetext{$\client$ 用块$l$的内容更新 $\digest'$。}
                      \item 完成之后,$\client$ 从储存中删除此块。
                    \end{itemize}
                \item \bluetext{$\client$ 检查是否有 $\digest = \digest'$。如果不成立, $\client$ 终止协议并输出 $\bot$。}
        \end{itemize}
        \paragraph{在线阶段:}
        \begin{itemize}
            \item \textbf{Query:}
                  \begin{itemize}
                      \item 记 $\dbidx$ 所在块为 $\blockidx\coloneqq \lfloor \dbidx/\sqrt{\dbsize}\rfloor$。 \redtext{\client 记录每一块被查询了多少次。如果块 $\blockidx$ 已经被查询了超过 $\hintcount_2$ 次, \client 随机选择一个被查询少于$\hintcount_2$ 的块中索引 $\dbidx'$ 作为查询对象,并重新运行在线阶段。} $\client$ 在存储中找到一个Hint $\hint_\hintidx = (\setkey_\hintidx,x_\hintidx,\sumhint_\hintidx,\bluetext{\randomset_\hintidx,\randomhint_\hintidx})$。这个Hint需要包含 $\dbidx$ ( $x_\hintidx=\dbidx$ 或是 $Eval(\setkey_\hintidx, \blockidx) + \sqrt{\dbsize}\cdot \blockidx = \dbidx $ 且有 $x_\hintidx = \bot \vee \lfloor x_\hintidx/\sqrt{\dbsize}\rfloor\neq \blockidx$). 如果没有这样的Hint,$\client$ 终止查询并输出 $\bot$。
                      \item $\client$ 将 $\setkey_\hintidx$ $Expand$ 为集合 $S$。如果$x_\hintidx\neq \bot$,将$S[\lfloor x_\hintidx/\sqrt{\dbsize}\rfloor]$ 替换为 $x_\hintidx$。$\client$找到一个块$\blockidx$中的Crumb $\crumb_{\blockidx, \crumbidx}=(\crumbvalue_{\blockidx, \crumbidx},\crumboffset_{\blockidx, \crumbidx})$ 并将 $S[\blockidx]$ 替换为  $\crumboffset_{\blockidx, \crumbidx}$。 \bluetext{$\client$ 将 $\randomset_\hintidx$ $Expand$ 为集合 $SR$,将 $SR[\blockidx]$ 替换为随机数 $r\leftarrow \recordfield$。} $\client$ 将 $S$ \bluetext{与 $SR$} 发送给 $\queryserver$。
                  \end{itemize}
            \item \textbf{Answer:} $\queryserver$ 计算校验值 $\sumanswer\coloneqq \sum_{k\in [\sqrt{\dbsize}]}\db_{S[k]}$ \bluetext{以及 $\randomanswer\coloneqq \sum_{k\in [\sqrt{\dbsize}]}\db_{S[k]}\cdot SR[k]$}。$\queryserver$ 将 $\sumanswer$ \bluetext{与 $\randomanswer$} 发送给 $\client$。
            \item \textbf{Reconstruct:} $\client$ 重构出记录 $\db_\dbidx \coloneqq  \sumhint_\hintidx-(\sumanswer-\crumbvalue_{\blockidx, \crumbidx})$。 \bluetext{$\client$ 验证是否有  $\randomhint_\hintidx-(\randomanswer-r\cdot \crumbvalue_{\blockidx, \crumbidx}) = Eval(\randomset_\hintidx, \blockidx)\cdot \db_\dbidx$。如果验证失败,$\client$ 输出 $\bot$。否则, } $\client$ 输出 $\db_\dbidx$。
            \item \textbf{Refresh:}
                  \begin{itemize}
                      \item $\client$ 找到一个块$\blockidx$中的未使用备用Hint $\hint_{\blockidx,j}=(\setkey_{\blockidx,j},\bot,\sumhint_{\blockidx,j},\bluetext{\randomset_{\blockidx,j},\randomhint_{\blockidx,j}})$。如果块中已经没有备用Hint了,$\client$ 直接结束$Refresh$算法。
                      \item $\client$ 更新Hint:$\hint_\hintidx \coloneqq (\setkey_{\blockidx,j}, \dbidx, \sumhint_{\blockidx,j} + \db_\dbidx, \bluetext{\randomset_{\blockidx,j},\randomhint_{\blockidx,j}+Eval(\randomset_{\blockidx,j},\blockidx)\cdot \db_\dbidx})$.
                  \end{itemize}
        \end{itemize}
    \end{mdframed}
    % \caption{单服务器PIR协议。蓝色部分是验证过程。红色与蓝色部分都可以在半诚实模型中移除。}
    \label{fig:single-server}
% \end{figure*}

\noindent \paragraph{效率}
设$\hintcount = \hintcount_1 + \sqrt{\dbsize}\hintcount_2$。在离线阶段将数据库流式传输到客户端需要$\recordsize\dbsize$的通信量,并且计算量为$O_\lambda(\sqrt{\dbsize}\hintcount) + O(\recordsize\sqrt{\dbsize}\hintcount)$。这与当前最先进的单服务器亚线性PIR方案 \cite{Piano} 一致。在在线阶段,成本与双服务器方案的查询服务器部分类似。然而,客户端可以在$O(\recordsize)$时间内更新Hint,因为它不需要从$\sqrt{\dbsize}$条记录中重新计算Hint。与双服务器方案不同的是,离线阶段必须在$\querycount$次查询后重新运行。

\paragraph{选择失败攻击引入的问题}
为了应对选择性失败攻击,我们希望采用更强的正确性定义,尤其是允许敌手选择被查询的索引。然而,单服务器方案无法支持这种情况,因为敌手可能会使查询在不同的块之间不平衡,从而导致正确性失效,这会阻碍隐私保护。为解决这一问题,我们在单服务器协议中引入了一些额外措施(以红色标记),即当客户端无法查询给定的索引时,会随机查询一个索引,从而隐藏失败。然而,客户端不会隐藏由于服务器不当行为引起的查询失败。

我们认为:(i) 这并不与正确性的定义相矛盾,因为在定义中索引是在进行置换之前选择的;(ii) 该解决方案与本文的目标一致,因为报告此类失败对客户端或服务提供商没有任何好处;(iii) 该解决方案不会影响实用性,因为协议的诚实执行不受影响,并且敌手总是可以通过拒绝响应来阻止客户端检索所需的索引。

\subsection{离线处理方案的可行性}
单服务器版本的方案以及一些文献\cite{Piano, EC:CorHenKog22}中提出的构造引入了大量的离线通信。另一种方案\cite{EC:CorHenKog22}使用同态加密计算Hint。该论文提出了一个基于LHE(线性同态加密)的具体方案以及一个基于FHE(全同态加密)的理论方案。我们尚不清楚如何在实践中使用FHE实现该方案。LHE方案需要$\softO(\recordsize\sqrt{\dbsize \querycount} + \dbsize)$的离线通信。当将本文的参数$\querycount=\magictotal$代入时,结果为$\softO(\recordsize\dbsize^{3/4} + \dbsize)$,且隐藏了较大的对数和常数因子。此外,该方案需要$\softO(\dbsize^{3/4}+\recordsize\sqrt{\dbsize})$的在线通信和计算。与本文$\recordsize\dbsize$离线通信相比,当数据库的记录数超过 $\lambda^6 \approx 2^{42}$ 时,可能会有一定的优势。此时若每条记录的大小为$\recordsize = \Theta(\lambda)$,数据库的大小至少为 64 TiB。并且,该方案在线查询的效率较低。

\subsection{针对在线查询的优化}
\label{sec:optimized-model}
使用备用Hint进行更新的想法也可以应用于双服务器方案。具体来说,客户端从Hint服务器获取额外的备用Hint和Crumb,从而可以和单服务器场景一样,在线查询时仅与一台服务器交互。然而,在离线阶段客户端与服务器并不需要流式传输整个数据库。只需要让客户端将PRF密钥发送给Hint服务器,采用与双服务器协议类似的离线Hint算法即可。如此,客户端可以避免与半诚实服务器进行在线交互,同时最大限度地减少离线通信。这一方案为服务提供商提供了一个实用框架。数据库所有者可以利用空闲带宽提供离线Hint获取服务,并将实时查询委托给不可信的服务器。
\section{协议分析}

\label{sec:analysis}


\subsection{安全分析}
% {这里可以写好几页,也可以写一句话,看情况}
本节将聚焦于图\ref{fig:two-server-verify}中描述的双服务器协议,提供对该协议安全性证明的直观解释。单服务器协议类似于仅包含查询服务器的双服务器协议的在线阶段。详细的安全性证明可在附录\ref{appendix:security}中找到。

\paragraph{正确性}
只有当客户端无法找到包含查询索引的集合时,它才无法获取查询的记录。由于客户端持有$\hintcount = \magictotal$个Hint,这种情况发生的概率可以被忽略。由于来自服务器的诚实答案始终被接受,因此正确性的缺陷是可忽略的。

\paragraph{完整性}
设 $\blockidx = \lfloor \dbidx / \sqrt{\dbsize} \rfloor$表示包含查询索引的块,$r = Eval(\randomset, \blockidx)$是从查询中移除的随机元素。在$Reconstruct$算法中,客户端检查是否满足$\randomhint-\randomanswer = r \cdot (\sumhint-\sumanswer)$。恶意查询服务器提供的错误答案可以表示为$(\sumanswer+\Delta^+, \randomanswer+\Delta^\times)$,其中$(\Delta^+, \Delta^\times) \neq (0, 0)$。此检查可以简化为 $r\Delta^+ - \Delta^\times = 0$。在不知道$r$的情况下,此方程成立的概率为$2^{-\recordsize}$。

\paragraph{隐私性}
在离线阶段,生成的Hint和集合与任何特定的索引都无关。在在线阶段,客户端针对任何服务器的每个查询都会提供$[\sqrt{\dbsize}]$上的$\sqrt{\dbsize}$个随机数,以及$\recordfield$中的$\sqrt{\dbsize}$个随机元素,与查询的索引无关。

本文的协议能够抵抗“选择失败攻击”。值得注意的是,验证结果不会向Query服务器透露任何信息。如果服务器如实回答,由协议的正确性保证,客户端将通过验证并接受答案。如果服务器提供错误答案,根据完整性分析,客户端只有$2^{-\recordsize}$的概率接受该答案。因此,敌手的优势不会超过$2^{-\recordsize}$。因为 $\recordsize \ge \lambda$,这可以被视作一个关于 $\lambda$ 的可忽略函数。

\subsection{性能分析}
\paragraph{双服务器协议}
在离线阶段,客户端向Hint服务器发送一个大小为$O_\lambda(1)$的PRF密钥。Hint服务器将该密钥扩展为$\hintcount$个Hint密钥,并计算这些Hint。随后,服务器向客户端发送$\hintcount$个校验值和$\sqrt{\dbsize}$个Crumb。校验值的大小为$2\recordsize\hintcount$,$\sqrt{\dbsize}$个Crumb的大小为$\recordsize\sqrt{\dbsize} + O(\sqrt{\dbsize})$。因此,离线通信总量为$2\recordsize\hintcount + \recordsize\sqrt{\dbsize} + O(\sqrt{\dbsize})$,离线计算开销为$O_\lambda(\hintcount) + O(\recordsize\sqrt{\dbsize}\hintcount)$。

在在线阶段,客户端搜索包含查询索引$\dbidx$的Hint,预期计算开销为$O_\lambda(\sqrt{\dbsize})$。随后的Hint处理,包括展开集合、用Crumb替换索引以及生成新集合,同样需要$O_\lambda(\sqrt{\dbsize})$的计算量。客户端向Query服务器发送的查询大小为$\recordsize\sqrt{\dbsize} + O(\sqrt{\dbsize})$,向Hint服务器发送的查询大小为$O(\sqrt{\dbsize})$。每个服务器在$O(\recordsize\sqrt{\dbsize})$时间内计算校验值。Hint服务器回复$\sqrt{\dbsize}$条记录,Query服务器回复2个校验值。然后,客户端在$O(\recordsize\sqrt{\dbsize})$时间内验证答案并刷新Hint。

\paragraph{单服务器协议}
设$\hintcount = \hintcount_1 + \sqrt{\dbsize}\hintcount_2$。在离线阶段,将数据库流式传输到客户端需要$\recordsize\dbsize$的通信量,并且计算量为$O_\lambda(\sqrt{\dbsize}\hintcount) + O(\recordsize\sqrt{\dbsize}\hintcount)$。这与当前最先进的单服务器亚线性PIR协议\cite{Piano}一致。在在线阶段,成本与双服务器协议的Query服务器部分类似。因为不需要从$\sqrt{\dbsize}$条记录中重新计算Hint,客户端可以在$O(\recordsize)$时间更新Hint。单服务器协议与双服务器协议不同,必须在$\querycount$次查询后重新运行离线阶段。
\section{可更新的PIR}
\label{sec:handling-updates}
本节讨论如何更新PIR的数据库。并给出一种可行的在亚线性PIR中更新数据库的方法。

\subsection{PIR更新的复杂性}

\todo{对比一下其他协议的更新,聊一下OOPIR和Checklist,非通用与通用的方法}
在PIR中,更新数据库是一个出乎意料的复杂问题。与传统的数据库更新不同,PIR的更新有两个难点:
\begin{enumerate}
    \item 一部分PIR协议设计需要服务器预处理数据库,更新数据库时可能需要重新进行预处理,这会导致更新的复杂度增加。
    \item 一部分PIR协议需要客户端保存额外的数据,这部分额外的数据可能需要重新计算。
\end{enumerate}

本文中涉及到的亚线性PIR协议不存在第一类问题,但是需要处理第二类问题。目前,在亚线性PIR协议中,主要有两类做法:
\begin{enumerate}
    \item 瀑布式更新\cite{Checklist}:将数据库划分为一系列大小不等的子数据库,处理数据库变更时,只修改受影响的子数据库。
    \item 增量式更新\cite{USENIX:MZRA22}:修改Hint的构造,支持新增元素。
\end{enumerate}

本文使用的Hint构造,以及对数据库进行划分的需求不能有效支持增量式更新的做法,因此,本文采用了一种类似于瀑布式更新的做法。具体来说,瀑布式更新的框架如下:

\begin{figure*}
    \begin{mdframed}
    \centering
    \textbf{瀑布式更新协议:}
        \raggedright
        \paragraph{符号约定:} 记已有亚线性PIR协议为$\Pi$,协议运行于数据库$\db$之上,数据库大小为$\dbsize$。
        \begin{itemize}
            \item 记 $n = \lfloor\log_2{\dbsize}\rfloor$,将数据库按序划分为 $n$ 个大小分别为 $2^n, 2^{n-1}, \dots, 2^1, 2^0$ 的子数据库。若某时刻剩余数据库大小不足某个2的幂次$2^k$,则尝试划分$2^{k-1}$大小。
            \item 对划分生成的一系列子数据库,分别运行PIR协议$\Pi$。
        \end{itemize}
    \end{mdframed}
    \caption{瀑布式更新协议框架}
    \label{fig:checklist}
\end{figure*}

我们分别讨论如何处理数据库的改、删、增操作。

\subsection{处理数据库修改}
\todo{给个协议就好}
处理数据库修改是所有更改中最容易实现的。服务器将需要修改的记录发送给客户端,客户端修改对应Hint的校验值即可。具体地,设修改的记录索引为$\dbidx$,起在数据库中对应块为$\blockidx\coloneqq \lfloor \dbidx/\sqrt{\dbsize}\rfloor$,其值由$\db_\dbidx$修改为$\db_\dbidx'$。对于每个本地Hint $\hint_\hintidx = (\setkey_\hintidx,\sumhint_\hintidx,\randomset_\hintidx,\randomhint_\hintidx)$,客户端检查是否有 $Eval(\setkey_\hintidx, \blockidx) + \sqrt{\dbsize}\cdot \blockidx = \dbidx $。如果为真,则更新$\sumhint_\hintidx \coloneqq \sumhint_\hintidx - \db_{\dbidx} + \db_{\dbidx}', \randomhint_\hintidx \coloneqq \randomhint_\hintidx - Eval(\randomset_\hintidx,\blockidx)\cdot(-\db_{\dbidx} + \db_{\dbidx}')$。

\subsection{处理数据库删除}
\todo{解释为什么做不了,为什么其他协议也有这个问题}
数据库删除需要分两种情况讨论:
\begin{enumerate}
    \item 删除的记录不需要向客户端保密。此时删除仅仅是作为一种释放空间的方式,服务器可以通过将数据修改为某一提前约定的标记值来表示删除。
    \item 删除的记录需要向客户端保密。也就是说,服务器在删除这一记录时,希望客户端此后无法访问被删除的记录。而这在本文所提出的PIR协议中是不可能实现的。具体来说,客户端在离线阶段接收到的Hint里面有极大概率(由正确性保证)包含了被删除的记录,通过查询一个Hint中除了被删除记录以外的其他记录,客户端就可以恢复出被删除的记录。
\end{enumerate}

我们注意到这是亚线性PIR中的一个普遍问题。如何解决这一问题会是很有价值的研究方向。

\subsection{处理数据库插入}
\todo{Checklist与分离/合并方案}
由于我们预先对数据库进行了分块,在插入时我们可以进行倍增操作。具体来说,每次插入都从最小的数据库(大小为$2^0$)开始,如果此数据库已满,就把这个数据库内的全部内容移动至下一个更大的数据库中。我们重新运行每个有所更改的数据库的离线阶段。由倍增的性质,不难证明每次插入的均摊复杂度是对数级别的。

在实践中,我们可以考虑如下更简单的做法:将数据库分为两部分,静态数据与动态数据。于静态数据上运行本文提出的亚线性PIR协议,动态数据上运行任一不需要客户端存储数据库信息的线性PIR协议(如SealPIR\cite{SP:ACLS18}),每过一固定周期或当动态数据因新插入数据达到一定大小时,将静态数据与动态数据合并,重新运行亚线性PIR协议的预处理。
\section{实验与分析}
\label{sec:evaluation}
\subsection{实验设计}
本文实现了单服务器与多服务器两种方案。我们根据$\dbsize$、$\lambda$ 和$\querycount$计算参数,以保证安全级别至少为 $\lambda=128$,并且至少支持$\querycount=\sqrt{\dbsize}\log{\dbsize}$次查询($\dbsize$是数据库中的记录数)。所有测试均在单线程上进行,并取10次运行的平均值。对于双服务器PIR方案,两个服务器的操作是串行执行的。我们测试了这些方案的离线阶段和在线阶段,还考虑了存储成本。一些方案,例如Piano,要求在在线阶段缓存查询结果,这种存储开销被包含在存储成本中。数据传输的时间未包括在时间计算中,但总的通信量会明确给出。测试在配备16核Intel(R) Xeon(R) E-2288G CPU @ 3.70GHz和128GiB RAM的服务器上进行。共计于五个数据库上进行测试,数据库大小从256MiB到32GiB不等,但都由32字节的记录组成。

我们对本文方案的性能进行了全面评估,并与当前领先的双服务器PIR、单服务器PIR和可验证PIR方案进行了对比。由于本文的验证过程设计为可插拔模块,我们测试了两种不同版本:一种是启用验证功能的版本,另一种是未启用验证功能的版本。在评估结果的展示中,“本文(NV)”表示未包含验证过程的本文版本。对于双服务器PIR,我们选择了DPF-PIR \cite{EC:GilIsh14}、TreePIR \cite{C:LazPap23}和APIR \cite{APIR23}作为我们的比较对象。在单服务器协议中,我们的比较对象包括SimplePIR \cite{SimplePIR}、VeriSimplePIR \cite{VeriSimplePIR}和Piano \cite{Piano}。其中,VeriSimplePIR和APIR是可验证PIR方案,这意味着它们能确保恶意服务器响应的完整性。其余方案则基于半诚实服务器的假设。

我们在测试中主要考虑了两个问题:(i) 验证的成本是多少?(ii) 本文的方案与当前领先的PIR方案相比表现如何?

\begin{table*}[]
    \caption{双服务器PIR方案的对比。DPF-PIR和APIR方案没有采用离线-在线模型。在数据库大于8 GiB时,APIR会消耗超过128 GiB的内存,无法完成测试。}
    \label{tab:two-server-evaluation}
    \begin{tabular}{@{}lc|ccc|cccccc@{}}
        \toprule
                  & \multicolumn{1}{l|}{}     & \multicolumn{3}{c|}{离线} & \multicolumn{6}{c}{在线}                                                                                                                      \\ \midrule
                  & \multicolumn{1}{l|}{}     & 通信量(MiB)                     & 存储(MiB)                & 时间(s) & \multicolumn{2}{c|}{通信量(KiB)} & \multicolumn{4}{c}{时间(ms)}                                             \\
                  & \multicolumn{1}{l|}{}     &                              &                            &         & 下载                      & \multicolumn{1}{c|}{上传}  & Query & Answer   & Reconstruct & 共计    \\ \midrule
        本文     & \multirow{5}{*}{128 MiB} & 4.95                         & 9.21                       & 8.28    & 64.07                         & \multicolumn{1}{c|}{80.02}   & 0.04  & 0.09     & 0.02        & 0.15     \\
        本文(NV) &                           & 2.51                         & 4.64                       & 2.08    & 64.04                         & \multicolumn{1}{c|}{16.02}   & 0.03  & 0.06     & 0.00        & 0.09     \\
        DPF-PIR       &                           & 0.00                         & 0.00                       & 0.00    & 0.06                          & \multicolumn{1}{c|}{0.84}    & 0.02  & 45.10    & 0.00        & 45.12    \\
        TreePIR   &                           & 4.27                         & 4.88                       & 1.66    & 128.20                        & \multicolumn{1}{c|}{0.58}    & 0.62  & 0.72     & 0.00        & 1.35     \\
        APIR      &                           & 0.00                         & 0.00                       & 0.00    & 2980.00                       & \multicolumn{1}{c|}{0.50}    & 0.00  & 295.85   & 0.02        & 295.87   \\ \midrule
        本文     & \multirow{5}{*}{512 MiB} & 9.89                         & 18.42                      & 38.20   & 128.07                        & \multicolumn{1}{c|}{160.02}  & 0.09  & 0.25     & 0.04        & 0.37     \\
        本文(NV) &                           & 5.02                         & 9.28                       & 8.77    & 128.04                        & \multicolumn{1}{c|}{32.02}   & 0.05  & 0.15     & 0.00        & 0.21     \\
        DPF-PIR       &                           & 0.00                         & 0.00                       & 0.00    & 0.06                          & \multicolumn{1}{c|}{0.91}    & 0.03  & 185.91   & 0.00        & 185.94   \\
        TreePIR   &                           & 8.53                         & 9.75                       & 6.77    & 256.20                        & \multicolumn{1}{c|}{0.61}    & 1.26  & 1.60     & 0.00        & 2.85     \\
        APIR      &                           & 0.00                         & 0.00                       & 0.00    & 6472.00                       & \multicolumn{1}{c|}{1.00}    & 0.00  & 1231.91  & 0.02        & 1231.93  \\ \midrule
        本文     & \multirow{5}{*}{2 GiB}   & 20.28                        & 37.78                      & 166.70  & 256.07                        & \multicolumn{1}{c|}{320.02}  & 0.23  & 0.69     & 0.08        & 1.00     \\
        本文(NV) &                           & 10.28                        & 19.03                      & 36.31   & 256.04                        & \multicolumn{1}{c|}{64.02}   & 0.14  & 0.38     & 0.01        & 0.53     \\
        DPF-PIR       &                           & 0.00                         & 0.00                       & 0.00    & 0.06                          & \multicolumn{1}{c|}{0.98}    & 0.03  & 710.96   & 0.00        & 710.99   \\
        TreePIR   &                           & 17.50                        & 20.00                      & 28.48   & 512.20                        & \multicolumn{1}{c|}{0.64}    & 2.65  & 3.52     & 0.00        & 6.17     \\
        APIR      &                           & 0.00                         & 0.00                       & 0.00    & 13968.00                      & \multicolumn{1}{c|}{2.00}    & 0.01  & 5247.01  & 0.02        & 5247.03  \\ \midrule
        本文     & \multirow{5}{*}{8 GiB}   & 41.56                        & 77.44                      & 827.72  & 512.07                        & \multicolumn{1}{c|}{640.02}  & 0.49  & 2.07     & 0.16        & 2.72     \\
        本文(NV) &                           & 21.06                        & 39.00                      & 243.44  & 512.04                        & \multicolumn{1}{c|}{128.02}  & 0.33  & 1.12     & 0.02        & 1.47     \\
        DPF-PIR       &                           & 0.00                         & 0.00                       & 0.00    & 0.06                          & \multicolumn{1}{c|}{1.05}    & 0.03  & 2832.52  & 0.00        & 2832.55  \\
        TreePIR   &                           & 35.88                        & 41.00                      & 123.42  & 1024.20                       & \multicolumn{1}{c|}{0.67}    & 5.52  & 7.69     & 0.00        & 13.20    \\
        APIR      &                           & -                            & -                          & -       & -                             & \multicolumn{1}{c|}{-}       & -     & -        & -           & -        \\ \midrule
        本文     & \multirow{5}{*}{32 GiB}  & 85.13                        & 158.63                     & 3838.58 & 1024.07                       & \multicolumn{1}{c|}{1280.02} & 1.30  & 4.73     & 0.53        & 6.56     \\
        本文(NV) &                           & 43.13                        & 79.88                      & 1311.07 & 1024.04                       & \multicolumn{1}{c|}{256.02}  & 1.11  & 2.50     & 0.04        & 3.65     \\
        DPF-PIR       &                           & 0.00                         & 0.00                       & 0.00    & 0.06                          & \multicolumn{1}{c|}{1.12}    & 0.03  & 11413.70 & 0.00        & 11413.73 \\
        TreePIR   &                           & 73.50                        & 84.00                      & 510.92  & 2048.20                       & \multicolumn{1}{c|}{0.70}    & 11.33 & 16.86    & 0.00        & 28.19    \\
        APIR      &                           & -                            & -                          & -       & -                             & \multicolumn{1}{c|}{-}       & -     & -        & -           & -        \\ \bottomrule
    \end{tabular}
\end{table*}

\begin{table*}[]
    \caption{单服务器PIR方案的对比。VeriSimplePIR在数据库大于8GiB时以及SimplePIR在数据库大于32GiB时消耗超过128 GiB的内存,无法完成测试。VeriSimplePIR和SimplePIR的离线计算是针对数据库进行的,并且可以供多个客户端重复使用。}
    \label{tab:single-server-evaluation}
    \begin{tabular}{@{}cc|ccc|cccccc@{}}
        \toprule
                      &                           & \multicolumn{3}{c|}{离线} & \multicolumn{6}{c}{在线}                                                                                                                    \\ \midrule
                      &                           & 通信量(MiB)                     & 存储(MiB)                & 时间(s)  & \multicolumn{2}{c|}{通信量(KiB)} & \multicolumn{4}{c}{时间(ms)}                                          \\
                      &                           &                              &                            &          & 下载                      & \multicolumn{1}{c|}{上传}  & Query  & Answer & Reconstruct & 总计  \\ \midrule
        本文         & \multirow{5}{*}{128 MiB} & 128.00                       & 23.89                      & 21.43    & 0.06                          & \multicolumn{1}{c|}{72.02}   & 0.04   & 0.05   & 0.00        & 0.09   \\
        本文(NV)     &                           & 128.00                       & 14.10                      & 10.02    & 0.03                          & \multicolumn{1}{c|}{8.01}    & 0.03   & 0.02   & 0.00        & 0.05   \\
        SimplePIR     &                           & 41.00                        & 41.00                      & 20.79    & 41.00                         & \multicolumn{1}{c|}{41.00}   & 14.26  & 9.49   & 2.20        & 25.95  \\
        VeriSimplePIR &                           & 231.17                       & 101.25                     & 285.06   & 50.63                         & \multicolumn{1}{c|}{50.57}   & 25.99  & 15.40  & 8.52        & 49.91  \\
        Piano         &                           & 128.00                       & 20.92                      & 42.29    & 63.99                         & \multicolumn{1}{c|}{8.02}    & 0.54   & 0.16   & 0.00        & 0.70   \\ \midrule
        本文         & \multirow{5}{*}{512 MiB} & 512.00                       & 49.94                      & 90.81    & 0.06                          & \multicolumn{1}{c|}{144.02}  & 0.08   & 0.15   & 0.00        & 0.23   \\
        本文(NV)     &                           & 512.00                       & 29.61                      & 42.26    & 0.03                          & \multicolumn{1}{c|}{16.01}   & 0.05   & 0.05   & 0.00        & 0.10   \\
        SimplePIR     &                           & 84.00                        & 84.00                      & 89.42    & 84.00                         & \multicolumn{1}{c|}{84.00}   & 29.38  & 42.12  & 4.33        & 75.83  \\
        VeriSimplePIR &                           & 488.43                       & 212.50                     & 1231.67  & 106.25                        & \multicolumn{1}{c|}{106.02}  & 53.81  & 66.14  & 17.65       & 137.59 \\
        Piano         &                           & 512.00                       & 43.46                      & 245.58   & 127.99                        & \multicolumn{1}{c|}{16.02}   & 1.04   & 0.30   & 0.00        & 1.34   \\ \midrule
        本文         & \multirow{5}{*}{2 GiB}   & 2048.00                      & 108.68                     & 392.20   & 0.06                          & \multicolumn{1}{c|}{288.02}  & 0.22   & 0.52   & 0.00        & 0.73   \\
        本文(NV)     &                           & 2048.00                      & 64.67                      & 176.90   & 0.03                          & \multicolumn{1}{c|}{32.01}   & 0.14   & 0.14   & 0.00        & 0.28   \\
        SimplePIR     &                           & 169.00                       & 169.00                     & 363.72   & 169.00                        & \multicolumn{1}{c|}{169.00}  & 58.93  & 161.94 & 8.52        & 229.39 \\
        VeriSimplePIR &                           & 984.05                       & 424.75                     & 4864.49  & 212.38                        & \multicolumn{1}{c|}{212.16}  & 107.59 & 263.29 & 34.94       & 405.83 \\
        Piano         &                           & 2048.00                      & 93.93                      & 759.98   & 255.99                        & \multicolumn{1}{c|}{32.02}   & 1.91   & 0.55   & 0.00        & 2.46   \\ \midrule
        本文         & \multirow{5}{*}{8 GiB}   & 8192.00                      & 227.48                     & 1620.31  & 0.06                          & \multicolumn{1}{c|}{576.02}  & 0.52   & 1.46   & 0.00        & 1.98   \\
        本文(NV)     &                           & 8192.00                      & 135.66                     & 783.71   & 0.03                          & \multicolumn{1}{c|}{64.01}   & 0.45   & 0.46   & 0.00        & 0.91   \\
        SimplePIR     &                           & 344.00                       & 344.00                     & 1612.35  & 344.00                        & \multicolumn{1}{c|}{344.00}  & 118.35 & 684.98 & 18.21       & 821.54 \\
        VeriSimplePIR &                           & -                            & -                          & -        & -                             & \multicolumn{1}{c|}{-}       & -      & -      & -           & -      \\
        Piano         &                           & 8192.00                      & 196.20                     & 3319.58  & 511.99                        & \multicolumn{1}{c|}{64.02}   & 3.82   & 1.13   & 0.00        & 4.95   \\ \midrule
        本文         & \multirow{5}{*}{32 GiB}  & 32768.00                     & 475.07                     & 6811.69  & 0.06                          & \multicolumn{1}{c|}{1152.02} & 1.17   & 3.29   & 0.00        & 4.46   \\
        本文(NV)     &                           & 32768.00                     & 283.87                     & 3117.62  & 0.03                          & \multicolumn{1}{c|}{128.01}  & 0.71   & 0.94   & 0.00        & 1.66   \\
        SimplePIR     &                           & -                            & -                          & -        & -                             & \multicolumn{1}{c|}{-}       & -      & -      & -           & -      \\
        VeriSimplePIR &                           & -                            & -                          & -        & -                             & \multicolumn{1}{c|}{-}       & -      & -      & -           & -      \\
        Piano         &                           & 32768.00                     & 409.08                     & 13490.68 & 1023.99                       & \multicolumn{1}{c|}{128.02}  & 7.64   & 2.36   & 0.00        & 10.00  \\ \bottomrule
    \end{tabular}
\end{table*}
\subsection{结果分析}

\paragraph{双服务器方案}
双服务器PIR方案的基准测试结果如表 \ref{tab:two-server-evaluation} 所示。比较本文两个版本,可以明显看出,验证功能会导致离线计算增加约3-4倍,在线计算增加2倍,通信和存储成本翻倍。

在这些方案中,本文的方案在在线计算方面超越了当前领先的亚线性方案TreePIR~\cite{C:LazPap23},且的总体通信量更小。这种优越的性能主要源于我们方案简化的查询构造以及对虚构查询的消除。在线阶段不再需要昂贵的wPPRF和虚构查询,显著提高了性能。关于离线阶段,本文方案通过使用一个主密钥表示集合,通信成本稍低,但计算成本稍高,主要原因是缺少AVX指令的支持。

DPF-PIR~\cite{EC:GilIsh14} 是一个线性PIR方案,这意味着其在线计算时间随着数据库大小线性增长。对于32 GiB的数据库,DPF-PIR执行一次查询需要超过10秒。而相比之下,我们的方案在执行相同操作时只需不到10毫秒。这一显著的性能差异强调了使用亚线性PIR方案的重要性,尤其是在处理大规模数据库时,它能显著提高查询效率,减少计算时间。

APIR \cite{APIR} 是目前领先的双服务器可验证PIR方案。本文的方案在所有数据库大小中均在计算效率上超越了APIR。此外,基于Merkle证明的APIR导致数据库大小显著增加,这使其受限于内存不足问题,无法在大于8 GiB的数据库上运行。

\paragraph{单服务器方案}
单服务器PIR方案的测试结果如表 \ref{tab:single-server-evaluation} 所示。

Piano~\cite{Piano} 是当前最先进的单服务器亚线性方案。Piano要求服务器计算$\sqrt{\dbsize}$个答案以隐藏被查询的索引,因此增加了计算和通信成本。因此,本文的方案在计算和通信上均优于Piano,在线计算速度提高了6倍,在线上传量相同,但在线下载量显著减少。

在离线阶段,Piano和本文方案都采用将数据库流式传输到客户端的策略,导致类似的通信成本。理论上,这两个方案在离线计算和存储成本上相似。表中的差异主要归因于实现细节。

与线性单服务器PIR方案SimplePIR~\cite{SimplePIR}相比,本文方案在在线计算上显示出了显著的改进。SimplePIR需要线性数量的操作来计算答案,而本文方案仅需要亚线性的操作数量。这种差异在大规模数据库中特别明显,SimplePIR的在线计算时间是本文方案的500倍以上。

与领先的单服务器可验证PIR方案VeriSimplePIR~\cite{VeriSimplePIR}相比,本文方案的在线上传量略大,但在线下载量却小得多,在线计算成本大约减少了500倍。表中有两个有趣的点。首先,SimplePIR和VeriSimplePIR导致了相当大的数据库膨胀,这意味着它们对服务器RAM提出了显著的挑战。其次,SimplePIR和VeriSimplePIR的客户端存储成本实际上高于Crust。

\begin{figure*}
    \begin{subfigure}{0.5\textwidth}
        \centering
        \includegraphics[width=0.8\linewidth]{figure/cost_2gb.png}
        \caption{2 GiB}
    \end{subfigure}%
    \begin{subfigure}{0.5\textwidth}
        \centering
        \includegraphics[width=0.8\linewidth]{figure/cost_8gb.png}
        \caption{8 GiB}
    \end{subfigure}%
    \caption{单服务器PIR方案在财务成本方面的对比。VeriSimplePIR的8 GiB是估计得到的。}
    \label{fig:single-server-cost}
\end{figure*}

\paragraph{财务成本}
为了展示本文方案在实际场景中的实用性,我们在表 \ref{fig:single-server-cost} 中提供了本文方案和其他方案的财务成本估算。我们以客户端直接下载并存储数据库为基准。图中的数据以相对于该基准方案的成本比率显示。成本是基于AWS的定价\footnote{每GiB下载费用为0.05美元,每GiB数据存储在EBS gp3上的费用为0.08美元,c7a.2xlarge实例的每小时CPU费用为0.41美元。},针对之前评估中的2 GiB和8 GiB数据库计算的。成本按每月每个客户端计算。我们假设客户端每月运行一定数量的查询,并且至少运行一次完整的离线阶段。由于SimplePIR和VeriSimplePIR的离线计算不是按每个客户端计算的,因此离线计算成本不包含在内。

从之前的对比中,我们可以看到本文方案的在线上传量略大,但在线下载量却低得多。因此,本文方案在成本方面非常具有竞争力。当客户端每月运行超过$2^{12}$次查询时,本文方案比SimplePIR更具成本效益,并且在查询次数少于$2^{18}$时,本文方案的成本仍然低于直接存储数据库的方案。当数据库规模增大时,本文方案的优势保持到了更大的查询规模。这一结果清楚地表明,本文方案是一种在实际场景中,尤其是在客户端存储资源有限的情况下,提供PIR服务的可行方案。
\section{本章小结}

在本章中,本文介绍了一种新的亚线性PIR方案。本章提出了可验证PIR的详细定义,并且讨论了如何将验证过程融入进PIR协议中,将其转化为可验证的PIR方案。本章还讨论了PIR的更新问题,提出了一种瀑布式更新的方案。关于本章提出的方案,我们进行了详细的测试和对比,结果表明本章提出的方案在大数据库、多查询次数的场景下,相较于现有的亚线性PIR方案有着更好的性能与经济性。

